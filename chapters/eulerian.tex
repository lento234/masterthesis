\chapter{Eulerian Domain: Finite Element Method}

%------------------------------------------------------------------------------------------------------
%------------------------------------------------------------------------------------------------------
%------------------------------------------------------------------------------------------------------
%\section{Purpose of eulerian domain}
%
%\subsection{Generation of vorticity}


%------------------------------------------------------------------------------------------------------
%------------------------------------------------------------------------------------------------------
%------------------------------------------------------------------------------------------------------
\section{Introduction to Finite Element Method}


\subsection{Finite element discretization}


\subsection{Finite element function and function space}


%------------------------------------------------------------------------------------------------------
%------------------------------------------------------------------------------------------------------
%------------------------------------------------------------------------------------------------------
\section{Solving the Finite Element problem}

\subsection{Introduction to FEniCS Project}

\subsection{Mesh generation using GMSH}


%------------------------------------------------------------------------------------------------------
%------------------------------------------------------------------------------------------------------
%------------------------------------------------------------------------------------------------------
\section{Solving Incompressible Navier-Stokes Equations}

\subsection{Velocity-pressure formulation}

\subsection{Incremental pressure correction scheme}

\subsection{Determining the vorticity field}

\subsection{Determining the body forces}

\subsubsection*{Frictional Forces}

\subsubsection*{Pressure Forces}


\section{Validation of eulerian method}

%\subsection{Lamb-oseen vortex}

\subsection{Clercx-Bruneau dipole collison at $Re=625$}

\subsection{Impulsively started cylinder at $Re=550$}

\section{Summary}

