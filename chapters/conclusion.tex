\chapter{Conclusion and Recommendation}
\label{ch:ConclusionandRecommendation}

\section{Conclusion}

\subsection{Lagrangian domain}

\subsection{Eulerian domain}

\subsection{Hybrid method}

\begin{itemize}
\item Sub-stepping: This observation states that the linear interpolation used for sub-stepping process, has potential for improvement. A possible solution might be to employ a higher-order interpolation method for determining the Eulerian Dirichlet boundary condition at the sub-steps.

\item Adaptive discreetization of blobs: In conclusion, we see that a high resolution discretization of the Lagrangian method inside the Eulerian domain $\Omega_L \cap \Omega_E$ is paramount for accurate transfer of information to and from the Eulerian method. For a lower resolved Lagrangian method in this region introduces artificial vorticity at the boundary of the Eulerian domain $\Sigma_d$, corrupting the solution of the coupling.

\end{itemize}

%\subsection{Feasibility of hybrid vortex method for compressor cascade}

\section{Recommendations}

\subsection{Lagrangian domain}

\subsection{Eulerian domain}

\subsection{Hybrid method}

%\subsection{RBF kernel representation of boundary}
%
%\subsection{Recommended numerical simulation for compressor cascade}
