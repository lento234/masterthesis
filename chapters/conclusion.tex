\chapter{Conclusion and Recommendation}
\label{ch:ConclusionandRecommendation}

\section{Conclusion}

\subsection{Lagrangian domain}

\subsection{Eulerian domain}

\subsection{Hybrid method}

%\begin{itemize}

%\end{itemize}

\section{Recommendations}

\subsection{Lagrangian domain}


% adaptive discretization
%Adaptive discretization of blobs: In conclusion, we see that a high resolution discretization of the Lagrangian method inside the Eulerian domain $\Omega_L \cap \Omega_E$ is paramount for accurate transfer of information to and from the Eulerian method. For a lower resolved Lagrangian method in this region introduces artificial vorticity at the boundary of the Eulerian domain $\Sigma_d$, corrupting the solution of the coupling. It is recommended that a further focused study should be performed on the artificial vorticity generated from the boundary of Eulerian domain $\Sigma_d$. If this artificial vorticity can be further minimized, we could potential attain more accurate results.


% turbulent flow

\subsection{Eulerian domain}
%explicit time marching scheme, \indexAcron{Forward Euler}{FE} 

% laminar flow -> turbulent flow

\subsection{Hybrid method}

% better sub-stepping
%Sub-stepping: This observation states that the linear interpolation used for sub-stepping process, has potential for improvement. A possible solution might be to employ a higher-order interpolation method for determining the Eulerian Dirichlet boundary condition at the sub-steps.


% moving geometry

% RBF kernels representation of boundary

%\subsection{RBF kernel representation of boundary}

