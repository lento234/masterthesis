\chapter{Introduction}
\label{ch:Introduction}

We, the humankind, are now facing several challenges in finding a cheap and reliable energy source. Conventional energy resources such as fossil fuels and nuclear energy are not only limited supply but also pose adhere effects on the environment. Therefore, we are striving to find a cheap and renewable source of energy. Wind energy is such source of energy and is therefore getting more popular and also become more affordable and novel renewable technologies such as \printAcron{Vertical-Axis Wind Turbine}{VAWT} is now an interested research field.\\
\index{VAWT}

Vertical-Axis Wind turbines are unlike the normal wind turbine. Typical wind turbines are mounted on a mast away from the ground and generates energy by spinning normal to the ground. However, a VAWT spins parallel to the ground with its hub located at the ground \cite{website:wikiVAWT}. The advantages of the vertical axis wind turbine are what makes them ideal for a source of renewable energy.  As the turbine is located at the ground (unlike the Horizontal-Axis Wind Turbine), it is easily accessible and can be easily maintained. The second main advantage of the VAWT is the way it dissipates its wake \cite{Ferreira} \cite{Vermeer2003}. As the fluid past the turbine is more turbulent, the flow is able to smooth out much earlier. This means that it possible to places VAWTs much closer to each other is so in future this means that a VAWT farm can potentially give more power per area. Furthermore, operate independent of the flow direction and can operate at low wind speeds (low tip-speed ratios).\\
\index{HAWT}

\section{Motivation}
However, with these advantages also comes drawbacks. As the blades passes through its own dirty air (the wake), complex wake-body interactions take places. These have adhere effect on the blade structure and therefore is more susceptible to fatigue. This happens because the blades are constantly pitching in front the free-stream flow and complex flow behaviour such as dynamic stall and constant vortex shedding occurs \cite{SimaoFerreira2008}. This complex fluid behaviours makes it hard to predict the performance of a VAWT and this is one of the reasons why VAWTs are not mainstream. In addition, as the VAWT operates at large Reynolds number, accurate numerical methods are computationally very expensive. Therefore, it is vital to have a good understanding of the flow structure evolution and the wake generation of the VAWT using not only an efficient method, but also an accurate one.\\

\section{Research Aim}

%They key interest of this project is the develop a efficient, reliable and an accurate fluid dynamics solver for the modelling the flow around a 2D VAWT. This numerical tool should not only be accurate at capturing the small scale phenomenons such as vortex shedding, dynamic stall and wake-body interaction, but should be efficient at capturing the large scale phenomenons such as the wake evolution of the VAWT. This is one of the main challenges of this project because if one is interested in the phenomena of one scale, it will compromise the other. Therefore, one of the main goals of the project will be accurate development of method and for modelling fluid around multiple moving geometries of non-trivial shapes. To verify the accurate implementation, various benchmark cases will be investigated and finally will be validated against numerical and experimental data of VAWT.

%To overcome this dilemma, we are going to employ a hybrid numerical method which couples the accurate Navier-Stokes grid solver near the body and the efficient vortex particle method at the domain away from any boundaries. This hybrid vortex particle method, will be not only be able to capture the small scale phenomenons of the boundary but can also be enabled to be parallelized making ideal in understand the multiple wake-body interactions and computation of large-scale phenomenons of wind farms. Such methodology has been already been investigated \cite{Daeninck2006} \cite{Cheng1997} \cite{Ould-Salihi2001} \cite{Zhao2007} \cite{Stock}, however has yet not been used for VAWT problem.\\

\subsection*{Research Goal}
To summarize, we are now able to formulate a research goal. The key interest of this project is to develop an efficient, reliable, and an accurate numerical method for modelling the flow around a 2D VAWT. For now, only 2D problems are considered because 3D method is build upon the methodology of the 2D. Thus, once the 2D methodology is made, a 3D numerical method should be a straightforward extension.

Furthermore, the numerical method efficient at capturing both the near-wake phenomenons such as the vortex shedding, dynamic stall, \& the wake-body interaction, and should be able to capture the large scale flow structure such as the evolution of the VAWT wake. From this criterias, we are able to formulate the research question.

\paragraph*{Research Question:} \textit{Is it possible to develop a numerical method that is both efficient at capturing the small-scale phenomenons and the large scale phenomenons? Is is possible to apply this to a 2D VAWT?}

Similarly, the research aim of this thesis has also been summarized.

\paragraph*{Research aim and plan:}
\begin{itemize}
\item Develop a numerical method for capturing small-scale phenomenons and large scale phenomenons.
\item Ensure this tool is efficient, reliable, and accurate.
\item Verify, Validate the tools with model problems.
\item Apply the model to the 2D flow of VAWT.
\end{itemize}

With the above formulate research question, aim and plan we are able to thoroughly perform the literature study to determine whether the research goal stated here is feasible. Finally, this report will answer why a Hybrid Eulerian-Lagrangian Vortex Particle Method will be used to the achieved the goals. 

%Vertical-Axis Wind Turbines or VAWTs are unlike the typical wind turbines (Horizontal-Axis Wind Turbine). The turbine not only spins in a different axis but aerodynamically, the flow structure around the wind turbine is highly unsteady, and complex flow phenomenons such as vortex shedding and dynamic stall occurs around the aerofoil. Futhermore, far behind the VAWT, the wake evolution is still key topic of research \cite{SimaoFerreira2008} and therefore an efficient and accurate numerical tool is sought out.\\

%The key interest of this project to develop an efficient, accurate and reliable numerical solver that can model the full flow problem of a VAWT. This numerical approach should not only excel at capturing the small scale structures near the body like dynamic stall but should also be efficient at modelling the far wake behind the turbine. To tackle this challenge, it has been decided that a hybrid vortex particle method should be used.\\

%The hybrid vortex particle method is a coupled Navier-Stokes grid solver and a vortex particle method where they solve the flow near the body and away from the body respectively. However, before stumbling into the development of the tool, several choices have to be made in regard to the ideal techniques that should be employed. Therefore, this paper will start of with the general introduction to all the approaches, compare them and finally try to reach a conclusion on what might be the ideal choices for further research.\\

\section{Thesis Outline}


%\section{Research question}
%\label{sec:ResearchQuestion}
%
%\section{Research objective}
%\label{sec:ResearchObjective}
%
%\section{Importance of study}
%
%\section{Scope of thesis}
%\label{sec:scope}
%
%\section{Structure of the report}
%\label{sec:Structure of the report}


%%%%%%%%%%%%%%%%%%%%%%%%%%%%%%%%%%%%%%%%%%%%%%%%%%%%%%%%%%%%%%%%%%%%%%%%%%%%%%%%%%%%%%%%%%%%%%%
%\nomenclature[am]{MNQC}{Multi-Nozzle Quite Combustor}
%\nomenclature[ag]{GE}{General Electric (Company)}
%\nomenclature[ac]{CRN}{Chemical Reaction Network}
%\nomenclature[ac]{CFD}{Computational Fluid Dynamics}
%\nomenclature[an]{NOx}{Nitrogen Oxides}
%\nomenclature[aa]{ANSYS}{Engineering simulation Software (software)}
%\nomenclature[af]{FLUENT}{ANSYS computation fluid dynamics package (software)}