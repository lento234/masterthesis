\nonumchap{Summary}
    %
	The wake geometry of a \printAcron{Vertical Axis Wind Turbine}{VAWT} is unlike the standard \printAcron{Horizontal Axis Wind Turbine}{HAWT}. The blades of the turbine continuously passes through its own wake, creating complex wake-body interactions such as flow separation and dynamic stall, and convectional grid-based numerical method which is capable of describing such near-body phenomena fails at efficiently resolving the wake geometry. However, as these phenomena have a direct impact on the performance of the VAWT, it is paramount that there exists a numerical method that is not only capable of accurately resolving the small-scale near-body phenomena but also excels at efficiently resolving the unsteady large-scale wake geometry.
		
	%and a numerical method that can accurate model such phenomena fails at efficiently evolving the wake. On the other hand, numerical methods that are efficient at describing the wake of the VAWT fails at accurately describing the near-wake phenomena such as the flow separation.
	
	%Thus goal of the research was to develop a numerical method that is not only capable of accurately describe such near-wake phenomena but is also capable of efficiently evolving the large-scale wake structures.
	
	This was the goal of the research, and the numerical method that satisfied these requirements was a domain decomposition method known as the \printAcron{Hybrid Eulerian-Lagrangian Vortex Particle Method}{HELVPM}, based on the doctoral thesis of Daeninck \cite{Daeninck2006} and the additional study performed by Stock \cite{Stock2010a}. In the present study, we coupled an Eulerian Finite Element Method, which only resolves the near-body domain, with a Lagrangian Vortex Particle Method, which resolves the entire wake. The advantage of the fluid domain segregation was that the Eulerian method could focus on accurately describing the near-body features whereas the Lagrangian method could focus on efficiently evolving the wake using simulation acceleration methods such as \printAcron{Fast Multipole Method}{FMM} and parallel computation in \printAcron{Graphical Processing Units}{GPU}. 
	
	%The numerical method that we developed coupled a Vortex Particle Method (Lagrangian method) with a grid based Finite Element Method (Eulerian method).

	The present study initially developed, verified and validated the Finite Element method and the Vortex Particle method separately ensuring it performs according to the theory. These methods were then coupled using the algorithm of Daeninck \cite{Daeninck2006} and Lagrangian correction strategy developed by Stock \cite{Stock2010a}. However, during the study we determined that additional modifications to the coupling strategy was required to ensure conservation of circulation. Furthermore, it was determined that the spatial resolution of numerical method at the overlap region, where coupling takes place, plays a crucial role in the accuracy of the coupling. 
	
	Even though this hybrid method sacrificed some efficiency to ensure an accurately coupled scheme, we must not that it is still at its infancy. With the help of advanced techniques such as varying particle core size, higher order time marching scheme in the Eulerian method, and boundary element method acceleration techniques such as FMM and/or GPU calculation, the hybrid method has the potential to substantially outperform standard grid based methods. In conclusion, the hybrid method that has been developed here has the potential to accurately describing the near-wake phenomena and efficiently evolving the wake. 

%The proof-of-concept test cases 
%%	
%%	
%%	 	 	 
%%	In conclusion, even though the present implementation of the hybrid method is comparatively slower due to the need for high spatial resolution at the overlap through advanced techniques such as spatially varying core sizes for vortex blobs, and FMM/GPU accelerated methods for the boundary element method can enable the hybrid method to outperform any numerical methods.
%%	%The resulting wake geometry of VAWT becomes difficult to model and makes it hard to predict the performance of the VAWT. Therefore, the numerical method should be able to accurately simulate the near-body flow phenomena and should be efficient at evolving the wake. 
