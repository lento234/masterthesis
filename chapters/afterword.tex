\nonumchap{Afterword}

This section is written for the purpose of giving an insight into the development of the hybrid method. This section summarizes my personal experience, the issues I had to tackle, and the lessons I have learned during the development of the hybrid method presented in this paper. The author hopes this personal note provides somewhat of a guideline in tackling the development of such method.

The work presented here spanned a much larger time period than originally indented. The work was extended due to challenges I faced during the master thesis but also some of the challenges outside work. I realized that for a successful master thesis of within the prescribed time, what matters more is the personal mindset. One must be willing to confront any issues with a positive outlook. I have found out that the support of the supervisors is critical during the master thesis. When you approach a dead end where the challenges seems impossible to tackle, a fresh perspective from someone who as already experienced similar issues is invaluable.

The main problem of developing a hybrid method is that one cannot be a ``Jack of all trades, master of none", but you must have a fundamental understanding in each of the numerical methods of the coupling. Without a deep understanding of the various numerical methods used during the coupling, it will be impossible to determine sources of the issues that emerges. The solution to this problem was the collaboration of the work. The present work was culmination of the work done by the author and the direct, daily supervisor Artur Palha. In additional, the researches done by Bjarnthor in continuation to the present work helped determine some of early overlooked issues of the hybrid method. 

If I were to tackle the challenge of developing the hybrid method again, I would dedicated more time on a single, simpler test case, and ensure that the coupling is performed accurately. Also we see that a challenge in the present method is the artificial vorticity when there is a mismatch in the methods. During the study, these problems where only known after first few critical stages of the master thesis. As a large time was devoted to the development of the sub numerical methods required for the coupling. This was one of the problems of the study, as time had to be spend on understanding, implementing and validating these methods before even tackling the hybrid method. If the issues where known beforehand, focus could have been shifted to determine the proper procedure for coupling.

By the time the coupling of the methods began, the aforementioned problem of the artificial vorticity became a large concern. During the investigation of the literatures, these results were not highlighted and due to this large time was wasted in determining the source of the error. The present paper therefore, tried to be more transparent and highlight any downsides in the numerical method such that reader is fully informed.

However, the present method was successful in the end, but required detailed investigation and development of the modifications to the coupling strategy of the hybrid method. As one could assume, this had a negative effect on the time span of the present study and could have mitigated if the issues where known beforehand. However, in hindsight I believe the work done here was necessary for further development by future researchers.

However, one could argue that this is one of the key challenges in academic research where one is found in unexplored territory. Therefore, from tackling this research the recommendations that I would like to give to future researcher are as follows:
\begin{itemize}
\item Focus on development of only one aspect of the hybrid method, instead of tackling multiple issues. The groundwork provided in the present thesis should enable the future research to perform this challenge.
\item Collaboration is the key to successful development. Collaborating work can share the load of tackling the multiple challenges of the development of the hybrid method. %e with someone who has deep background in development of the hybrid method. As there are many aspects of the method that can influence, experienced person can orientate you in the right direction.
\item Make sure that you are transparent with your work and your results. This enables the an easier continuation of future research.
\end{itemize}

Time was dedicated to be as transparent with the methodology as possible. Furthermore, accurate and eligible documentation of the hybrid method in the code and in the report was at most importance during the thesis. This was done to provide a solid ground work for the further development of the hybrid method for the VAWT.


%Reflecting on the present work, we see that 

%Looking back at the work done here, several recommendations come into mind

%During the initial stages of the research, 

% of tackling the master thesis for making a hybrid numerical method. The section is meant as a reflection of my work and the issues that had to be tackled for the completion of the master thesis with possible recommendation for whoever is going to continue the work laid out here. 


%of the personal reflection of working on the thesis. The section prov


%-summary, reflection, collaboration.
%-lessons learned
% - recommendations
%-insight into developing hybrid methods
%-thoughts, opinions

%-issues dealth, how it was overcome

%- things that would have been useful to know before tackling hybrid methoid


%-letter