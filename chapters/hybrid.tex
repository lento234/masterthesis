\chapter{Hybrid Eulerian-Lagrangian Vortex Particle Method}
\label{ch:helvpm}

% Summarize the sections. : domain decomposition, coordinate systems assosciated to each
% subdomains, and the coupled evolution of the hybrid method.
% Reference to the literatures: Cottet and others, stock and daenick
% We use approach of stock based on the phd research of daeninck

	\section{Introduction to Hybrid Eulerian-Lagrangian Vortex Particle Method}
	
	The \printAcron{Hybrid Eulerian-Lagrangian Vortex Particle Method}{HELVPM} is a domain decomposition method, where the Eulerian method and the Lagrangian method solves different regions of the fluid. The domain decomposition method simply splits the domain of interest and uses appropriate method in each domain. The Eulerian formulation will be used at the near-wall region, where we need proper description of the vorticity generation at the boundary, and the Lagrangian formulation is used away from the body, where we only need to evolve the vorticity field. Figure \ref{fig:domainDecomposition} shows the decomposition of the domain to the gridded and the non-gridded region.
	
		\begin{figure}[!h]
			\centering
			\includegraphics[width=0.6\linewidth]{figures/hybrid/domainDecomposition_typical_type2-crop.pdf}
			\caption{Standard domain decomposition using Schwartz iteration for coupling the two methods. Eulerian subdomain $\Omega_E$ (near the body), and Lagrangian subdomain $\Omega_L$ (away from the body). Figure is based on Guermond (2000) \cite{Guermond2000a}.}
			\label{fig:domainDecomposition}
		\end{figure}	
	
	Several studies have already been done: Cottet and Koumoutsakos (2000a)\cite{Cottet2000a}, Guermond and Lu (2000) \cite{Guermond2000a} simulated the advection dominated flows; Ould-Salhi et al. (2001) \cite{Ould-Salihi2001a} blended the finite difference and vortex method together; Winckelmans et al. (2005a) \cite{Winckelmans2005} investigated the trailing vorticies; Daeninck (2006) \cite{Daeninck2006} used a simplified coupling strategy, coupling Vortex Particle Method and Finite Diference Method; Stock (2010) \cite{Stock2010a} expanded Daeninck's strategy, coupling Vortex Particle Method and Finite Volume Method and modeled a 3D rotor.

	\section{Convectional Coupling Strategy}
	\label{sec:helvpm-ccs}
	
	When investigating the literature works, we see that not all domain decomposition methods are the same. The main difference between the methods is their coupling strategies. Most works employ the\textit{ Schwartz alternating method} to couple the vortex particle method and the grid solver. The Schwartz alternating method (or sometimes referred to as Schwartz iterative method), couples the vortex particle method and the grid solver by iteratively determining the boundary condition such that the stream functions in both domains, $\psi_L$ and $\psi_E$ in $\Omega_L$ and $\Omega_E$ respectively, match at the overlap region $\Omega_E-\Omega_L$, shown in Figure \ref{fig:domainDecomposition}. The summary of a single iteration of the Schwartz alternating method is as follows:
	
		\begin{itemize}
		\item Determine the Eulerian boundary condition, the stream function $\psi_{\Sigma_1}$ at the Eulerian boundary $\Sigma_1$, extracted from the Lagrangian stream function $\psi_L$ in the Lagrangian subdomain $\Omega_L$.
		\item Solve for the stream function $\psi_E$ in the Eulerian subdomain $\Omega_E$ with the new boundary condition $\Sigma_1$.
		\item Determine the Lagrangian condition, the stream function $\psi_{\Sigma_0}$ at the Lagrangian boundary $\Sigma_0$, extracted from the Eulerian stream function $\psi_E$ in the Eulerian subdomain $\Omega_E$.
		\item Solve the stream function $\psi_L$ in the Lagrangian subdomain with the boundary conditions $\psi_{\Sigma_0}$ at the Lagrangian boundary $\Sigma_0$.
		\end{itemize}
	
	This procedure is iterated until the stream functions of both domains converge \cite{Ould-Salihi2001a}. Once the stream function is determined in both the domains, the velocity field can be obtained. Using the velocity field, we can then evolve the vorticity field in the Lagrangian subdomain.

	\section{Simplified Coupling Strategy}
	\label{sec:helvpm-scs}
	
	As we realized now, the downside to this procedure is that we have to solve the stream function in both $\Omega_E$ and $\Omega_L$ iteratively, until we converge to a solution. This makes the computation very expensive, especially when we are dealing with large numbers of vortex particles. Therefore, for this project, we are using the coupling technique that is based on the research work of Daeninck (2006) \cite{Daeninck2006} and Stock (2010) \cite{Stock2010a}. However, through the course of present work, we will see that we have to perform a modification to the scheme, to ensure that the total circulation of the Lagrangian domain is conserved at all times.	
	
%	\subsection{Coupling Eulerian and Lagrangian Methods}
	
	The simplified coupling strategy was first demonstrated in the doctoral thesis of Daeninck \cite{Daeninck2006}. Daeninck showed that it is possible to coupled the Lagrangian and the Eulerian method without the use of Schwartz iterative method. Daeninck proposed this approach from the following statements:
	
	\begin{itemize}
	\item The Lagrangian vortex method solves the full fluid domain $\Omega_L$ (see Figure \ref{fig:domainDecomposition_daenick}), but under-resolves the near-wall region $\Omega_E$ as it is less efficient at resolving the boundary layer of the flow.
	
	\item Eulerian method is used to resolve the near-wall region $\Omega_E$, efficiently capturing the boundary layer features and flow separation.
	
	\item The Lagrangian subdomain in the near-wall region $\Omega_L\cap\Omega_E$ is corrected using the more accurate Eulerian solutions to compensate the aforementioned under-resolution.
	
	\item The boundary conditions for the Eulerian method is directly obtained from the evolved solution of the Lagrangian method.
	\end{itemize}
	
	The grid solver therefore essentially acts as the correction for the under-resolved regions of the Lagrangian method. The Lagrangian vortex method in the full fluid domain focuses only on capturing and efficiently evolving the wake.
	
		\begin{figure}[!h]
			\centering
			\includegraphics[width=0.6\linewidth]{figures/hybrid/domainDecomposition_daenick_type2-crop.pdf}
			\caption{Modified domain decomposition \underline{without} Schwartz alternating method. Lagrangian subdomain extends up to the surface of the body. Figure is based on Daeninck (2006) \cite{Daeninck2006}.}
			\label{fig:domainDecomposition_daenick}
		\end{figure}	
	
	Furthermore, Daeninck's simplified coupling strategy handles the Lagrangian boundary condition differently from the convectional domain decomposition method. In convectional method, the shedding of the vorticity from the wall is also defined in the Lagrangian method as well. However, in Daeninck's strategy, as the Lagrangian method is under-resolved at the boundary, it cannot be used to resolve the vorticity flux at the body. Instead, the Eulerian method is used to solve vorticity generation from the wall boundary, and acts as the vorticity generator for the Lagrangian method. 
	
	For this coupling strategy to be valid, there are some assumptions that we must satisfy:

	\begin{itemize}
	\item At $t_n$ before the evolution of both method to $t_{n+1}$, the Lagrangian solution matches Eulerian solution at the boundary of the near-wall region $\Sigma_d$ (see Figure \ref{fig:domainDecomposition_daenick}).
	\item Even though the Lagrangian subdomain is under-resolved in the near-wall region, it should still be able to provide accurate boundary conditions for the Eulerian method at the external boundary $\Sigma_d$.
	\item After the evolution to $t_{n+1}$, the deviation of the Lagrangian solution (due to lack of vorticity flux at Lagrangian boundary), should be minimal.
	\end{itemize}	
	
	Daeninck's the simplified coupling strategy focused on the vorticity-velocity formulation for the Eulerian domain. However, he briefly showed that it is also possible to couple the Eulerian method with the velocity-pressure formulation. The advantage of using the velocity-pressure formulation is that it will be easier to extend to a 3D problem, unlike the vorticity-velocity formulation for the Eulerian method.
	
	\subsection{Coupling Algorithm}	
	\label{subsec:hybrid-ca}
	The coupling of the solvers was described in one global time stepping algorithm. As the Eulerian methods suffers from a larger stability constraint on the time step, and the Lagrangian time marching is computationally more expensive, a  different time discretization for both methods was employed. The Lagrangian method and the Eulerian method had the time steps $\Delta t_L$ and $\Delta t_E=\Delta t_L/k_E$, respectively, where $k_E$ is the number of Eulerian sub-steps.
	
	Assuming that we known the solutions of both solver at $t_n$, the algorithm for the coupled time marching from $t_n$ to $t_n+\Delta t_L$ for Eulerian method (with velocity-pressure formulation) and the Lagrangian method is summarized as follows:
	
	\begin{enumerate}
	\item At $t_n$, \textbf{correct the Lagrangian solution} in the near-wall region $\Omega_L\cap\Omega_E$ from the Eulerian field, Figure \ref{fig:domainDecomposition_daenick}. The vorticity in $\Omega_E$ is determined by taking the curl of the velocity field of the Eulerian method. The vortex particles strengths are determined by interpolating the vorticity from the Eulerian grid.
	
	\item \textbf{Advance the Lagrangian method} from $t_n$ to $t_{n}+\Delta t_L$, with the corrected Lagrangian solution. Before the evolution, there exists a slip velocity at the solid wall $\Sigma_w$. Therefore, the vortex method needs to enforce the \textit{no-slip} boundary condition at the wall by computing the vortex sheet $\gamma$ that cancels this slip velocity. At the end of the evolution, classic vortex methods diffuse the computed vortex sheet to the particles but in Daeninck's work, it is handled by the Eulerian method.
	
	\item\textbf{ Determine the Eulerian boundary conditions} for the velocity field $\mathbf{u}$ at $t_{n}+\Delta t_L$ from the Lagrangian solution at $t_{n} + \Delta t_L$. The Eulerian method requires the Dirichlet velocity boundary condition at $\Sigma_d$ (the Eulerian Dirichlet velocity boundary). The velocity boundary condition at the wall boundary $\Sigma_w$ for a velocity-pressure formulation is simply the zero slip velocity. 
	
	\item \textbf{Advance the Eulerian method} from $t_n$ to $t_n + \Delta t_L$ using $k_E$ Eulerian substeps. The boundary conditions on $\mathbf{u}$ at each substep is obtained by linear interpolation of the boundary condition at $t_n$ and $t_{n} + \Delta t_L$.
	\end{enumerate}	
	
	To enhance the coupling of the Eulerian and the Lagrangian method, Daeninck further modified the Eulerian solution in the most external region of the Eulerian subdomain $\Omega_E$ from interpolation the Lagrangian solution, and observed that it provided better results. Figure \ref{fig:daeninckInterpolation} the modified adjustments regions used by Daeninck in his work.
	
	\begin{figure}[!t]
		\centering
		\includegraphics[width=0.6\linewidth]{figures/hybrid/daeninckInterpolationRegions.png}
		\caption{The domain decomposition and interpolation regions used by Daeninck \cite{Daeninck2006}. The Eulerian domain is also modified to enchace the coupling of the methods.}
		\label{fig:daeninckInterpolation}
	\end{figure}		
	
	\subsection{Lagrangian Correction Step}
	\label{subsec:hybrid-lcs}
	The coupling strategy demonstrated by Daeninck \cite{Daeninck2006}, was studied and was further extended by Stock \cite{Stock2010a}. Stock's work focused on the overlap region $\Omega_E\cap\Omega_L$ (Figure \ref{fig:domainDecomposition_daenick}) and correction of the Lagrangian solution. Following observations was determined by the work:
	
	\begin{itemize}
	\item Eulerian solution is only assumed to be correct from the body surface $\Sigma_w$ to somewhat inside of the outer Eulerian domain $\Sigma_d$. Therefore, the transfer of the Eulerian solution to the Lagrangian method should take in account of the potential inaccuracy of the Eulerian solution at the outer boundary.
	
	\item The very strong gradient in vorticity (vortex sheet) cannot be efficiently and accurately transfered to the Lagrangian method. This is especially problematic at high Reynolds number flows, and interpolating this vorticity from Eulerian method to Lagrangian method results in numerical problems. Therefore, to avoid the noise in the interpolation, the correction step has to ignore the region very near to the wall.
	\end{itemize}
	
	The resulting Lagrangian correction domain, or the interpolation domain $\Omega_I$, using Stock's coupling approach is shown in Figure \ref{fig:interpolationDomainDefinition}. The interpolation domain $\Omega_I$ is defined with an offset from the Eulerian domain boundaries $\Omega_E: \partial\Omega_E=\Sigma_w \cup \Sigma_d$, Figure \ref{fig:interpolationDomainExpanded}, such that regions of the Eulerian domain that introduces issues with coupling are ignored. The outer boundary of the interpolation domain $\Sigma_i$ is defined with an offset $d_{bdry}$ from the Eulerian Dirichlet velocity boundary $\Sigma_d$ such that potential inaccuracy of the Eulerian solution is ignored, shown in Figure \ref{fig:interpolationDomainCloseup}. Similarly, the inner boundary of the interpolation domain $\Sigma_o$ is defined with an offset $d_{surf}$ from the Eulerian wall boundary $\Sigma_w$ such that the very strong vorticity is ignored. The offsets $d_{surf}$ and $d_{bdry}$  where defined in the order of the Lagrangian vortex particle size.
	
	\begin{figure}[!t]
        \centering
        \begin{subfigure}[b]{0.45\textwidth}
                \includegraphics[width=\textwidth]{figures/hybrid/interpolationDomain/interpolationDomainExpanded-crop.pdf}
                \caption{Definition of the Domains}
                \label{fig:interpolationDomainExpanded}
        \end{subfigure}%
        \qquad %add desired spacing between images, e. g. ~, \quad, \qquad etc.
          %(or a blank line to force the subfigure onto a new line)
        \begin{subfigure}[b]{0.45\textwidth}
                \includegraphics[width=\textwidth]{figures/hybrid/interpolationDomain/interpolationDomainCloseup-crop.pdf}
                \caption{Definition of the boundaries}
                \label{fig:interpolationDomainCloseup}
        \end{subfigure}
        \caption{Definition of the interpolation domain $\Omega_{int}$ for correcting the Lagrangian solution, with boundaries $\Omega_I: \partial\Omega_I=\Sigma_{i}\cup\Sigma_{o}$.}
        \label{fig:interpolationDomainDefinition}
	\end{figure}		

	The resulting Lagrangian correction step employed by Stock is summarized as follows:
	
	\begin{enumerate}
	\item Interpolate the vorticity of the Eulerian method from a non-uniformly structured (or an unstructured grid) onto a temporary uniformly structured Cartesian grid covering the entire Eulerian domain $\Omega_E$. This is done to performed an easier correction of the Lagrangian solution with the Eulerian solution. The interpolation ignores the very strong vorticity present in the boundary layer that could cause numerical problem.
	\item Determine all the particles within the interpolation domain $\Omega_I$ that is to be corrected.
	\item Correct or reset the strengths of the particles using the local particle area and the vorticity interpolated from the temporary structured Cartesian grid.
	\end{enumerate}
	
	Using this approach, Stock demonstrated the feasibility of simulating a 3D compressible flow problem around a sphere at $Re=100$, a finite airfoil at $Re=\num{1.5e6}$, and 4-Bladed advancing rotor at $Re=865,500$.
	
	\section{Evolution of the Hybrid Method}

	In the present work, we will therefore employ Daeninck's simplified coupling strategy with the detailed Lagrangian correction approach of the Stock. The evolution of the hybrid method is classified into four parts and is as follows:

		\begin{enumerate}
		\item \textbf{Correct Lagrangian:} Use the solution of the Eulerian subdomain $\Omega_E$, to correct the solution of the Lagrangian subdomain $\Omega_L$, using the strategy of Stock. Chapter \ref{ch:coupling} provides a detailed investigation on the implementation of Stock's Lagrangian correction strategy. However, during the implementation, we saw that conservation of total circulation in the Lagrangian method is paramount for an accurate correction.
		
		\item \textbf{Evolve Lagrangian:} With the modified solution, evolve the Lagrangian solution from time step $t_n$ to $t_{n}+\Delta t_L$. Chapter \ref{ch:lagrangian} provides the detailed investigation on the theory and the algorithm of the Lagrangian method used for the present work.
		
		\item \textbf{Determine Eulerian boundary conditions:} Use the Lagrangian solution of time $t_{n}+\Delta t_L$ to determine the boundary conditions of the Eulerian subdomain at $t_{n}+\Delta t_L$.
		
		\item \textbf{Evolve Eulerian:} With the boundary condition, evolve the Eulerian solution from $t_n$ to $t_{n}+\Delta t_L$ using $k_E$ Eulerian substeps. Chapter \ref{ch:eulerian} provides the detailed investigation on the theory and the algorithm of the Eulerian method used for the present work.
		\end{enumerate}
	
	Figure \ref{fig:flowchart_simpleCoupling} shows the flowchart of the evolution of the hybrid method. To ensure that the coupling of the hybrid method performs as explained in theory, we required a verification and validation test on the functionality of each segregate methods.
	
	
	\begin{figure}[!t]
		\centering
		\begin{tikzpicture}
			[node distance=.8cm, start chain=going below,]
			\node[punktchain, join] (correct) {Correct the \\Lagrangian subdomain};
		    \node[punktchain, join] (evolveL) {Evolve the Lagrangian solution};
		    \node[punktchain, join] (bcE)     {Determine the \\Eulerian boundary conditions};
		    \node[punktchain, join] (evolveE) {Evolve the Eulerian solution};
		\end{tikzpicture}
		\caption{Flowchart of the simple coupling strategy. The flowchart shows the procedure to evolve both methods from $t_n$ to $t_{n+1}$.}
		\label{fig:flowchart_simpleCoupling}
	\end{figure}




% Summarize daenincks approach:
	% Methodology, algorithm
	% Domain decomposition
% Summarize stock's approach:
	% Methodology, algorithms
	% Resulted domain decomposition.