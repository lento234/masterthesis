\chapter{Development of the Hybrid Vortex Method for 2D VAWT}
%\label{ch:LiteratureReview}

% Comparison of hybrid vortex methods.
% choice of hybrid method. Example domain decomposion, coupling technique

\section{Methodology}
% What is hybrid vortex method?
% What is the general idea behind the hybrid vortex method?
% What does it mean to couple particle solver and grid solver?
% What is the advantage?
% What is the drawback?

\section{Vortex Method}
% All the info on the development of vortex method
% What approach are you using?
% G. Daenick, G. Winckelmans method
\subsection{Blob Discretization}
% Particle initialization
% info on cut-off function
% what are the governing variables? Overlap ratio, grid size, 

\subsubsection{Error analysis of discretization}
% importance of time-shift correction when comparing with Lamb-oseen
% overlap ratio vs. L2-norm and max.relative error
% number of particles/ deltaH vs. L2-norm and max.relative error

\subsection{Vortex blob convection}
% Info on time-stepping scheme
% source of error during convection = lagrangian distortion from fluid strain
% importance of population control

\subsubsection{Remeshing for lagrangian grid distortion}
% show with and without remeshing figure.
% investigate on remeshing frequence -> treatment of vortex diffusion

\subsubsection{Error of time-Stepping}
% Time-step vs. L2-norm and max.relative error
% Show the convergence of time-step scheme

\subsection{Treatment of vortex diffusion with remeshing}
% refer to Daehyun Wee, Ahmed F. Ghoniem 2006
% Brief overview of other methods
% motivation of the choice of modified interpolation kernel, diffusion through remeshing


\subsubsection{Modified interpolation kernel}
% Equation for modified interpolation kernel, 
% M4 kernel
% Stability condition, And also conclusion to c2 parameter. the range for the value 1/6<= c2 <=1/2

\subsubsection{Convergence of modified interpolation kernel}
% Comparison with L.Barba, Phd.Thesis
% evolution of error in vorticity of Lamb-oseen.
% 

%%%%%%%%
\subsection{Treatment of geometry in fluid}

\subsubsection{Vortex sheet for inviscid boundary condition}
% using vortex sheet to solve boundary integral equation
% refer to using RBF kernels as a mesh-less solution, recommendations.

\subsubsection{Convergence of panel method}


%%%%%%%
\section{Coupling grid to particle solver}

\subsection{Coupling algorithm and modified overlap region}


\section{Moving boundaries}
\subsection{Modification to grid solver}
\subsection{Modification to vortex method}


%\section{Former Work}
%\label{sec:FormerWork}
%
%\subsection{Overview of the Work}
%\label{subsec:OverviewoftheWork}

%\section{Purpose of further research}

%%%%%%%%%%%%%%%%%%%%%%%%%%%%%%%%%%%%%%%%%%%%%%%%%%%%%%%%%%%%%%%%%%%%
%\nomenclature[ak]{$K$}{Kelvin (temperature)}
%\nomenclature[ar]{rpm}{Revolutions per minute (frequency)}
%\nomenclature[ac]{CO}{Carbon Monoxide}
%\nomenclature[ac]{CRM}{Chemical Reaction Modelling}
%\nomenclature[ah]{H2}{Molecular hydrogen}
%\nomenclature[ag]{GSP}{Gas Turbine Simulation Program (Software)}
%\nomenclature[rr]{$\rho$}{Density \nomunit{[$kg/{m^3}$]}}
%\nomenclature[sm]{$\dot{m}$}{Mass flow rate \nomunit{[$kg/s$]}}
%\nomenclature[ab]{bar}{Pressure}

%\nomenclature[rr]{$Re$}{Reynolds number \nomunit{[-]}}
%\nomenclature[rw]{$M$}{Mach number \nomunit{[-]}}
%\nomenclature[rw]{$\mu$}{Dynamic viscosity of air \nomunit{[$kg/{s \cdot m}$]}}