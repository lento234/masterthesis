% \iffalse meta-comment
%
% This is file `apacite.tex'.
%
% Copyright 1994-2005 Erik Meijer and any individual authors listed
% elsewhere in this file.  All rights reserved.
%
% This file is part of the `apacite' package.
% -------------------------------------------
%
% It may be distributed and/or modified under the
% conditions of the LaTeX Project Public License, either version 1.3
% of this license or (at your option) any later version.
% The latest version of this license is in
%   http://www.latex-project.org/lppl.txt
% and version 1.3 or later is part of all distributions of LaTeX
% version 2003/12/01 or later.
%
% The list of all files belonging to the `apacite' package, with a
% short description, is given in the file `manifest.txt'.
%
% \fi

%% apacite.tex : User's manual and test of the apacite package
%% Written by Erik Meijer <e.meijer@eco.rug.nl>
%% This version: [2005/06/08]
%%
%% See the README file for instructions how to process this file.

%%%%%%%%%%%%%%%%%%%%%%%%%%%%%%%%%%%%%%%%%%%%%%%%%%%%%%%%%%%%%%%%%%%%%%%
%%% BASIC DOCUMENTCLASS AND LOADING OF PACKAGES
%%%%%%%%%%%%%%%%%%%%%%%%%%%%%%%%%%%%%%%%%%%%%%%%%%%%%%%%%%%%%%%%%%%%%%%
\documentclass{article}

%%% The following line can be uncommented to test compatibility with TeX4ht.
%\usepackage{tex4ht}

%%% The following line is used for references to internet sources.
%%% It is not generally necessary for apacite, but used in the examples.
\usepackage{url}

%%% The following lines can be uncommented to test compatibility with
%%% the mentioned other packages and test the language support of
%%% apacite.
%\usepackage[dvips,pagebackref]{hyperref}% citations are not links here
%\usepackage[english]{babel}% babel must be loaded before apacite
%\usepackage{german}

%%% Loading apacite with (or without) author index option.
%\usepackage[tocindex,hyper]{apacite}
\usepackage[tocindex]{apacite}
%\usepackage{apacite}

%%% More compatibility tests with other packages.
%\usepackage{backref}% the order of backref vs. apacite seems irrelevant
%\usepackage[dvips,pagebackref]{hyperref} % 1991a does not work here

%%% Explicitly check loading of backref. This information
%%% is used at the end of the document to change a few settings in order
%%% to prevent some errors. This explicit check is more convenient than
%%% simply defining the corresponding \if..., because the \usepackage's
%%% above can now be commented or uncommented without having to worry
%%% about these side-effects.
\makeatletter
\newif\ifbackrefloaded
\AtBeginDocument{%
  \@ifpackageloaded{backref}{\backrefloadedtrue}{\backrefloadedfalse}%
}
\makeatother

%%%%%%%%%%%%%%%%%%%%%%%%%%%%%%%%%%%%%%%%%%%%%%%%%%%%%%%%%%%%%%%%%%%%%%%
%%% DOCUMENT-SPECIFIC DEFINITIONS THAT MAY BE USEFUL FOR WIDER USE.
%%%%%%%%%%%%%%%%%%%%%%%%%%%%%%%%%%%%%%%%%%%%%%%%%%%%%%%%%%%%%%%%%%%%%%%
% Corporate author
\newcommand{\bibcorporate}[1]{#1}

% Echo argument literally. Can be used to define a certain author type
% (``Producer'') in a place where \BibTeX{} expects a first name.
% Using this command between a pair of braces makes sure that the text
% is not abbreviated into an initial.
\newcommand{\bibliteral}[1]{#1}

% If the argument is a period (`.'), \BibTeX{} puts no period after it.
\newcommand{\bibnodot}[1]{}

% For ``On cdtitle [CD]'' etc., to be used instead of ``In''.
\newcommand{\BOn}{On}

% Reverse order. Can be used for various reasons.
\newcommand{\SwapArgs}[2]{#2#1}

% The following macro is esp. useful if the year field must be different
% in text and in reference list. E.g., if 1992/1993 in text and 1992
% in reference list. Then year can be defined as \bibreftext{1992}{1992/1993}.
% The definition below displays only the second argument (``text'').
% Just before the reference list, this can then be changed to
% display only the first argument (``ref'').
\newcommand{\bibreftext}[2]{#2}

% The same command, but now intended for Dutch-style ``von'' parts.
\newcommand{\Dutchvon}[2]{#2}

% The following macro changes the definitions of the macros \BED and \BEDS
% to the argument and then immediately changes the definitions back to their
% old definitions. So it makes a one-time change. Can be used to use
% ``Producer'' instead of ``Ed.'' etc.
\newcommand{\bibeditortype}[1]{\relax}%
\def\renewbibeditortype{%
  \def\oldBED{}%
  \def\oldBEDS{}%
  \def\bibeditortype##1{%
    \global\let\oldBED\BED
    \global\let\oldBEDS\BEDS
    \global\def\BED{##1\global\let\BED\oldBED  \global\let\BEDS\oldBEDS}%
    \global\def\BEDS{##1\global\let\BED\oldBED \global\let\BEDS\oldBEDS}%
  }%
}

% The following macro changes the definition of the macro \BIn to the
% argument and then immediately changes the definition back to the old
% definition. So it makes a one-time change. Can be used to use ``On''
% instead of ``In'' for a CD-track.
\newcommand{\bibInstring}[1]{%
  \global\let\oldBIn\BIn
  \global\def\BIn{#1\global\let\BIn\oldBIn}%
}

% The \bibskipbracenodot macro suppresses the period after a following closing
% brace. It does not work entirely well at the moment. In particular, it
% works differently in an \AX macro in text and an \AX macro in the
% reference list. This has something to do with robust vs. fragile macros
% and/or expansion of the macro. I have experimented a lot with many
% variations, but don't seem to get it right. I don't understand these
% issues well yet.
\makeatletter
\def\swallownext#1{\relax}
\def\swallowdot{%
    \@ifnextchar.%
      {\swallownext}%
      {\relax}%
}

% When the above does not work, \bibskipbracenodot should reduce to nothing.
% This is the current definition. Just before the bibliography, it is
% redefined.
\def\bibskipbracenodot{\noexpand\bibskipbracenodot}


%%%%%%%%%%%%%%%%%%%%%%%%%%%%%%%%%%%%%%%%%%%%%%%%%%%%%%%%%%%%%%%%%%%%%%%
%%% CITATIONS TO THE BIBLE
%%%%%%%%%%%%%%%%%%%%%%%%%%%%%%%%%%%%%%%%%%%%%%%%%%%%%%%%%%%%%%%%%%%%%%%
% The APA manual gives some rules how the Bible and other classic
% works should be cited. These rules are not implemented in standard
% apacite, so I provide a workaround here.

%%% To define which Bible version you used.
\def\bibleversion#1{\gdef\@bibleversion{#1}}
\def\@bibleversion{Revised Standard Version}

\def\bibleciteA#1{%
   % If this is the first citation, add bible version.
   {\@ifundefined{flag@bible}%
      {\fullbibleciteA{#1}}%
      {\shortbibleciteA{#1}}%
   }%
}

\def\fullbibleciteA#1{%
   % Insert relevant chapter and verse.
   {#1}%
   { \BBOP\@bibleversion\BBCP}%
   % Flag that bible has been cited.
   {\global\expandafter\def\csname flag@bible\endcsname{DUMMY}}%
}

\def\shortbibleciteA#1{%
   % Insert relevant chapter and verse.
   {#1}%
   % Flag that bible has been cited (is this appropriate?).
   {\global\expandafter\def\csname flag@bible\endcsname{DUMMY}}%
}

\def\biblecite#1{%
   % If this is the first citation, add bible version.
   {\@ifundefined{flag@bible}%
      {\fullbiblecite{#1}}%
      {\shortbiblecite{#1}}%
   }%
}

\def\fullbiblecite#1{%
   % Open parenthesis.
   {\BBOP}%
   % Insert relevant chapter and verse.
   {#1}%
   % Add bible version.
   {\BBAY{} \@bibleversion}%
   % Close parenthesis.
   {\BBCP}%
   % Flag that bible has been cited.
   {\global\expandafter\def\csname flag@bible\endcsname{DUMMY}}%
}

\def\shortbiblecite#1{%
   % Open parenthesis.
   {\BBOP}%
   % Insert relevant chapter and verse.
   {#1}%
   % Close parenthesis.
   {\BBCP}%
   % Flag that bible has been cited (is this appropriate?).
   {\global\expandafter\def\csname flag@bible\endcsname{DUMMY}}%
}

\def\bibleciteNP#1{%
   % If this is the first citation, add bible version.
   {\@ifundefined{flag@bible}%
      {\fullbibleciteNP{#1}}%
      {\shortbibleciteNP{#1}}%
   }%
}

\def\fullbibleciteNP#1{%
   % Insert relevant chapter and verse.
   {#1}%
   % Add bible version.
   {\BBAY{} \@bibleversion}%
   % Flag that bible has been cited.
   {\global\expandafter\def\csname flag@bible\endcsname{DUMMY}}%
}

\def\shortbibleciteNP#1{%
   % Insert relevant chapter and verse.
   {#1}%
   % Flag that bible has been cited (is this appropriate?).
   {\global\expandafter\def\csname flag@bible\endcsname{DUMMY}}%
}

%%%%%%%%%%%%%%%%%%%%%%%%%%%%%%%%%%%%%%%%%%%%%%%%%%%%%%%%%%%%%%%%%%%%%%%
%%% COMMANDS TO CITE SONGS.
%%%%%%%%%%%%%%%%%%%%%%%%%%%%%%%%%%%%%%%%%%%%%%%%%%%%%%%%%%%%%%%%%%%%%%%
% The APA manual gives some rules how songs should be cited. These
% rules require some tweaking of standard apacite. This is done here.

\newcommand{\bibsong}[4]{}

\newcommand{\citeAsong}[1]{%
  \def\bibsong##1##2##3##4{%
    \APACciteatitle{##3} {\BBOP}{##1}{\BBAY}{##2}{\BBN}{##4}{\BBCP}%
  }%
  \citeauthor{#1}%
  \def\bibsong##1##2##3##4{}%
}

\newcommand{\fullciteAsong}[1]{%
  \def\bibsong##1##2##3##4{%
    \APACciteatitle{##3} {\BBOP}{##1}{\BBAY}{##2}{\BBN}{##4}{\BBCP}%
  }%
  \fullciteauthor{#1}%
  \def\bibsong##1##2##3##4{}%
}

\newcommand{\shortciteAsong}[1]{%
  \def\bibsong##1##2##3##4{%
    \APACciteatitle{##3} {\BBOP}{##1}{\BBAY}{##2}{\BBN}{##4}{\BBCP}%
  }%
  \shortciteauthor{#1}%
  \def\bibsong##1##2##3##4{}%
}

\newcommand{\citesong}[1]{%
  \def\bibsong##1##2##3##4{%
    {\BBOP}\APACciteatitle{##3},{\BAP}{##1}{\BBAY}{##2}{\BBN}{##4}{\BBCP}%
  }%
  \citeauthor{#1}%
  \def\bibsong##1##2##3##4{}%
}

\newcommand{\fullcitesong}[1]{%
  \def\bibsong##1##2##3##4{%
    {\BBOP}\APACciteatitle{##3},{\BAP}{##1}{\BBAY}{##2}{\BBN}{##4}{\BBCP}%
  }%
  \fullciteauthor{#1}%
  \def\bibsong##1##2##3##4{}%
}

\newcommand{\shortcitesong}[1]{%
  \def\bibsong##1##2##3##4{%
    {\BBOP}\APACciteatitle{##3},{\BAP}{##1}{\BBAY}{##2}{\BBN}{##4}{\BBCP}%
  }%
  \shortciteauthor{#1}%
  \def\bibsong##1##2##3##4{}%
}

\newcommand{\citesongNP}[1]{%
  \def\bibsong##1##2##3##4{%
    \APACciteatitle{##3},{\BAP}{##1}{\BBAY}{##2}{\BBN}{##4}%
  }%
  \citeauthor{#1}%
  \def\bibsong##1##2##3##4{}%
}

\newcommand{\fullcitesongNP}[1]{%
  \def\bibsong##1##2##3##4{%
    \APACciteatitle{##3},{\BAP}{##1}{\BBAY}{##2}{\BBN}{##4}%
  }%
  \fullciteauthor{#1}%
  \def\bibsong##1##2##3##4{}%
}

\newcommand{\shortcitesongNP}[1]{%
  \def\bibsong##1##2##3##4{%
    \APACciteatitle{##3},{\BAP}{##1}{\BBAY}{##2}{\BBN}{##4}%
  }%
  \shortciteauthor{#1}%
  \def\bibsong##1##2##3##4{}%
}

%%%%%%%%%%%%%%%%%%%%%%%%%%%%%%%%%%%%%%%%%%%%%%%%%%%%%%%%%%%%%%%%%%%%%%%
%%% COMMANDS TO CITE DSM.
%%%%%%%%%%%%%%%%%%%%%%%%%%%%%%%%%%%%%%%%%%%%%%%%%%%%%%%%%%%%%%%%%%%%%%%
% The APA manual gives some rules how the DSM should be cited, which
% is a bit different from ordinary citations. These rules are not
% implemented in standard apacite, so I provide a workaround here.

\newcommand{\DSMciteA}[1]{%
  \@for\@citec:=#1\do{%
    % Check whether citation is defined.
    \@ifundefined{flag@\@citec\@extra@b@citeb}%
      {\citeA{\@citec}}%
      {\citeauthor{\@citec}}%
  }%
}

\newcommand{\DSMfullciteA}[1]{%
  \fullciteA{#1}%
}

\newcommand{\DSMshortciteA}[1]{%
  \shortciteauthor{#1}%
}

\newcommand{\DSMcite}[1]{%
  {\BBOP}%
  \@for\@citec:=#1\do{%
    % Check whether citation is defined.
    \@ifundefined{flag@\@citec\@extra@b@citeb}%
      {\citeNP{\@citec}}%
      {\citeauthor{\@citec}}%
  }%
  {\BBCP}%
}

\newcommand{\DSMfullcite}[1]{%
  \fullcite{#1}%
}

\newcommand{\DSMshortcite}[1]{%
  {\BBOP}\shortciteauthor{#1}{\BBCP}%
}

\newcommand{\DSMciteNP}[1]{%
  \@for\@citec:=#1\do{%
    % Check whether citation is defined.
    \@ifundefined{flag@\@citec\@extra@b@citeb}%
      {\citeNP{\@citec}}%
      {\citeauthor{\@citec}}%
  }
}

\newcommand{\DSMfullciteNP}[1]{%
  \fullciteNP{#1}%
}

\newcommand{\DSMshortciteNP}[1]{%
  \shortciteauthor{#1}%
}

%%%%%%%%%%%%%%%%%%%%%%%%%%%%%%%%%%%%%%%%%%%%%%%%%%%%%%%%%%%%%%%%%%%%%%%
%%% DOCUMENT-SPECIFIC DEFINITIONS
%%%%%%%%%%%%%%%%%%%%%%%%%%%%%%%%%%%%%%%%%%%%%%%%%%%%%%%%%%%%%%%%%%%%%%%
% These are some commands that are used in this document, but are
% presumably not especially relevant for users.

\def\BibTeX{{\rm B\kern-.05em{\sc i\kern-.025em b}\kern-.08em
    T\kern-.1667em\lower.7ex\hbox{E}\kern-.125emX}}% copied from bibtex.web
\newcommand{\MakeIndex}{\textit{MakeIndex}}
\newcommand{\latextohtml}{\textup{\LaTeX2\texttt{HTML}}}% from html.sty
\newcommand{\LaTeXrtf}{\textup{\LaTeX2RTF}}%
\newcommand{\TeXht}{\textup{\TeX4ht}}%
\newcommand{\Perl}{\textsl{Perl}}%

% My comments accompanying the examples.
\newcommand{\EM}{\ensuremath{\triangleright\quad}}

% The date of a version of apacite. Makes it easy to search the document
% for these dates. The \relax is a safety measure (probably not needed),
% which ensures that the square brackets are never interpreted as an
% optional argument.
\newcommand{\APACversiondate}[1]{\relax [#1]}

\newcommand{\LC}{\mbox{${}^{\dag}$}}% \LaTeX{} Companion
\newcommand{\X}{\ensuremath{\times}}% for one table
\newcommand{\Y}{\ensuremath{c}}% for one table

\newcommand{\pkg}[1]{\textsf{#1}}% package name
\newcommand{\fname}[1]{\texttt{#1}}% file name
\newcommand{\fieldname}[1]{\texttt{#1}}% field name
\newcommand{\entryname}[1]{\texttt{#1}}% entry type name
\newcommand{\pkgoption}[1]{\texttt{#1}}% package option
\newcommand{\cmd}[1]{\texttt{\string#1}}% command, incl. `\'

% Some trickery with backslashes and @ signs.
\newcommand{\bs}{{\ttfamily \@backslashchar}}% backslash
\newcommand{\opt}[1]{$\langle${\itshape #1}$\rangle$}% generic text

% For \nocite{keys}:
\newcommand{\cmdtwo}[2]{{\mdseries\bs\texttt{#1\{}\opt{#2}%
                        \texttt{\}}}}%
% For \cite[post]{keys}:
\newcommand{\cmdthree}[3]{{\mdseries\bs\texttt{#1[}\opt{#2}%
                        \texttt{]\{}\opt{#3}%
                        \texttt{\}}}}%
% For \cite<pre>{keys}:
\newcommand{\cmdthreepre}[3]{{\mdseries\bs\texttt{#1<}\opt{#2}%
                        \texttt{>\{}\opt{#3}%
                        \texttt{\}}}}%
% For \usepackage[options]{apacite}:
\newcommand{\cmdthreealt}[3]{{\mdseries\bs\texttt{#1[}\opt{#2}%
                        \texttt{]\{#3\}}}}%
% For all \cite<pre>[post]{keys} variants:
\newcommand{\cmdfour}[4]{{\mdseries\bs\texttt{#1<}\opt{#2}%
                        \texttt{>[}\opt{#3}%
                        \texttt{]\{}\opt{#4}%
                        \texttt{\}}}}%

% It is not clear where and how the \flqq command is defined, so provide
% an escape if it is not available.
\AtBeginDocument{%
  \@ifundefined{flqq}{%
    \newcommand{\flqq}{\raisebox{2pt}{\ensuremath{{\scriptscriptstyle\ll}}}}%
    \newcommand{\frqq}{\raisebox{2pt}{\ensuremath{{\scriptscriptstyle\gg}}}}%
  }{}}%

%%%%%%%%%%%%%%%%%%%%%%%%%%%%%%%%%%%%%%%%%%%%%%%%%%%%%%%%%%%%%%%%%%%%%%%
\makeatother
%%%%%%%%%%%%%%%%%%%%%%%%%%%%%%%%%%%%%%%%%%%%%%%%%%%%%%%%%%%%%%%%%%%%%%%
%%% START OF ``NORMAL'' PART OF DOCUMENT
%%%%%%%%%%%%%%%%%%%%%%%%%%%%%%%%%%%%%%%%%%%%%%%%%%%%%%%%%%%%%%%%%%%%%%%
\title{The \pkg{apacite} package\vspace{0.3\baselineskip}\\
       \large Citation and reference list with \LaTeX{} and \BibTeX{}
       according to the rules of the American Psychological Association}
\author{Erik Meijer\\
        \normalsize\itshape
        Department of Econometrics,
        University of Groningen\\
        \normalsize\itshape
        PO Box 800,
        9700 AV Groningen,
        The Netherlands\\
        \normalsize
        E-Mail: \textsf{e.meijer@eco.rug.nl}
}
\date{June 8, 2005}
%%%%%%%%%%%%%%%%%%%%%%%%%%%%%%%%%%%%%%%%%%%%%%%%%%%%%%%%%%%%%%%%%%%%%%%
\begin{document}
%%%%%%%%%%%%%%%%%%%%%%%%%%%%%%%%%%%%%%%%%%%%%%%%%%%%%%%%%%%%%%%%%%%%%%%
\maketitle
%%%%%%%%%%%%%%%%%%%%%%%%%%%%%%%%%%%%%%%%%%%%%%%%%%%%%%%%%%%%%%%%%%%%%%%
\begin{abstract}
  \noindent
  This document describes and tests the \pkg{apacite} package
  \APACversiondate{2005/06/08}. This is a package that can be used with
  \LaTeX{} and \BibTeX{} to generate citations and a reference list,
  formatted according to the rules of the American Psychological Association.
  Furthermore, \pkg{apacite} contains an option to (almost) automatically
  generate an author index as well. The package can be customized in many
  ways.
\end{abstract}
%%%%%%%%%%%%%%%%%%%%%%%%%%%%%%%%%%%%%%%%%%%%%%%%%%%%%%%%%%%%%%%%%%%%%%%
\tableofcontents
%%%%%%%%%%%%%%%%%%%%%%%%%%%%%%%%%%%%%%%%%%%%%%%%%%%%%%%%%%%%%%%%%%%%%%%
\section{Introduction}
%%%%%%%%%%%%%%%%%%%%%%%%%%%%%%%%%%%%%%%%%%%%%%%%%%%%%%%%%%%%%%%%%%%%%%%
The American Psychological Association (APA) is very strict about the style in
which manuscripts submitted to its journals are written and formatted. The
requirements of the APA are described in the \emph{Publication Manual of the
American Psychological Association}, the latest version of which is the 5th
edition \cite{APAManual}. In the sequel, this is simply called the \emph{APA
manual}.

The APA manual discusses how candidate authors should write their manuscripts:
writing style, parts of a manuscript and their order, presentation of the
results in the form of tables and figures, and so forth. Candidate authors
should study this and adhere to this.

The APA manual also gives specific rules about the formatting of a manuscript.
This includes double spacing, a running head, the typographic style of section
headings, the placement of tables and figures on separate pages at the end of
the document, and so forth. \LaTeX{} users will recognize these as ``style''
elements that should be defined in a package (\fname{.sty} file) or class
(\fname{.cls} file). Their specific documents (\fname{.tex} file) should be
largely style-independent. This idea of separating content and logical
structure from specific formatting is one of the basic elements of \LaTeX{}
\cite[p.~7]{LaTeXbook}.

An implementation of the formatting rules of the APA manual for use with
\LaTeX{} is the \pkg{apa} class of Athanassios Protopapas. This handles all
kinds of issues about general document formatting, title page, section
headings, figures and tables, and so forth. Therefore, if you intend to submit
a manuscript to an APA journal, I strongly recommend using the \pkg{apa}
class.

An important part of the APA style is the way citations and the reference list
should be formatted. This takes 75 pages in the APA manual (pp.~207--281,
excluding the references to legal materials). This part is not handled by the
\pkg{apa} class, but by the \pkg{apacite} package. The \pkg{apa} class
requires the \pkg{apacite} package, but \pkg{apacite} can be used without
\pkg{apa}. The current document, for example, does not use the \pkg{apa}
class, because I find it desirable that possible users can study the
\pkg{apacite} package and its documentation without having to install several
other classes and packages first. Therefore, the current document uses
standard \LaTeX{} as much as possible.

%%%%%%%%%%%%%%%%%%%%%%%%%%%%%%%%%%%%%%%%%%%%%%%%%%%%%%%%%%%%%%%%%%%%%%%
\subsection*{Some history}
%%%%%%%%%%%%%%%%%%%%%%%%%%%%%%%%%%%%%%%%%%%%%%%%%%%%%%%%%%%%%%%%%%%%%%%
Before the advent of the first version of \pkg{apacite} in 1994, several
authors have written \TeX{} and \LaTeX{} packages (\fname{.sty}) and \BibTeX{}
style files (\fname{.bst}) with suggestive names as \pkg{apalike},
\pkg{newapa}, and \pkg{theapa} to try to capture some of the requirements of
the APA manual. A severe problem with the APA citations is that, if the number
of authors is between three and five, inclusive, the first citation must
include all authors, whereas subsequent citations should refer to the first
author ``et al.''. This problem had been solved by the \pkg{newapa} and
\pkg{theapa} packages.

A lot of APA peculiarities, however, were not covered by these citation
styles. Examples are:
\begin{itemize}
    \item In the reference list, authors should be formatted with
          their initials after their surnames. Editors, however,
          should be formatted with their initials \emph{before} their
          surnames if they are not in the author position.
    \item If the work referred to is a non-English-language source,
          the English translation of the title should be given in
          brackets after the title.
    \item Edition and volume of a book and the pages of an article
          in that book should be gathered between a single pair
          of parentheses.
    \item The publisher of a book and its address should be given as
          ``address: publisher'', and not as ``publisher, address''.
    \item When the work used is a reprint or translation of an
          earlier work, this should be mentioned in the reference
          list. In text, the year must be given in the form
          ``1923/1961''.
    \item There are several ``tricky'' things with brackets and
          parentheses, for example, with the type of an article
          (e.g., ``Letter to the editor'',
          ``Review of the book \emph{The social life of information}'',
          ``Abstract'', ``Special issue''). Furthermore, magazine and
          newspaper articles are treated differently than journal
          articles; papers presented at meetings must be treated
          differently; translators of articles and books and chairs
          of symposia must be introduced; an article can have editors
          instead of authors (e.g., with a special issue); a PhD thesis
          can be abstracted in \emph{Dissertation Abstracts International};
          a (technical) report of some institution can be an (edited)
          collection of articles.
\end{itemize}
The \pkg{apacite} package is devised to get (much) closer to the APA rules
than the abovementioned other packages and styles.

The original version of the \fname{apacite.sty} \LaTeX{} package consisted for
a large part of the citation part of \fname{theapa.sty}. The current version,
however, has been changed and extended in many ways and can therefore not be
considered a straightforward adaptation of \fname{theapa.sty} anymore,
although it provides largely the same commands, such as \cmd{\citeA}, and
peculiarities in the command definitions, such as the use of \verb+<...>+ for
a prefix note and \verb+[...]+ for a postfix note.

Although the \fname{apacite.bst} \BibTeX{} style started out as a slightly
adapted version of \fname{theapa.bst}, the number of changes became so large
and things became so unmanageable that I decided to write a completely new
bibstyle, although using some small parts of the original. The
\fname{apacite.bst} file can therefore not anymore be seen as an adaptation of
\fname{theapa.bst}. Due to the complex rules of the APA manual, it has become
a large file. In the original process of writing (back in 1994), the error
\begin{verbatim}
You've exceeded BibTeX's wizard-defined function space
\end{verbatim}
was very hard to overcome. It appears that later versions of \BibTeX{} allowed
much more space, because I have not received this message anymore in recent
years, although the bibstyle has been expanded considerably. However, if you
run an old \BibTeX{}, you may encounter this (and similar) error messages.

In the ``dark years'' between 1994 and 2003, in which there was no official
new version of \pkg{apacite}, I have been fixing small bugs, discovered either
by me or by users who sent in their bug reports, and adding some small options
that were easy to incorporate and that I found useful or that were requested
by a user. Furthermore, in 1998, I spent considerable effort in adding an
(almost) automatic author indexing option, first used for my PhD thesis
\cite{meijer1998}.

In the first few months of 2003, I fixed a number of problems with the
compatibility between \pkg{apacite} and some other important packages.
Although I had been planning (or wishing, rather) to release a proper official
update of \pkg{apacite} for some time, I had not done so, mainly because I had
not yet taken the time to update the documentation. The compatibility issues
were so severe, however, that I decided in March 2003 to release an
intermediate update, which thus became the first official release since the
original version. This version contained a lot of files with compatibility
patches and tests. The main shortcomings of that version were the lack of
up-to-date documentation and the lack of agreement with the latest version of
the APA manual (5th ed., 2001).

In September 2003, I finally released a complete, fully updated and
significantly extended version of \pkg{apacite}, including extensive
documentation. However, that version still was not able to format some
citations correctly. The \APACversiondate{2004/07/01} version was a small
improvement of the \APACversiondate{2003/09/05} version, but fixed the
important inadequacies of previous versions.

The current version is also a relatively minor update, but it may be important
for some users. The main difference with previous versions is that it offers
support of non-English languages, although this is still limited. Moreover,
this is clearly a non-APA direction, because all APA publications are in
English. However, many journals, universities, etc.\ in non-English speaking
countries base their rules on the APA rules, and thus it may be useful to have
the possibility to use \pkg{apacite} in combination with documents in
non-English languages as well. See section~\ref{sec:compat-babel} for a
description of the implementation of this feature.

Apart from this language support and the facilities for customization, the
current version of \pkg{apacite} is aimed at conforming with the 5th edition
of the APA manual. It remains, however, imperfect, although the instances of
wrongly formatted cases will be rare. Many problems can be solved by
``tricking'' the style (e.g., by defining the \fieldname{number} field as
``\verb+2, Serial No. 186+'' if the issue number and serial number of a
journal are both important), but this makes the \fname{.bib} file less
compatible with other styles. In the test examples in section~\ref{sec:xmpl}
below, many tricks are used and discussed. Remaining problems, omissions, and
things-to-be-done can be found in section~\ref{sec:todo}.

%%%%%%%%%%%%%%%%%%%%%%%%%%%%%%%%%%%%%%%%%%%%%%%%%%%%%%%%%%%%%%%%%%%%%%%
\subsection*{Philosophy of \pkg{apacite}}
%%%%%%%%%%%%%%%%%%%%%%%%%%%%%%%%%%%%%%%%%%%%%%%%%%%%%%%%%%%%%%%%%%%%%%%
As can be deduced from the discussion thus far, the first priority of
\pkg{apacite} is to implement the rules of the APA manual with regard to
citation and reference list as closely as possible. However, just like its
predecessors, \pkg{apacite} offers some possibilities for customization as
well. Many details of \pkg{apacite}, particularly punctuation and some fixed
texts (e.g., ``Tech.\ Rep.'') can be changed easily by the user by redefining
some commands in \LaTeX{}. Furthermore, \pkg{apacite} also offers several
proper options to change some of its settings.

Whether certain options or customizable aspects are implemented depends on two
criteria: (1) Is it possible, easy (enough), and convenient to implement them
without compromising the ability to adhere to the APA rules, and (2) Do I (EM)
consider them important or useful enough to spend time to implement them.
Actually, the decision process is the reverse of this: First, I decide whether
I find it a relevant or useful option. If not, I will not implement it. If so,
I will think about if and how I can implement it. If I have an idea for a
solution that is practically feasible, I will pursue it. If I don't see a
solution, if I think it will take me too much time, or if I think a solution
will be inconvenient to other users, then I will not pursue it.

%%%%%%%%%%%%%%%%%%%%%%%%%%%%%%%%%%%%%%%%%%%%%%%%%%%%%%%%%%%%%%%%%%%%%%%
\subsection*{Changes since the \APACversiondate{2004/07/01} version}
%%%%%%%%%%%%%%%%%%%%%%%%%%%%%%%%%%%%%%%%%%%%%%%%%%%%%%%%%%%%%%%%%%%%%%%
The changes since the \APACversiondate{2004/07/01} version are:
\begin{itemize}
\item Addition of the \cmd{\BBYY} command (see page~\pageref{cmd:BBYY}).
\item Limited language support (see section~\ref{sec:compat-babel}).
\item Resolved some compatibility problems between \pkg{apacite} and
      \pkg{natbib} (see section~\ref{sec:compat-natbib}).
\item Fixed a bug that caused a compatibility problem when \pkg{apacite}
      was used without one of the author indexing options, but other
      indexes were requested in the document.
\item Updated the manual (the current document); rewrote some sections.
\item Improved some of the ``dirty tricks'' in \fname{apacite.tex}
      (see section~\ref{sec:xmpl}).
\item Some minor changes that most users will not notice.
\end{itemize}

%%%%%%%%%%%%%%%%%%%%%%%%%%%%%%%%%%%%%%%%%%%%%%%%%%%%%%%%%%%%%%%%%%%%%%%
\section{Basic usage and options}
\label{sec:usage}
%%%%%%%%%%%%%%%%%%%%%%%%%%%%%%%%%%%%%%%%%%%%%%%%%%%%%%%%%%%%%%%%%%%%%%%
The current \pkg{apacite} distribution consists of the following files:
\begin{description}
\item[\fname{apacite.sty}] The \LaTeX{} citation package. This must
  be placed in a directory where \TeX{} can find it;
\item[\fname{apacite.bst}] The \BibTeX{} reference list style. This
  must be placed in a directory where \BibTeX{} can find it;
\item[\fname{apacitex.bst}] The \BibTeX{} reference list style with
  added author index support. This must also be placed in a directory
  where \BibTeX{} can find it;
\item[\fname{apacite.tex}] Extensive user's manual and test of the
  \pkg{apacite} package;
\item[\fname{apacite.pdf}] The processed version of \fname{apacite.tex}
  so you can read the manual without having to install the package first;
\item[\fname{apa5ex.bib}] The file with bibliographic information
  about the references in \fname{apacite.tex};
\item[\fname{manifest.txt}] The list of files belonging to the \pkg{apacite}
  package, with a short description;
\item[\fname{README}] A brief description of the package;
\item[\fname{lang/}] A directory containing language-specific files
  (see section~\ref{sec:compat-babel}).
\end{description}
The \fname{apacite.sty} \LaTeX{} package is loaded by putting
\mbox{}\vspace{0.7\baselineskip}\\
\cmdthreealt{usepackage}{options}{apacite}
\mbox{}\vspace{0.7\baselineskip}\\
somewhere in your document between \verb+\documentclass+ and
\verb+\begin{document}+, or putting
\mbox{}\vspace{0.7\baselineskip}\\
\cmdthreealt{RequirePackage}{options}{apacite}
\mbox{}\vspace{0.7\baselineskip}\\
in your own personal \LaTeX{} package (say, \fname{mysettings.sty})
that is loaded by your document.

The following options are recognized by \pkg{apacite}:
\begin{description}
\item[\pkgoption{BCAY}] \mbox{}\\
  This is a technical option for backwards compatibility with old versions
  (pre-\APACversiondate{2003/09/05}) of \pkg{apacite}. In those versions of
  \pkg{apacite}, the \cmd{\BCAY} construction was used to pass relevant
  citation information from the \fname{.bbl} file (\BibTeX{}'s output) to
  \LaTeX{}. This was taken over from its immediate predecessor, Young U. Ryu's
  \pkg{theapa}. However, \pkg{natbib} does not recognize the \cmd{\BCAY}
  construction, but it does recognize the analogous \cmd{\citeauthoryear}
  construction, which was also used by an earlier predecessor of
  \pkg{apacite}, \pkg{newapa}. Therefore, \pkg{apacite} has reverted to
  \cmd{\citeauthoryear} as well. This makes different versions of
  \pkg{apacite} incompatible with each other, because it is not possible to
  support both constructions at the same time. This option is used to fix
  that: In the (unlikely) event that you \emph{must} use a \fname{.bbl} file
  that is generated by an old version of \pkg{apacite}, you can turn this
  option on.

\item[\pkgoption{accentfix}] \mbox{}\\
  A cedilla accent (`\c{c}': \verb+\c{c}+) in an author name used to lead to
  errors, only due to the check whether subsequent authors in the same
  citation are equivalent. The same problem is anticipated with the
  \verb+\b{...}+ (`\b{o}') accent, the \verb+\d{...}+ (`\d{o}') accent, and
  the \verb+\copyright+ (\copyright) symbol (Will the latter ever turn up in
  an author name?), due to the same use of \cmd{\crcr}, cf.\
  \citeA[p.~356]{TeXbook}.

  This fix therefore redefines these to sensible alternatives, only for
  comparison of authors, not for any real formatting. Therefore, the user
  should not notice this, apart from the lack of error messages, of course.
  Therefore, it is also a default option. However, because I am not entirely
  sure that there are no possible adverse effects, I give the user the option
  to turn it off (\pkgoption{noaccentfix}). The user is then responsible for
  fixing any error messages generated by such characters.

\item[\pkgoption{noaccentfix}] \mbox{}\\
  Turns \pkgoption{accentfix} off.

\item[\pkgoption{index}]
\item[\pkgoption{stdindex}]
\item[\pkgoption{tocindex}]
\item[\pkgoption{emindex}] \mbox{}\\
  These four options turn author indexing on, but differ in the way the index
  is formatted. See section~\ref{sec:autindex} for a discussion of the author
  indexing facility. These should be used with the \fname{apacitex.bst}
  \BibTeX{} style, see below, although they also work with
  \fname{apacite.bst}, but that does not give author index entries, so these
  options then typically do not have any effect (and an undesirable effect if
  they do).

\item[\pkgoption{noindex}] \mbox{}\\
  Turns author indexing off (the default). Typically used with
  \fname{apacite.bst}, but can also be used with \fname{apacitex.bst}. In the
  latter case, the author indexing commands are simply ignored. Therefore,
  \fname{apacite.bst} is actually superfluous, but because author indexing
  will be used rarely and it is more likely to lead to errors or
  incompatibilities, a ``clean'' (no author indexing) version,
  \fname{apacite.bst}, is provided as well.

\item[\pkgoption{numberedbib}] \mbox{}\\
  This option implies that the bibliography (reference list) is a numbered
  section or chapter, e.g., ``6.~References'', instead of just ``References''.

\item[\pkgoption{unnumberedbib}] \mbox{}\\
  The reverse of \pkgoption{numberedbib}: The bibliography is an unnumbered
  section or chapter. This is the default. However, it is possible that when
  using the \pkg{apa} document class, then \pkgoption{numberedbib} works
  better, because that class turns section numbering off anyway and it may be
  that \pkg{apa}'s page headings work well if the reference list is a
  \cmd{\section} and not if it is a \cmd{\section*}. I have not experimented
  with this (yet), however.

\item[\pkgoption{sectionbib}] \mbox{}\\
  With this option, the bibliography is a section and not a chapter. Mainly
  useful in combination with the \pkg{chapterbib} package. Therefore, it will
  be discussed in more detail in section~\ref{sec:compat-chapterbib}.

\item[\pkgoption{nosectionbib}] \mbox{}\\
  With this option, the bibliography is a chapter, if the \cmd{\chapter}
  command is defined. Otherwise, it is always a section. Again, see
  section~\ref{sec:compat-chapterbib}.

\item[\pkgoption{tocbib}] \mbox{}\\
  This puts the bibliography in the table of contents, even if it is
  unnumbered, provided of course that a table of contents is requested in the
  document (by \cmd{\tableofcontents}). This is the default.

\item[\pkgoption{notocbib}] \mbox{}\\
  This does not put the bibliography in the table of contents if it is an
  unnumbered section or chapter. If it's numbered, it is always in the table
  of contents.

\item[\pkgoption{bibnewpage}] \mbox{}\\
  The bibliography is started on a new page. This is required by some journal
  styles, including the APA manual. The \pkg{apa} class already contained this
  in its \pkgoption{man} option, but now it has been made available directly
  in \pkg{apacite}.

\item[\pkgoption{nobibnewpage}] \mbox{}\\
  The bibliography is not explicitly started on a new page, although if the
  bibliography is a chapter, it will be started on a new page anyway, because
  chapters are started on a new page. This is the default in \pkg{apacite} and
  thus is the only time a non-APA setting is used as default instead of an
  available APA setting. Therefore, to satisfy the APA rules, you have to
  request the \pkgoption{bibnewpage} option explicitly.

\item[\pkgoption{hyper}] \mbox{}\\
  This switches on some code provided by Ross Moore that makes \pkg{apacite}
  and \pkg{hyperref} compatible to some extent. However, this also violates
  the APA manual rules somewhat, so it is not a default option. See
  section~\ref{sec:compat-hyper} for a more detailed description.

\item[\pkgoption{nohyper}] \mbox{}\\
  This switches off the code of the \pkgoption{hyper} option mentioned above.
  This is the default.
\end{description}
%
To load the \fname{apacite.bst} or \fname{apacitex.bst} bibliography style in
\BibTeX{}, put
\mbox{}\vspace{0.7\baselineskip}\\
\verb+\bibliographystyle{apacite}+
\mbox{}\vspace{0.7\baselineskip}\\
or
\mbox{}\vspace{0.7\baselineskip}\\
\verb+\bibliographystyle{apacitex}+
\mbox{}\vspace{0.7\baselineskip}\\
respectively, in your document before the \cmd{\bibliography} command. The
position of the bibliography (reference list) is determined by the line
\mbox{}\vspace{0.7\baselineskip}\\
\cmdtwo{bibliography}{bibfiles}
\mbox{}\vspace{0.7\baselineskip}\\
where \opt{bibfiles} is a list of filenames with \fname{.bib} extension, which
contain the bibliographic information that is used by \BibTeX{} to construct
the reference list. Usually, the \cmd{\bibliographystyle} and
\cmd{\bibliography} are kept together (immediately follow each other) in the
document, although when you are using the \pkg{apa} document class,
\pkg{apacite} is used by definition and you are not supposed to use the
\cmd{\bibliographystyle} command yourself. See the documentation of the
\pkg{apa} documentclass for details about this.

If you use one of the author indexing options, the author index is put in
the \LaTeX{} output by the line
\begin{verbatim}
\printindex[autx]
\end{verbatim}
If you put this line in your document, but use the \pkgoption{noindex} option
(either explicitly or implicitly by requesting no other index option), it will
be ignored. For more on author indexing, see section~\ref{sec:autindex}.

To get all parts in the final output, the following sequence of runs should
typically be taken (when starting from scratch): (1) \LaTeX{}, (2) \BibTeX{},
(3) \LaTeX{}, (4) \LaTeX{}, and, when author indexing is on, (5) \MakeIndex{},
(6) \LaTeX{}, and (7) \LaTeX{}. The last one is to get the index in the table
of contents. If the table of contents is on a regular page, i.e., an
arabic-numbered page instead of a roman-numbered page in the front matter, it
may even be necessary to run \MakeIndex{} another time, followed by \LaTeX{}
once or twice. Occasionally, somewhere in the process, \LaTeX{} may complain
about labels that may have changed, which requires even more additional
\LaTeX{} runs at that stage. So the number of runs that are necessary to get
everything right may become large.

%%%%%%%%%%%%%%%%%%%%%%%%%%%%%%%%%%%%%%%%%%%%%%%%%%%%%%%%%%%%%%%%%%%%%%%
\section{The citation commands}
\label{sec:cite}
%%%%%%%%%%%%%%%%%%%%%%%%%%%%%%%%%%%%%%%%%%%%%%%%%%%%%%%%%%%%%%%%%%%%%%%
In this section, the commands are described that can be used to cite a work.
Their default behavior will be explained. In section~\ref{sec:custom},
customization of this behavior will be discussed. Extensive examples are given
in section~\ref{sec:xmpl}.

If the \pkg{apacite} package is loaded, the following citation commands can be
used.
\begin{description}
    \item[\cmdfour{cite}{prefix}{postfix}{keys}]
    \item[\cmdfour{fullcite}{prefix}{postfix}{keys}]
    \item[\cmdfour{shortcite}{prefix}{postfix}{keys}] \mbox{}\\
        These three commands produce parenthetical citations of the
        following form:
        ``(\opt{prefix} \opt{Authors1}, \opt{Year1};
        \opt{Authors2}, \opt{Year2}; \ldots; \opt{AuthorsN}, \opt{YearN},
        \opt{postfix})''.
        The command \cmd{\fullcite} uses the ``full'' author list, and
        the \cmd{\shortcite} command uses the abbreviated author list.
        The \cmd{\cite} command uses the ``full'' list the first time
        a work is referenced and the ``short'' list the second and
        subsequent times. In accordance with the APA manual, the
        ``full'' list consists of all authors if their number is five
        or less, and the first author suffixed with ``et al.{}''
        if their number is six or more. Similarly, the ``short'' list
        consists of all authors if their number is two or less and the
        first author with ``et al.{}'' if their  number is three or more.
        There are some nuances for complicated situations. These will
        be discussed in section~\ref{sec:xmpl}.

        If the (full) author lists of subsequent citations within the
        same citation command are the same, they are not repeated; only
        the years of the later citations are given.
        This holds for all analogous cases as well.

        The prefix and postfix are optional. Thus, e.g.,
        \cmdtwo{cite}{keys}, \cmdthree{cite}{postfix}{keys},
        and \cmdthreepre{cite}{prefix}{keys} are also allowed.
        This holds for all analogous cases as well.

    \item[\cmdfour{citeA}{prefix}{postfix}{keys}]
    \item[\cmdfour{fullciteA}{prefix}{postfix}{keys}]
    \item[\cmdfour{shortciteA}{prefix}{postfix}{keys}] \mbox{}\\
        These commands are similar to the commands without the
        ``\verb+A+''  suffix. They produce in-text citations of the form
        ``\opt{prefix} \opt{Authors1} (\opt{Year1}),
        \opt{Authors2} (\opt{Year2}), \ldots,
        \opt{AuthorsN} (\opt{YearN}, \opt{postfix})''.
        Typically, the \opt{prefix} will be empty, because it is
        part of the main text, and there typically will only be
        one citation in \opt{keys} (unless the authors of
        multiple citations are the same), because the authors'
        names are also part of the main text. But it was easy
        to provide the additional options, and this enhances
        the similarity with the other citation commands, which
        I found desirable.

    \item[\cmdfour{citeauthor}{prefix}{postfix}{keys}]
    \item[\cmdfour{fullciteauthor}{prefix}{postfix}{keys}]
    \item[\cmdfour{shortciteauthor}{prefix}{postfix}{keys}] \mbox{}\\
        These commands are similar to their \cmd{\cite}-type
        counterparts, except that they produce citations of the form
         ``\opt{prefix} \opt{Authors1}, \opt{Authors2}, \ldots,
         \opt{AuthorsN}, \opt{postfix}''
        (the years and parentheses are left out).
        This is typically used in a paragraph when a proper
        citation (incl.\ year) to the same work has already been
        given in the paragraph.%
            \footnote{Note that a paragraph is not
            a section. Nor does it need to be declared as a
            \cmd{\paragraph}. A reminder for Dutch readers:
            ``section'' = ``paragraaf'' and ``paragraph'' = ``alinea''.}
        In such a case, according to the APA manual, the year must not be
        repeated for repeated references to the same work in the same
        paragraph.

        A second important application of these commands is
        when some complex citations must be constructed, like
        ``\citeauthor{4.05-1}'s \citeyear{4.05-1} theory'',
        which makes combined use of \cmd{\citeauthor} and
        \cmd{\citeyear}, see below.

        Normally, I would expect only one citation per
        \cmd{\citeauthor}-type command, and no \opt{prefix}
        or \opt{postfix}.

    \item[\cmdfour{citeyear}{prefix}{postfix}{keys}] \mbox{}\\
        Produces citations of the form
        ``(\opt{prefix} \opt{Year1}, \opt{Year2}, \ldots, \opt{YearN},
        \opt{postfix})''.
        See the discussion of \cmd{\citeauthor} above. Typically,
        the \opt{prefix} would be empty and multiple citations
        would only be used if they have the same authors.

    \item[\cmdfour{citeyearNP}{prefix}{postfix}{keys}] \mbox{}\\
        A ``no parentheses'' version of \cmd{\citeyear}.
        Produces citations of the form
        ``\opt{prefix} \opt{Year1}, \opt{Year2}, \ldots, \opt{YearN},
        \opt{postfix}''. Can be used for complex citations within
        parenthetical material, like
        ``the work of \citeauthor{3.99-15} (\citeyearNP{3.99-15};
        but see also \citeNP{3.99-16,3.99-17})'',
        which also uses the \cmd{\citeNP} command, see below.

    \item[\cmdfour{citeNP}{prefix}{postfix}{keys}]
    \item[\cmdfour{fullciteNP}{prefix}{postfix}{keys}]
    \item[\cmdfour{shortciteNP}{prefix}{postfix}{keys}] \mbox{}\\
        ``No parentheses'' versions of \cmd{\cite}, \cmd{\fullcite},
        and \cmd{\shortcite}. They produce citations of the form
        ``\opt{prefix} \opt{Authors1}, \opt{Year1};
        \opt{Authors2}, \opt{Year2}; \ldots; \opt{AuthorsN}, \opt{YearN},
        \opt{postfix}''. Can be used for complex citations within
        parenthetical material, see the discussion of
        \cmd{\citeyearNP} above.

    \item[\cmdtwo{nocite}{keys}] \mbox{}\\
        The entries are included in the bibliography without
        being cited in text. This is standard \LaTeX.
        Note that the APA manual normally does not allow this.
        The only exception concerns works that are studied in a
        meta-analysis, but for these the \cmd{\nocitemeta}
        command should be used.

    \item[\cmdtwo{nocitemeta}{keys}] \mbox{}\\
        This is similar to \cmd{\nocite}. It is used to get the
        list of works included in a meta-analysis in the
        bibliography. This command has the additional effect that
        the corresponding items in the bibliography are preceded
        by an asterisk ($\star$) and a remark explaining this
        is inserted just before the bibliography, as required
        by the APA manual.
\end{description}
In order to format the examples from the APA manual correctly in
section~\ref{sec:xmpl}, I have also defined the following additional citation
commands: \cmd{\DSMcite}, \cmd{\biblecite}, and \cmd{\citesong}, and obvious
variations with \verb+full+, \verb+short+, \verb+A+, and \verb+NP+. However, I
have decided not to include these in \fname{apacite.sty}. The most important
reason for this is that I find it unattractive to introduce different types of
citation commands for different types of citations. This also makes it more
difficult to use the same document with another citation package. I hope to be
able to provide, in a future version of \pkg{apacite}, more elegant solutions
to the problems for which these commands were defined.

Nevertheless, I used these commands for the examples in section~\ref{sec:xmpl}
and therefore included their definitions at the top of the file
\fname{apacite.tex}. Thus, if you want to use them, you can copy their
definitions to the top of your own document or personal style file (say,
\fname{mysettings.sty}, do not change \fname{apacite.sty}), you can study
their behavior, and perhaps improve them.

%%%%%%%%%%%%%%%%%%%%%%%%%%%%%%%%%%%%%%%%%%%%%%%%%%%%%%%%%%%%%%%%%%%%%%%
\section{Contents of the \fname{.bib} file}
\label{sec:bib}
%%%%%%%%%%%%%%%%%%%%%%%%%%%%%%%%%%%%%%%%%%%%%%%%%%%%%%%%%%%%%%%%%%%%%%%
The information that is used by \LaTeX/\BibTeX{} to generate
the citations and reference list must be stored by the user in one
or more files with the \fname{.bib} extension.
A detailed overview of the contents of the \fname{.bib} file is
given in \citeA[section~13.5]{LaTeXcomp}. Roughly speaking, the
\fname{.bib} file consists of a collection of \emph{entries}
of the form
\mbox{}\vspace{0.7\baselineskip}\\
\verb+@+\opt{entryname}\verb+{+\opt{contents}\verb+}+
\mbox{}\vspace{0.7\baselineskip}\\
Most entries describe a work that may be referenced, e.g., a book
or article. There are, however, two exceptions: a \entryname{preamble}
entry (with \opt{entryname} = \verb+preamble+) and a \entryname{string}
entry (with \opt{entryname} = \verb+string+). There is typically
at most one \entryname{preamble} entry. Its \opt{contents} consist of a
string---according to \BibTeX, i.e., between double quotes (\verb+"+)
or an additional pair of braces (\verb+{+ and \verb+}+). This
string, which typically consists of \LaTeX{} commands like
\verb+\newcommand{\SortNoop}[1]{}+, is literally included
in the document before the bibliography. It can therefore be used
to define commands that are used in the (other) entries of the
\fname{.bib} file and that are not standard \LaTeX{}.

A \entryname{string} entry is similar, but the contents are now not
included in the document. Instead, the string is assigned to a kind of
``variable'', the name of which may contain characters
not usually associated with variable names, like colons and
hyphens, see \citeA[pp.~402--403]{LaTeXcomp}. An example of a
\entryname{string} entry is
\begin{verbatim}
@string{ JPSP = {Journal of Personality and Social Psychology} }
\end{verbatim}
Then, if in a later entry, the journal is
\emph{Journal of Personality and Social Psychology}, you can format this
as
\begin{verbatim}
  journal = JPSP,
\end{verbatim}
instead of the usual
\begin{verbatim}
  journal = {Journal of Personality and Social Psychology},
\end{verbatim}
This is not only convenient because it saves typing of common
long journal names, but it can also be used to define
style-specific variations. For example, some styles abbreviate
journal names, such as
\emph{J. Pers. Soc. Psych.} instead of
\emph{Journal of Personality and Social Psychology}.
You could then put the definitions of all full journal names
in one \fname{.bib} file (\fname{fulljou.bib}, say) and the definitions
of the abbreviated journal names in another (\fname{abbrjou.bib}, say).
Suppose that the information about the referenced works is stored in a
third file, say, \fname{myrefs.bib}, and that the ``NotAPA''
style rules, implemented in the \fname{notapa.bst} \BibTeX{} style file,
require the abbreviated journal names. Then you can use
\begin{verbatim}
\bibliographystyle{notapa}
\bibliography{abbrjou,myrefs}
\end{verbatim}
to obtain a reference list complying with the ``NotAPA'' rules. If
you change your mind and decide to switch to using the rules of the
APA manual, you only need to change the two lines above into
\begin{verbatim}
\bibliographystyle{apacite}
\bibliography{fulljou,myrefs}
\end{verbatim}
(and additionally load the \fname{apacite.sty} \LaTeX{} package
through \cmd{\usepackage}).

In the accompanying \fname{.bib} file, I have used the
\entryname{string} entry
to put some comments in the file:
\begin{verbatim}
@string{ comment = {
\end{verbatim}
$\mbox{}\qquad$\opt{comment text}
\begin{verbatim}
}}
\end{verbatim}
This is useful because \BibTeX{} does not have a comment character, e.g., the
percent sign does not work. If you put text between entries, this is ignored,
so you may not need a comment character, but I wanted to put my e-mail address
in the comments at the top of the file, and the \verb+@+ sign of an e-mail
address is interpreted as the start of a new entry by \BibTeX{}, unless it is
put into a string.

The remaining entry types are types that correspond with the
type of a referenced work, e.g., \entryname{book} or
\entryname{article}. These entry types (reference types) are
discussed in section~\ref{subsec:types} below.
Such entries have the following structure:
\mbox{}\vspace{0.7\baselineskip}\\
\verb+@+\opt{entryname}\verb+{+\opt{key}\verb+,+\\
$\mbox{}\qquad$\opt{fieldname1}\verb+ = +\opt{value1}\verb+,+\\
$\mbox{}\qquad$\opt{fieldname2}\verb+ = +\opt{value2}\verb+,+\\
$\mbox{}\qquad\qquad\vdots$\\
$\mbox{}\qquad$\opt{fieldnameN}\verb+ = +\opt{valueN}\\
\verb+}+
\mbox{}\vspace{0.7\baselineskip}\\
where \opt{key} corresponds to the key used in the citation
commands and matches the entry with the citation (and should
therefore be unique). The fieldnames are \verb+author+, \verb+year+,
etc., which are described in section~\ref{subsec:fields} below.
The values are strings, either defined previously by a
\entryname{string} entry or explicitly indicated as such here
by putting the relevant information between double quotes
(\verb+"+) or a pair of braces (\verb+{+ and \verb+}+).

To be able to obey the rules of the APA manual, \pkg{apacite}
provides several fields and reference types that are not
described in the standard \BibTeX{} documentation
\cite<e.g.,>[chap.~13]{LaTeXcomp}. Furthermore,
the meaning and usage of many fields and reference types that
\emph{are} described there have been altered somewhat. Therefore,
a complete description is given here.
The symbol \LC{} will be used to indicate that the field or
reference type is also described in \citeA[Appendix~B]{LaTeXbook}
or \citeA[chapter~13]{LaTeXcomp}, although, as mentioned above, the
specific meaning of the item may have been changed.

%%%%%%%%%%%%%%%%%%%%%%%%%%%%%%%%%%%%%%%%%%%%%%%%%%%%%%%%%%%%%%%%%%%%%%%
\subsection{Types of references}
\label{subsec:types}
%%%%%%%%%%%%%%%%%%%%%%%%%%%%%%%%%%%%%%%%%%%%%%%%%%%%%%%%%%%%%%%%%%%%%%%
In this section, the list of reference types that are recognized by
\pkg{apacite} is given. An overview of which fields can be used for which
reference type is given in Table~\ref{tab:fieldref}. If fields are not
relevant, but you use them anyway in your \fname{.bib} file, they are ignored.
The fields and their use are described in more detail in
section~\ref{subsec:fields}, but here some specific issues will already be
mentioned.

%%%%%%%%%%%%%%%%%%%%%%%%%%%%%%%%%%%%%%%%%%%%%%%%%%%%%%%%%%%%%%%%%%%%%%%
\begin{table}[p]
\begin{center}
\caption{List of fields that are used by the various reference types
         (blank = not used; \X{} = used;
          \Y{} = used, but only for citations).}
\label{tab:fieldref}
\small
\makebox[0pt]{% some manipulation to squeeze the table onto the page
\newlength{\oldtabcolsep}
\setlength{\oldtabcolsep}{\tabcolsep}
\setlength{\tabcolsep}{2pt}
\begin{tabular}{@{}lcccccccccc@{}}
\hline
 & \multicolumn{10}{c}{Reference type}\\
   \cline{2-11}
%
% Normally, I would use the sideways environment of the rotating
% package to format this nicely, but I do not want to require
% other packages to format this document, so I choose an uglier
% solution here.
%
 & article\\
 & magazine   &      &              &            &              &
 & phdthesis\\
Field
 & newspaper  & book & incollection & techreport & intechreport & lecture
 & mastersthesis
 & unpublished
 & misc
 & literal\\
\hline
address           &  &\X&\X&\X&\X&  &\X&\X&\X&  \\
annote            &  &  &  &  &  &  &  &  &  &  \\
author            &\X&\X&\X&\X&\X&\X&\X&\X&\X&  \\
booktitle         &  &  &\X&  &\X&  &  &  &  &  \\
chair             &  &  &  &  &  &\X&  &  &  &  \\
chapter           &  &  &\X&  &  &  &  &  &  &  \\
day               &\X&  &\X&\X&\X&\X&  &\X&\X&  \\
edition           &  &\X&\X&\X&\X&  &\X&  &  &  \\
editor            &\X&\X&\X&\X&\X&  &  &\X&\X&  \\
englishtitle      &\X&\X&\X&\X&\X&\X&\X&\X&\X&  \\
firstkey          &\Y&\Y&\Y&\Y&\Y&\Y&\Y&\Y&\Y&\Y\\
howpublished      &\X&  &\X&\X&\X&\X&\X&\X&\X&  \\
institution       &  &  &  &\X&\X&  &  &  &  &  \\
journal           &\X&  &  &  &  &  &\X&  &  &  \\
key               &\Y&\Y&\Y&\Y&\Y&\Y&\Y&\Y&\Y&\Y\\
month             &\X&  &\X&\X&\X&\X&  &\X&\X&  \\
note              &\X&\X&\X&\X&\X&\X&\X&\X&\X&  \\
number            &\X&\X&  &\X&\X&  &\X&\X&\X&  \\
organization      &  &  &  &  &  &  &  &\X&  &  \\
originaladdress   &\X&\X&\X&  &\X&  &  &  &  &  \\
originalbooktitle &\X&\X&\X&  &\X&  &  &  &  &  \\
originaledition   &\X&\X&\X&  &\X&  &  &  &  &  \\
originaleditor    &\X&\X&\X&  &\X&  &  &  &  &  \\
originaljournal   &\X&\X&\X&  &\X&  &  &  &  &  \\
originalnumber    &\X&\X&\X&  &\X&  &  &  &  &  \\
originalpages     &\X&\X&\X&  &\X&  &  &  &  &  \\
originalpublisher &\X&\X&\X&  &\X&  &  &  &  &  \\
originalvolume    &\X&\X&\X&  &\X&  &  &  &  &  \\
originalyear      &\X&\X&\X&\Y&\X&\Y&\X&\Y&\Y&\Y\\
pages             &\X&  &\X&  &\X&  &\X&  &  &  \\
publisher         &  &\X&\X&  &  &  &  &  &\X&  \\
school            &  &  &  &  &  &  &\X&  &  &  \\
series            &  &  &  &  &  &  &  &  &  &  \\
symposium         &  &  &  &  &  &\X&  &  &  &  \\
text              &  &  &  &  &  &  &  &  &  &\X\\
title             &\X&\X&\X&\X&\X&\X&\X&\X&\X&  \\
translator        &\X&\X&\X&\Y&\X&\Y&\Y&\Y&\Y&\Y\\
type              &\X&\X&\X&\X&\X&\X&\X&\X&\X&  \\
volume            &\X&\X&\X&\X&\X&  &\X&  &  &  \\
year              &\X&\X&\X&\X&\X&\X&\X&\X&\X&\Y\\
\hline
\end{tabular}%
\setlength{\tabcolsep}{\oldtabcolsep}
}
\end{center}
%
\end{table}
%%%%%%%%%%%%%%%%%%%%%%%%%%%%%%%%%%%%%%%%%%%%%%%%%%%%%%%%%%%%%%%%%%%%%%%

The following entry types (reference types) are defined in \pkg{apacite}:
\begin{description}
    \item[\entryname{article}\LC] \mbox{}\\
        A journal article or comparable. If the
        ``article'' referenced to is a special issue of a journal or
        something else that has editors instead of authors, the
        \fieldname{author} field should be empty and the \fieldname{editor}
        field should be used for the editors. If the journal paginates
        by issue instead of by volume, or when you are referring to a
        whole special issue, the issue number should be given
        in the \fieldname{number} field. In all other cases (including
        referring to an article within a special issue), the
        \fieldname{number} field must not be used. The \fieldname{type}
        field can be used to denote the type of article, for example,
        ``\verb+Letter to the editor+'', or
        ``\verb+Review of the book \emph{Life in the middle}+''.

    \item[\entryname{magazine}] \mbox{}\\
        A magazine article. Unlike in previous versions of \pkg{apacite},
        this is now equivalent to \entryname{article}. However, with
        an \entryname{article}, the \fieldname{month} and \fieldname{day}
        fields should generally not be used, unless there is a compelling
        reason to use them. For a magazine, the \fieldname{month}, and
        \fieldname{day} (for weeklies) fields are commonly used.

    \item[\entryname{newspaper}] \mbox{}\\
        A newspaper article. This is similar to \entryname{article} and
        \fieldname{magazine}, except that the pages are formatted a little
        different, with ``pp. 23--49'' instead of just ``23--49''.

    \item[\entryname{book}\LC] \mbox{}\\
        An entire book. The \fieldname{type} field can now be used
        to denote a specific type of item (so generally not strictly
        a book), e.g., ``\verb+Brochure+''. Similarly, the
        \fieldname{number} field can be used if this is considered
        useful. (It is used in example 24 in the APA manual, see
        below.) This seems quite rare to me, however. It should
        certainly \emph{not} be used for ISBN numbers and the like.

    \item[\entryname{incollection}\LC] \mbox{}\\
        An article in a (usually edited) book, or other kind of larger
        work, except a report, for which the \entryname{intechreport}
        entry is used. The \fieldname{booktitle} field contains the
        title of the whole collection (book). The \fieldname{type}
        field is currently used to denote the type of article, as
        with the \entryname{article} reference type, although in some
        cases, it may be more naturally to let the \fieldname{type}
        field contain the type of the whole work (e.g., CD). Therefore,
        I may change this somewhat in future versions of \pkg{apacite}.

    \item[\entryname{techreport}\LC] \mbox{}\\
        A report. This may be a ``technical'' report such as published
        by universities, or a report from government organizations or
        private companies. The \fieldname{type} field can be used to
        indicate what kind of report it is, e.g.,
        ``\verb+College Board Rep.+'' or ``\verb+{DHHS} Publication+''.
        If the \fieldname{type} field is missing, the default type
        ``\verb+Tech.\ Rep.{}+'' is used. If no type description should
        be given, \verb+\bibnotype+ should be used as the contents of
        the \fieldname{type} field. The \fieldname{number} field can
        be used to give the report number. The organization that
        published the report should be given in the \fieldname{institution}
        field.

    \item[\entryname{intechreport}] \mbox{}\\
        An article in a (usually edited) report. This is more or less
        a combination of \entryname{incollection} and
        \entryname{techreport}. It uses the same fields as the former
        to describe the article itself, except \fieldname{type}, e.g.,
        \fieldname{booktitle}, which is in this case the title of the
        entire report,
        and the same fields as the latter to describe the report.
        In particular, the \fieldname{type}, \fieldname{number},
        and \fieldname{institution} fields are used for the report.

    \item[\entryname{lecture}] \mbox{}\\
        A paper presented at a meeting. According to the APA manual,
        the year and month should be given, but \pkg{apacite} also
        uses the \fieldname{day} field when available.
        The \fieldname{symposium} field can be used for the
        name of the symposium and the \fieldname{chair} field for the chair of
        the meeting. The \fieldname{howpublished} field should be used to
        indicate the occasion at which the paper was presented if the
        \fieldname{symposium} field is empty, e.g.,
        ``\texttt{Paper presented at the meeting of the American
                  Professional Society on the Abuse of Children,
                  San Diego, CA}''.
        It can also be used to give additional information about
        the symposium if the \fieldname{symposium} field is not empty.

    \item[\entryname{phdthesis}\LC] \mbox{}\\
        A doctoral dissertation. The \fieldname{school} field is used
        to denote the university for which the thesis was written.
        The \fieldname{type} field can be used for the thesis type,
        e.g., ``\verb+PhD thesis+''. If it is empty, the default type
        is used, which is ``\verb+Unpublished doctoral dissertation+''
        or ``\verb+Doctoral dissertation+'', depending on whether
        the \fieldname{journal} field is empty or not.

        If the thesis is abstracted in \emph{Dissertation Abstracts
        International} or similar, then the \fieldname{journal},
        \fieldname{volume}, \fieldname{number}, and \fieldname{pages} fields
        can be used as with journal articles. The \fieldname{year} field
        should denote the year of the ``journal''.
        The \fieldname{originalyear} field should be used to denote the year
        of the original thesis, even if it is the same as the year of the
        abstract, provided that the original thesis is used, and not
        the abstract. Then, the \fieldname{school} field should also
        be used, and optionally the \fieldname{type} field. If only
        the abstract is used, it should presumably be treated as a
        journal article.


    \item[\entryname{mastersthesis}\LC] \mbox{}\\
        This is equivalent to the \entryname{phdthesis} type, except that
        the default ``unpublished'' and ``published'' \fieldname{type}s are
        ``\verb+Unpublished master's thesis+'' and
        ``\verb+Master's thesis+''.

    \item[\entryname{unpublished}\LC] \mbox{}\\
         For unpublished manuscripts and similarly ``obscure'' material.
         The \fieldname{howpublished} field will typically be used to
         indicate what kind of work is referred to (e.g.,
         ``\verb+Unpublished manuscript+''). The \fieldname{organization}
         field can be used to denote the organization in which the document
         was produced, e.g., ``\texttt{Johns Hopkins University, Center
         for Social Organization of Schools}''. Note that the larger
         organization should be given first and the department after that.
         The address (city and state, etc., see the discussion of the
         \fieldname{address} field in section~\ref{subsec:fields} below)
         of the organization should be given in the \fieldname{address}
         field, unless it is already mentioned in the name of the
         organization. Presumably, if the organization is
         ``\verb+University of Groningen+'', the city name ``Groningen''
         should not be given in the \fieldname{address} field, but
         the country ``The Netherlands'' should. When both the
         \fieldname{organization} and the \fieldname{address} field
         are available, \pkg{apacite} formats these as
         ``organization, address''. This is different from the
         usual ``address: publisher'' form. This behavior cannot
         explicitly be deduced from the 5th edition, nor from the
         4th edition, of the APA manual, but the 3rd edition
         \cite{APAManual3} gives an explicit example (example 53,
         p.~131): \citeA{APA-3rd-ed-ex53}.

    \item[\entryname{misc}\LC] \mbox{}\\
        For works that do not fit into the other categories, such as
        motion pictures, cassette recordings, computer software, etc.
        The \fieldname{type} field can be used to indicate the type
        of work, and \pkg{apacite} recognizes a few specific types
        that should be formatted a little differently, see the discussion
        of the \fieldname{type} field below. If the \fieldname{address}
        and \fieldname{publisher} fields are available, the address and
        publisher part is formatted as with books. The \entryname{unpublished}
        and \entryname{misc} types are very similar, but there are some
        differences. For example, \entryname{unpublished} uses the
        \entryname{organization} field and \entryname{misc} uses the
        \fieldname{publisher} field, and this part is formatted differently.
        Further, the \entryname{unpublished} type requires
        \fieldname{author}, \fieldname{editor}, \fieldname{title}, or
        \fieldname{type}, whereas with \fieldname{misc},
        \fieldname{howpublished} takes the first position if these fields
        are all empty. Finally, as briefly indicated above and discussed
        in more detail below, \entryname{misc} recognizes some special
        types in the \fieldname{type} field.

    \item[\entryname{literal}] \mbox{}\\
        If the other categories do not format the item correctly, this
        category can be used. The \fieldname{text} field is copied literally
        to the bibliography. The \fieldname{firstkey}, \fieldname{key},
        \fieldname{year}, and possibly \fieldname{originalyear} and
        \fieldname{translator} fields are necessary to get correct
        in-text citations. However, I have never needed to use this
        entry type and I think that \entryname{misc} should also
        be able to format the entry correctly. Furthermore, the
        correct placement of \entryname{literal} references in the
        reference list (alphabetizing, sorting), may be problematic.

    \item[\entryname{booklet}\LC]
    \item[\entryname{inbook}\LC]
    \item[\entryname{inproceedings}\LC]
    \item[\entryname{manual}\LC]
    \item[\entryname{proceedings}\LC] \mbox{}\\
        These categories are defined by all standard citation styles. They
        are, however, not needed for the examples in the APA manual. For
        compatibility, however, they are included and defined as follows:
        \entryname{booklet} = \entryname{manual} = \entryname{proceedings} =
        \entryname{misc}; \entryname{inbook} = \entryname{inproceedings} =
        \entryname{incollection}.
\end{description}

%%%%%%%%%%%%%%%%%%%%%%%%%%%%%%%%%%%%%%%%%%%%%%%%%%%%%%%%%%%%%%%%%%%%%%%
\subsection{Fields}
\label{subsec:fields}
%%%%%%%%%%%%%%%%%%%%%%%%%%%%%%%%%%%%%%%%%%%%%%%%%%%%%%%%%%%%%%%%%%%%%%%
The following fields can be used to describe a reference in the
\fname{.bib} file:
\begin{description}

    \item[\fieldname{address}\LC] \mbox{}\\%
       The address (usually the city and state or country) of the
       publisher, school, institution, or organization that published
       the item or at which the item was produced. The APA manual
       requires that the state or territory of a U.S.-city must be given
       in the official two-letter U.S. Postal Services form. Only for
       a specific list of 17 specific cities that are ``well known
       for publishing'' can (must) the state and/or country
       description be omitted.

    \item[\fieldname{annote}\LC] \mbox{}\\
       This is used in some annotated bibliography
       styles. It is not used by \pkg{apacite}, but no warning
       is given either. It is thus simply ignored.

    \item[\fieldname{author}\LC] \mbox{}\\
       The author(s) of the work. This may also be a corporate
       author when applicable, but some specific measures must
       then be taken to prevent the corporate name from being
       interpreted as a first name and last name
       (such as ``Association, A. P.''), see the examples.

       As discussed in the standard \BibTeX{} documentation,
       multiple authors must be separated by the word \verb+and+,
       and each author's name can either be given in the form
       ``\opt{firstnames}\verb+ +\opt{lastnames}'' or
       ``\opt{lastnames}\verb+, +\opt{firstnames}''. However,
       ``von'' parts and ``junior'' parts complicate issues
       a bit. See the examples and the discussion of them.

       I would generally give the full first name(s) in the \fname{.bib}
       file, even though the APA manual only requires initials,
       because other styles require full first name(s) and
       \pkg{apacite} abbreviates to initials automatically.
       I have done this only occasionally with the examples,
       because the APA manual only gives the initials.
       Therefore, I have only provided first names with a few
       examples for which I knew the authors' first names.

       If there are more than 6 authors, the APA manual requires
       that the first 6 should be named in the reference list,
       followed by ``et al.''. Therefore, \pkg{apacite} uses
       only the names of the first 6 authors, and inserts an
       ``et al.'' when applicable. Therefore, in the \fname{.bib}
       file, the first 6 (or more) authors may be given (separated by
       ``\verb+and+''), followed by ``\verb+and others+''. This phrase
       is recognized by \pkg{apacite}.
       It is, however, better to give all authors in
       the \fname{.bib} file, so that each style can select its
       own truncation number.

    \item[\fieldname{booktitle}\LC] \mbox{}\\
       The title of the larger work, typically book, but sometimes
       something else, like a report, in which the referenced item
       (article) was published.

    \item[\fieldname{chair}] \mbox{}\\
       The chair(s) of a symposium or meeting. Used for lectures.
       It is formatted the same way as editors.

    \item[\fieldname{chapter}\LC] \mbox{}\\
       The chapter number if the referenced item is a chapter in
       a larger collection. Typically used if the referenced work
       is a chapter in an internet document. If the collection is
       a book, page numbers (in the \fieldname{pages} field)
       should be used instead according to the APA rules.

    \item[\fieldname{crossref}\LC] \mbox{}\\
       The \pkg{apacite} package does not recognize this field,
       because the APA manual does not discuss explicit
       cross-referencing. If the referenced work is a chapter
       (article) in an edited book, all relevant information
       should be given in the reference list as part of the
       information about the referenced chapter, not as, e.g.,
       ``In Wainrib (1992)''. If several chapters from the
       book edited by Wainrib are referenced, the same information
       about this book is given with each referenced chapter,
       and the work itself is not a separate entry in the reference
       list (unless it is explicitly referred to).

       However, the \fieldname{crossref} field
       \emph{can} be used, as part of standard \BibTeX{} usage.
       Missing fields for the referenced work are then copied from
       the cross-referenced entry, which must come \emph{after}
       the referring entry in the \fname{.bib} file. If you use
       this, remember to put the title of the book (also) in the
       \fieldname{booktitle} field, because the \fieldname{title}
       field of the referring entry is not empty (it contains
       the chapter title).

       I have not tested cross-referencing in detail, so I do not
       know if it behaves well (i.e., complies with the implicit
       or explicit APA rules) under various circumstances.
       Therefore, I do not recommend using it without thorough
       testing.

    \item[\fieldname{day}] \mbox{}\\
       The day of the month on which the referenced item was published,
       produced, or presented. Mainly used for articles in daily or
       weekly magazines or newspapers, for lectures (although the
       APA manual only specifies the month), and electronic documents.

    \item[\fieldname{edition}\LC] \mbox{}\\
       The edition of the book or report. This must be of the form 1st, 2nd,
       3rd, etc., or ``Rev.'' for a revised edition. Future work may provide
       routines to handle numbers only (1, 2, 3, etc.), so that
       language-specific texts are avoided.

    \item[\fieldname{editor}\LC] \mbox{}\\
       The editor(s) of a book, report, or special issue of a journal.
       In the examples, this field is also ``misused'' for the
       producer of a television series.

    \item[\fieldname{englishtitle}] \mbox{}\\
       The English translation of the title of an item with a non-English
       title. The APA manual requires that if the referenced work has a
       non-English title, an English translation should be given as well. For
       an article in a journal with a non-English name or in a book with a
       non-English title, the journal name or book title should not be
       translated or put in this field, only the title of the referenced work
       itself must be translated. Because, as of version
       \APACversiondate{2005/06/01}, \pkg{apacite} contains some support of
       other languages for the main document (see
       section~\ref{sec:compat-babel}), this field needs rethinking, which
       will be deferred to a future version.

    \item[\fieldname{firstkey}] \mbox{}\\
       The \fieldname{firstkey} field, if not empty, is used as
       ``author'' for the first citation to an item. Subsequent citations
       then use the \fieldname{key} field. This can be used if there is
       no author or editor field that can be used for citations, or in
       certain cases with corporate authors, where the citation in the
       text uses an abbreviation of the author's name for second and
       subsequent citations, where the abbreviation is introduced in the
       first citation. It can also be used to ``trick'' the system
       in difficult cases. See also \fieldname{key}.

    \item[\fieldname{howpublished}\LC] \mbox{}\\
       A description of how something was published. For example,
       ``Unpublished manuscript'' or ``Paper presented at the meeting
       of the American Professional Society on the Abuse of Children,
       San Diego, CA''. Also used for retrieval information about
       electronic documents, e.g., ``Retrieved October 13, 2001, from
       \url{http://jbr.org/articles.html}\bibnodot{.}''. See also
       the \fieldname{note} field.

    \item[\fieldname{institution}\LC] \mbox{}\\
       The institution, university, or company that published a
       (technical) report.

    \item[\fieldname{journal}\LC] \mbox{}\\
       The journal, magazine, newspaper, etc.{} in which an
       article, a review, or an abstract of a thesis was published.

    \item[\fieldname{key}\LC] \mbox{}\\
       See \fieldname{firstkey}. The \fieldname{key} field, if not
       empty, is used for second and subsequent citations, or all citations
       if \fieldname{firstkey} is missing. If \fieldname{author} is
       empty and \fieldname{editor} is empty or can not be used as
       author (e.g., in \entryname{incollection}), the \fieldname{key}
       field may be necessary to obtain a useful citation. This is
       so, because the APA manual requires an abbreviation of the
       title as in-text citation, and the abbreviation must be
       sensible. In the current version, \pkg{apacite} uses the
       whole title if \fieldname{firstkey} and \fieldname{key} are missing.
       Therefore, if the title is considered too long to use in citations,
       the user must provide an abbreviation in the \fieldname{key} field.
       Furthermore, the user must then also define the formatting:
       If the key is used for a difficult kind of author (e.g.,
       corporate author), it should be formatted as an author,
       i.e., in plain text, with names capitalized.
       If the key is used for an abbreviation of an article
       title, it should be in plain text, with major words capitalized
       (unlike in the reference list), and between double quotes
       (`` and ''; also unlike in the reference list).
       If the key is used for an abbreviation of a book title, it
       should be emphasized, with major words capitalized
       (unlike in the reference list), but not between quotes.

       In a field in the \fname{.bib} file that is also
       used for the citations, such as the \fieldname{key} field,
       formatting through, e.g.,
\begin{verbatim}
  key = {{\itshape Text}},
  key = {{\em Text\/}},
\end{verbatim}
       can be done (note the extra pair of braces!), but the
       seemingly more logical variations
\begin{verbatim}
  key = {\textit{Text}},
  key = {{\textit{Text}}},
  key = {\emph{Text}},
  key = {{\emph{Text}}},
\end{verbatim}
       do not work.
       However, it is generally preferable to use as less explicit
       formatting in the \fname{.bib} files as possible, because this
       decreases the possibilities of successfully using the
       same \fname{.bib} file with other styles. The following works
       excellent:
\begin{verbatim}
  key = {{\APACcitebtitle{Text}}},
\end{verbatim}
       where the \cmd{\APACcitebtitle} macro recognizes that the argument
       should be formatted as a booktitle (i.e., in italics according to the
       APA rules). This macro, and its companion \cmd{\APACciteatitle} for
       article title formatting, are used by \pkg{apacite} if the
       \fieldname{firstkey} and \fieldname{key} fields are missing,
       but can also be used by the user. Note that if you use such a
       \fname{.bib} file with another style, you have to provide
       definitions of these macros yourself.

       \emph{Warning:} the \fieldname{key} field, which acts
       as a kind of pseudo-author, should not be confused with the
       (citation) \opt{key} that is used to match citations with
       entries in a \fname{.bib} file.

    \item[\fieldname{month}\LC] \mbox{}\\
       The month something was published. Mainly used for
       magazine or newspaper articles, lectures, and electronic documents.
       Use the month macros \verb+jan+, \verb+feb+, etc. These are
       predefined strings in \fname{apacite.bst} and \fname{apacitex.bst},
       but can be overridden when desired, e.g., when using another
       language (see section~\ref{sec:compat-babel}).

    \item[\fieldname{note}\LC] \mbox{}\\
       A note. This puts additional information between parentheses
       at the end of a reference list entry. In the examples from
       the APA manual, it is used for NTIS No., ERIC No., and UMI No.,
       and for notes about how to obtain the work
       (``Available from \opt{organization}, \opt{full address}'').
       If however, the work has been obtained from the internet
       or from an ``aggregated database'', then the
       \fieldname{howpublished} field should be used, because it
       should not be between parentheses then.

       Finally, the \fieldname{note} field is used to give the
       recording date (i.e., year) of a song when this is different
       from the date (year) of copyright. The latter should be
       put in the \fieldname{year} field.

       Do not use the \fieldname{note} field to denote the original
       publication of a reprint or translation, because the
       \fieldname{originalyear} field must be used in those cases to
       obtain the correct citation in the text \cite<e.g.,>{ex39}.

    \item[\fieldname{number}\LC] \mbox{}\\
       The number of a journal issue or a report. This can be quite
       complex, for example, ``\verb+PRM~92-01+'', or
       ``\verb+3, Pt.~2+'', or ``\verb+1, Serial No.~231+''.
       These latter examples imply that the generality (language and
       style independence) of the \fname{.bib} file is somewhat lost.
       Furthermore, if the journal paginates by year and not by issue,
       the issue number should not be mentioned at all (except when
       referring to a whole special issue). Of course, \pkg{apacite}
       does not know whether the journal paginates by issue or by year.
       If the \fieldname{number} field is available, \pkg{apacite}
       simply assumes that it should be included. It is the user's
       responsibility to ensure that this is indeed the case.

    \item[\fieldname{organization}\LC] \mbox{}\\
       Used for the \entryname{unpublished} reference type to denote
       the organization in which the unpublished work was produced.

    \item[\fieldname{originaladdress}]
    \item[\fieldname{originalbooktitle}]
    \item[\fieldname{originaledition}]
    \item[\fieldname{originaleditor}]
    \item[\fieldname{originaljournal}]
    \item[\fieldname{originalnumber}]
    \item[\fieldname{originalpages}]
    \item[\fieldname{originalpublisher}]
    \item[\fieldname{originalvolume}]
    \item[\fieldname{originalyear}] \mbox{}\\
        These (\fieldname{original*-}) fields have the same
        meaning as their counterparts without the ``original'' prefix,
        except that they refer to the book or journal in which the
        work was originally published. This can be used in referring to
        translations or reprints of articles in journals or books. If the
        original work is a book or other ``standalone'' work,
        only the \fieldname{originalyear} field should be used. If the
        original work is a (PhD or Master's) thesis that is also abstracted
        in \emph{Dissertation Abstracts International},
        \emph{Masters Abstracts International}, or comparable, the
        \fieldname{originalyear} field should be used to denote the year
        of the dissertation and the \fieldname{year} field to
        denote the year of publication of the abstract, even if they
        are the same.

    \item[\fieldname{pages}\LC] \mbox{}\\
        The (inclusive) page numbers of the article that is referred to,
        in the journal or book in which it was published. If a range of
        pages is given (which is usually the case), an  en-dash should
        be used: ``\verb+29--43+''. Other styles sometimes allow a
        single hyphen (``\verb+29-43+''), which is then automatically
        formatted as an en-dash (``29--43''), rather than as a hyphen
        (``29-43''), but I have encountered situations, esp.\ software
        manuals, in which the page number was of the form ``II-3'',
        meaning page~3 of chapter~2. In such a situation, you may
        get page numbers like ``II-1--II-15'', where the distinction
        between a hyphen and an en-dash becomes important. Therefore,
        \pkg{apacite} does not perform such automatic transformations.

    \item[\fieldname{publisher}\LC] \mbox{}\\
        The publisher of the item. Primarily used for books.

    \item[\fieldname{school}\LC] \mbox{}\\
        The school or university for which a PhD thesis or
        master's thesis was written.

    \item[\fieldname{series}\LC] \mbox{}\\
        Not used, although the APA manual has a rule for series.
        If you refer to a volume in a series, the series title,
        volume number, and volume title should be joined into
        a two-part title , e.g., \emph{Handbook of child psychology:
        Vol.~4. Socialization, personality, and social development},
        see ex.~36. In the current version of \pkg{apacite} this
        whole part must be put in the \fieldname{title} or
        \fieldname{booktitle} field (whichever is appropriate).
        Maybe in a next version I will use the \fieldname{series}
        field to allow disentangling such situations.

    \item[\fieldname{symposium}] \mbox{}\\
        The name of the symposium or meeting at which a
        lecture was given. This is typically used for the
        construction ``In \opt{chair} (Chair), \opt{symposium}'',
        see ex.~51. I have never used it myself, however,
        I always use the ``Paper presented \ldots'' form,
        which puts this information in the \fieldname{howpublished}
        field.

    \item[\fieldname{text}] \mbox{}\\
        Used for items of type \entryname{literal}. This field contains
        the complete literal text to be used in the bibliography.

    \item[\fieldname{title}\LC] \mbox{}\\
        The title of the work.

    \item[\fieldname{translator}] \mbox{}\\
        The translator of a book or article. This should
        be formatted in the same way as \fieldname{author} and
        \fieldname{editor}. If the \fieldname{translator} and
        \fieldname{editor} fields are identical, they are
        formatted as if there were only an editor, except that the
        editor receives the suffix ``(Ed. \& Trans.)'' instead of
        just ``(Ed.)'', or similarly if there is more than 1 editor.

    \item[\fieldname{type}\LC] \mbox{}\\
        The type of \entryname{phdthesis} (e.g.,
        ``Doctoral dissertation''), type of \entryname{article}
        (e.g., ``Letter to the editor''), type of \entryname{techreport}
        (e.g., ``DHHS Publication''), type of \entryname{book}
        (e.g., ``Brochure''), type of \entryname{misc}
        (e.g., ``Cassette recording''), and so forth.

        With a review (which is typically an \entryname{article}),
        the \fieldname{type} field contains a lot of information
        and formatting, e.g.:
\begin{verbatim}
  type = {Review of the book {\APACcitebtitle{Life in the middle:
            Psychological and social development in middle age}}},
\end{verbatim}
        where I used the \cmd{\APACcitebtitle} command introduced above in
        the discussion of the \fieldname{key} field.

        In one of the APA manual examples, I used the \fieldname{type}
        field to give a description of the subject of an unpublished
        raw data file (``Auditory response latencies in rat auditory
        cortex''), leaving the \fieldname{title} field blank.

        For most reference types, if the \fieldname{type} field
        is missing, it is simply ignored and no type description
        is given. For a \entryname{techreport}, however, if the
        \fieldname{type} field is missing, the default type
        (``Tech. Rep.'') is inserted. In ex.~42 of the APA manual,
        a report without a type description is given. To make it
        possible to format a report successfully without a
        type description, \pkg{apacite} recognizes
\begin{verbatim}
  type = {\bibnotype},
\end{verbatim}
        indicating that there should be no type description.

        Another special purpose use of the \fieldname{type}
        field is for a computer program, software, programming
        language and/or manual. These are put into a \entryname{misc}
        entry, but unlike other \entryname{misc} types (such as
        motion pictures) the titles of these types should not
        be italicized. This is now recognized by \pkg{apacite}
        in the following way: When referring to a computer program,
        software, programming language and/or manual, you should
        use the \entryname{misc} reference type with
        one of the following commands in the \fieldname{type} field:\\
        \cmd{\bibcomputerprogram},\\
        \cmd{\bibcomputerprogrammanual},\\
        \cmd{\bibcomputerprogramandmanual},\\
        \cmd{\bibcomputersoftware},\\
        \cmd{\bibcomputersoftwaremanual},\\
        \cmd{\bibcomputersoftwareandmanual}, or\\
        \cmd{\bibprogramminglanguage}.\\
        Using any of these commands as type designator of a
        \entryname{misc} entry ensures that the title is not italicized,
        as required. Note that the \entryname{manual} entry reduces to
        \entryname{misc}, so can be used sensibly, but only if the
        \fieldname{type} field is defined as above. Perhaps it would
        be logical to define a default type for this, but this is not
        implemented. Anyway, adding a \fieldname{type} field to a
        \entryname{manual} entry in a \fname{.bib} file will probably
        do not any harm with other bibstyles, so this is probably only
        a minor nuisance.

        Similarly, if you refer to a message that has been posted to
        a newsgroup, internet forum, etc., you should use the
        \entryname{misc} reference type with \cmd{\bibmessage} as
        \fieldname{type} field.

    \item[\fieldname{volume}\LC] \mbox{}\\
        The volume or volumes of the referenced book(s) or
        of the book in which the referenced article was published,
        or the volume of the journal (magazine, newspaper, \ldots)
        in which the referenced article was published.

    \item[\fieldname{year}\LC] \mbox{}\\
        The year in which the referenced item was published, or
        if it was not published, the year in which it was written
        or presented. For manuscripts or books that are accepted
        for publication but have not yet been published, ``in press''
        should be used according to the APA rules. The best way to do
        this is to use the ``\cmd{\BIP}'' command, which can be redefined
        by the user if the language or editorial style requires
        something else than ``in press''. Furthermore, \cmd{\BIP}
        is recognized by \pkg{apacite} and treated a little differently,
        because for multiple ``in press'' references with the same
        author(s), a hyphen (``-'') should be inserted between the
        ``year'' (i.e., ``in press'') and the ``a'', ``b'', etc., that
        follow it to distinguish the works, whereas this hyphen must
        be omitted if the ``year'' is an ordinary year (1991a, 1991b).

        If no date is explicitly given in (or on) the referenced work,
        a ``n.d.'' (no date) should be given as year description.
        For this, \pkg{apacite} supports the \cmd{\bibnodate}
        command. This sometimes leads to different formatting as well,
        e.g., by referencing a translation of a work of which there
        is no original date. Then the \fieldname{originalyear}
        field should be \cmd{\bibnodate} and the \fieldname{year} field
        is then, e.g., \verb+1931+. If the \fieldname{translator}
        field is not empty, the citation in the text is then formatted
        as, e.g., ``\citeA{3.100-2}''.
\end{description}


%%%%%%%%%%%%%%%%%%%%%%%%%%%%%%%%%%%%%%%%%%%%%%%%%%%%%%%%%%%%%%%%%%%%%%%
\section{Customization}
\label{sec:custom}
%%%%%%%%%%%%%%%%%%%%%%%%%%%%%%%%%%%%%%%%%%%%%%%%%%%%%%%%%%%%%%%%%%%%%%%
The description in the previous sections, with the exception of
section~\ref{sec:usage}, almost exclusively discussed the default behavior of
\pkg{apacite}. However, as mentioned in the introduction, in addition to the
options, \pkg{apacite} offers many possibilities for customization. Most
punctuation used in the citations and reference list are implemented through
\LaTeX{} commands instead of explicit symbols. Consequently, the user can
fine-tune the behavior of \pkg{apacite} by redefining these commands, through
\cmd{\renewcommand} after \pkg{apacite} has been loaded. Analogously, most
fixed texts, like ``Tech.\ Rep.{}'' and ``Eds.{}'' are implemented through
\LaTeX{} commands as well, and can similarly be changed by the user. The
commands used by \pkg{apacite} are discussed in this section. Of course, the
defaults are based on the rules of the APA manual.

Some commands, predominantly punctuation, are used both in citations and in
the reference list. If you want their definitions in citations to be different
from their definitions in the reference list, you can simply redefine their
definitions before starting the bibliography.

%%%%%%%%%%%%%%%%%%%%%%%%%%%%%%%%%%%%%%%%%%%%%%%%%%%%%%%%%%%%%%%%%%%%%%%
\subsection{Punctuation and formatting}
%%%%%%%%%%%%%%%%%%%%%%%%%%%%%%%%%%%%%%%%%%%%%%%%%%%%%%%%%%%%%%%%%%%%%%%
The following punctuation commands are provided and used for
the citations and reference list.
\begin{description}
  \item[\cmd{\BAstyle}] This defines the text style of the authors
      (or whatever takes their place) for an in-text citation. It
      defaults to nothing: The authors use the same fonts as the
      text surrounding it. Introduced because some journals use
      a different style. For example, \emph{Statistica Neerlandica}
      uses small-caps, so for that journal, you would define
\begin{verbatim}
\renewcommand{\BAstyle}{\scshape}
\end{verbatim}
      Do not use the \cmd{\textsc}-type commands, but \cmd{\scshape},
      \cmd{\bfseries}, etc. The \cmd{\BAstyle} command is used for
      the author-part by the citation commands
      \cmd{\cite},   \cmd{\shortcite},   \cmd{\fullcite},
      \cmd{\citeA},  \cmd{\shortciteA},  \cmd{\fullciteA},
      \cmd{\citeNP}, \cmd{\shortciteNP}, and \cmd{\fullciteNP}.

  \item[\cmd{\BAastyle}] This is the same as \cmd{\BAstyle}, except that
      \cmd{\BAastyle} is used for \cmd{\citeauthor}, \cmd{\shortciteauthor},
      and \cmd{\fullciteauthor}.

  \item[\cmd{\BBOP}] Open parenthesis, used for parentheses opening
      a citation, as in ``(Rao, 1973)'' or ``Rao (1973)'', and the
      year in the reference list, as in ``Rao, C. R. (1973).''
      Default is ``\verb+(+''.

  \item[\cmd{\BBCP}] The corresponding closing parenthesis.
      Default is ``\verb+)+''.

  \item[\cmd{\BAP}] This command is inserted after the prefix and before
      the first citation in a \cmd{\cite} command. It defaults to an
      ordinary space.

  \item[\cmd{\BBAA}] Last ``and'' between authors in a citation
      between parentheses and in the reference list, as in
      ``(Mooijaart \& Bentler, 1986)'' or ``Mooijaart, A., \&
      Bentler, P. M. (1986).''. Default is ``\verb+\&+''.

  \item[\cmd{\BBAB}] Last ``and'' between authors in a citation in text,
      as in ``Mooijaart and Bentler (1986)''. Default is ``\verb+and+''.

  \item[\cmd{\BBAY}] Punctuation between author(s) and year in a citation
      between parentheses, as in ``(Rao, 1973)''. Default is ``\verb+, +''.

  \item[\cmd{\BBYY}] \label{cmd:BBYY}
      Punctuation between two subsequent years, if two works
      by the same author(s) are referenced in a single citation command,
      as in ``(Rao, 1965, 1973)'' or ``Rao (1965, 1973)''.
      Default is ``\verb+, +''.

  \item[\cmd{\BBC}] Punctuation between multiple cites, as in
      ``(Rao, 1973; Mooijaart \& Bentler, 1986)''.
      Default is ``\verb+; +''.

  \item[\cmd{\BBN}] Punctuation before a note (postfix) after
      a citation, as in ``(Rao, 1973, chap.~2)''.
      Default is ``\verb+, +''.

  \item[\cmd{\BBOQ}] Opening quote for an article title in the
      reference list, as in
      ``\flqq Random polynomial factor analysis.\frqq'' Default is
      the empty string: no quotes used.

  \item[\cmd{\BBCQ}] Closing quote for an article title in the
      reference list. Default is the empty string.
      (The \pkg{theapa} package required the closing period to be part
      of the closing quote, but this has been
      changed, because the period must be left out if title comments,
      such as type, English translation of non-English title, or
      translator follow the title.)

  \item[\cmd{\BCBT}] Comma between authors in the reference section when
      the number of authors is two, as in
      ``Mooijaart, A., \& Bentler, P. M. (1986).'' The comma
      after the ``A.'' is this one. Default is ``\verb+,+''.
      The APA manual requires it, but other styles leave out this
      comma. In such a case, you would redefine this ``comma'' to
      be the empty string:
\begin{verbatim}
\renewcommand{\BCBT}{}
\end{verbatim}

  \item[\cmd{\BCBL}] Comma before the last author (for 3 or more authors)
      in a citation and in the reference section, as in ``(Gill, Murray,
      \& Wright, 1981)'' or ``Gill, P. E., Murray, W., \& Wright, M. H.
      (1981).'' Default is ``\verb+,+''. The APA manual requires it,
      which is standard U.S. usage, but other styles, particularly
      European, such as British English, leave out this comma. Again,
      in such a case, you would redefine this ``comma'' to
      be the empty string.

  \item[\cmd{\BAnd}] This is the ``and'' that is used in the reference list
      if someone is both editor and translator:
      ``In J. Strachey (Ed. \& Trans),''. Default is ``\verb+\&+''.

  \item[\cmd{\theBibCnt}] If there are multiple citations with the same
      author and year, a letter should be added to the year to distinguish
      the references. For example, one may refer to two or more articles
      by J. Smith published in 1982. They should be referred to as
      ``Smith (1982a)'', ``Smith (1982b)'', and so forth. To accomplish
      this, the counter \verb+BibCnt+ is defined in \fname{apacite.sty}.
      The \cmd{\theBibCnt} command defines how the value of \verb+BibCnt+
      is formatted. The default is ``\verb+\alph{BibCnt}+''. To
      emphasize (italicize) the ``a'', ``b'', etc., you can redefine
      this as
\begin{verbatim}
\renewcommand{\theBibCnt}{{\em\alph{BibCnt}\/}}
\end{verbatim}
      (This presumably works better than ``\verb+\emph{\alph{BibCnt}}+''.)
      To control this behavior, redefining the \cmd{\theBibCnt} command
      should normally be sufficient. However, the complete formatting
      commands are \cmd{\BCnt}, \cmd{\BCntIP}, and \cmd{\BCntND}. The
      second of these is for ``in press'' works, which need an extra hyphen
      between ``in press'' and the ``a'' and ``b'' suffixes. The third is
      similar, for works without a date. Their default definitions
      in \fname{apacite.sty} are
\begin{verbatim}
\newcommand{\BCnt}[1]{\setcounter{BibCnt}{#1}\theBibCnt}
\newcommand{\BCntIP}[1]{\setcounter{BibCnt}{#1}-\theBibCnt}
\newcommand{\BCntND}[1]{\setcounter{BibCnt}{#1}-\theBibCnt}
\end{verbatim}

  \item[\cmd{\APACciteatitle}] The formatting of the title of an
      article (or similar work)
      when used as a citation in the text when no author or editor
      is available for that purpose. Its default definition
      in \fname{apacite.sty} is
\begin{verbatim}
\newcommand{\APACciteatitle}[1]{``#1''}
\end{verbatim}
      i.e., the title is put between double quotes.

  \item[\cmd{\APACcitebtitle}] The formatting of the title of a book
      (or other independent work)
      when used as a citation in the text when no author or editor
      is available for that purpose. Its default definition
      in \fname{apacite.sty} is
\begin{verbatim}
\newcommand{\APACcitebtitle}[1]{{\em #1\/}}
\end{verbatim}
      i.e., the title is emphasized (in italics) but not put between
      double quotes.

  \item[\cmd{\APACmetastar}] The asterisk that precedes an item in the
      bibliography to denote that it is included in the meta-analysis.
      The default value is ``\verb+$\star$\ +''.

  \item[\cmd{\bibnewpage}] If the \pkgoption{bibnewpage} option
      is chosen, this command is included before the bibliography.
      Its default definition is \cmd{\clearpage}, but \fname{apa.cls}
      uses a similar construction with its \pkgoption{man} option
      through \cmd{\newpage}.
      See \citeA[p.~215]{LaTeXbook} for the differences between
      \cmd{\clearpage} and \cmd{\newpage}.

  \item[\cmd{\bibliographytypesize}] This command is used before
      the reference list, but after the section or chapter heading.
      It is intended for the font size of the reference list:
      For \citeA{WaMe00}, I defined it as \cmd{\small} (and even then
      the reference list took up 34 pages). The default value is
      \cmd{\normalsize}.

  \item[\cmd{\bibleftmargin}] This gives the indentation of the second
      and subsequent lines of a reference list entry, relative to the
      usual left margin. It is not a proper command, but a
      ``skip'' (rubber length), which means that it cannot be changed by
      \cmd{\renewcommand}, but by \cmd{\setlength}. Its default
      value is \verb+2.5em+.

  \item[\cmd{\bibindent}] This gives the indentation of the first
      line of a reference list entry, relative to the second line.
      It is also a ``skip''. Its default value is \verb+-\bibleftmargin+,
      which means that the first line starts at the original left margin,
      and the second and subsequent lines are indented by 2.5em.

  \item[\cmd{\bibitemsep}] This gives the vertical separation
      between two reference list entries. It is also a ``skip'', with
      default value ``\verb+\z@ \@plus .3\p@\relax+'', i.e., 0pt,
      but it may be stretched a little to fill the page nicely.
      This is an old (Plain \TeX) style definition, I will probably
      change that in a next version. If you want to change it,
      e.g., to get a blank line between reference list entries,
      you can use
\begin{verbatim}
\setlength{\bibitemsep}{\baselineskip}
\end{verbatim}
\end{description}

%%%%%%%%%%%%%%%%%%%%%%%%%%%%%%%%%%%%%%%%%%%%%%%%%%%%%%%%%%%%%%%%%%%%%%%
\subsection{Labels}
\label{sec:labels}
%%%%%%%%%%%%%%%%%%%%%%%%%%%%%%%%%%%%%%%%%%%%%%%%%%%%%%%%%%%%%%%%%%%%%%%
There are a lot of specific pieces of text that can be put into the reference
list or a citation by \pkg{apacite}. Here, these pieces are called
\emph{labels}. Almost all of these are to some extent language-specific, and
sometimes style-specific even within the same language. Therefore, they are
implemented through \LaTeX{} commands, so that users can easily change them.
In this section, these commands and their purposes are described, and their
(U.S.\ English) defaults are given. Section~\ref{sec:compat-babel} below will
discuss how these commands are changed if another language is used.

The following label commands are used:
\begin{description}
  \item[\cmd{\bibmessage}] This is the first of 9 specific type
      commands that are recognized by \pkg{apacite}. If you
      define the \fieldname{type} field of a \entryname{misc}
      entry as ``\cmd{\bibmessage}'', the formatting of the
      entry is changed (e.g., the title is not italicized),
      see the discussion of the \fieldname{type} field in
      section~\ref{subsec:fields} above. This one is used for
      messages in newsgroups, internet forums, etc.
      Default is ``\verb+Msg+''.

  \item[\cmd{\bibcomputerprogram}]
      Default is ``\verb+Computer program+''.

  \item[\cmd{\bibcomputerprogrammanual}]
      Default is ``\verb+Computer program manual+''.

  \item[\cmd{\bibcomputerprogramandmanual}]
      Default is ``\verb+Computer program and manual+''.

  \item[\cmd{\bibcomputersoftware}]
      Default is ``\verb+Computer software+''.

  \item[\cmd{\bibcomputersoftwaremanual}]
      Default is ``\verb+Computer software manual+''.

  \item[\cmd{\bibcomputersoftwareandmanual}]
      Default is ``\verb+Computer software and manual+''.

  \item[\cmd{\bibprogramminglanguage}]
      Default is ``\verb+Programming language+''.

  \item[\cmd{\bibnotype}]
      This one is a bit different from the previous 8. It is used for
      \entryname{techreport} entries to indicate that the type specifier
      should be suppressed. Its default value is the empty string, but this
      will not have an effect on \pkg{apacite} behavior, because the type
      specifier is suppressed anyway by \pkg{apacite}. However, by defining it
      as the empty string, formatting may become better (i.e., closer to the
      desired formatting) with other styles.

  \item[\cmd{\bibnodate}] Used in the \fieldname{year} and
      \fieldname{originalyear} fields to indicate that no
      publication date has been given. Default value is ``\verb+n.d.{}+''.
      The extra pair of braces ensures that \TeX\ does not treat
      the period before them as a sentence-ending period, after which
      more space is inserted. This also applies to several other
      labels discussed below.

  \item[\cmd{\BOthers}] Used for ``others'' if the number of authors
      or editors is too large, as in ``(Gill et al., 1981)''.
      The default is ``\verb+et al.{}+''.

      However, the definition contains a slight adaptation, because
      in a previous version of \pkg{apacite}, there was a problem
      implying that in some cases \BibTeX{} might put an extra period after
      \cmd{\BOthers}, not recognizing that it already contains a period,
      so that you would get ``et al.{}.'', which is undesirable.
      It depends on the definition of \cmd{\BOthers} whether there should
      be a period or not: If it is defined as ``\verb+et al.{}+'',
      there should not be an additional period, but if it is defined
      as, say, ``\verb+and others+'', then there should be an additional
      period. But \BibTeX{} does not know what the (later) definition
      of this command will be in \LaTeX{}. Therefore, the \cmd{\BOthers}
      command is defined to have one argument. In \BibTeX{},
      \pkg{apacite} inserts \verb+\BOthers{.}+ in the output, so no
      additional period is inserted by \BibTeX{}. By default, this
      command is defined as ``\verb+et al.{}+'', and the argument is
      simply ignored.

      If you redefine the \cmd{\BOthers} command and need the additional
      period, you can redefine it appropriately. However, in the current
      version, the \cmd{\BOthers} command is distinguished from the
      \cmd{\BOthersPeriod} command and I think the mentioned problems
      do not occur anymore. Nevertheless, I have kept the definition of the
      previous version. I may decide, after thorough testing, to change
      this again in a next version.

      There is still a complication, however. If you use a \cmd{\citeauthor}
      at the end of a sentence, there may be a period too many, if
      you end the sentence explicitly with it and \cmd{\BOthers} is
      ``et al.{}''. On the other hand, if you omit the sentence-finishing
      period but decide to redefine \cmd{\BOthers} to ``and others'',
      the period is missing. Therefore, it seems wise not to end a sentence
      with a \cmd{\citeauthor}, or you could define a command, e.g.,
      \cmd{\finishsentence}, and insert it in an appropriate place.
      Then, this command should be changed along with \cmd{\BOthers}.

  \item[\cmd{\BOthersPeriod}] Also used for ``others'' if the number of
      authors or editors is too large, as in ``Gill, P. E., et al. (1981)''.
      But this one is used for situations when it should always end with
      a period. The default is ``\verb+et al.{}+''.

  \item[\cmd{\BIP}] ``In press'', the string to be used as year for
      in-press references. In the \fname{.bib} file, the \fieldname{year}
      field should be ``\verb+\BIP+'', so that \pkg{apacite} can recognize
      this and use appropriate formatting and sorting.
      The default value is ``\verb+in press+''.

  \item[\cmd{\BIn}] Used for \entryname{incollection} and
      \entryname{intechreport}, for ``In \opt{editor} (Ed.{}),
      \opt{booktitle}'', and similar phrases. Default value is ``\verb+In+''.

  \item[\cmd{\BCHAP}] Used for \entryname{incollection} if the
      \fieldname{pages} field is empty, to denote the chapter number
      of the referenced work in the collection. Used primarily for
      internet documents, where there are no page numbers.
      Default value is ``\verb+chap.{}+''.

  \item[\cmd{\BCHAPS}] Just like \cmd{\BCHAP}, but this one is used
      if the \fieldname{chapter} field refers to more than 1 chapter.
      Default value is ``\verb+chap.{}+''.

  \item[\cmd{\BED}] Editor in reference list, as in
      ``In P. R. Krishnaiah (Ed.{})''. Default is ``\verb+Ed.{}+''.

  \item[\cmd{\BEDS}] Editors (more than one editor) in reference list,
      as in ``In C. R. Rao \& J. K. Ghosh (Eds.{})''.
      Default is ``\verb+Eds.{}+''.

  \item[\cmd{\BTRANS}] Translator, as in ``(J. Strachey, Trans.{})''.
      Default is ``\verb+Trans.{}+''.

  \item[\cmd{\BTRANSS}] Translators. Default is ``\verb+Trans.{}+''.

  \item[\cmd{\BTRANSL}] Translation. Used in a text citation when
      referring to a translated work for which there is no date
      for the original work. This is then formatted as
      ``Aristotle (trans.{} 1931)''. Default is ``\verb+trans.{}+''.

  \item[\cmd{\BCHAIR}] Chair of a symposium, etc.
      Default is ``\verb+Chair+''.

  \item[\cmd{\BCHAIRS}] Chairs. Default is ``\verb+Chairs+''.

  \item[\cmd{\BVOL}] Volume, as in ``Vol.\ 1''.
      Default is ``\verb+Vol.{}+''.

  \item[\cmd{\BVOLS}] Volumes, as in ``Vols.{} 1--4''.
      Default is ``\verb+Vols.{}+''.

  \item[\cmd{\BNUM}] Number, as in ``Tech.\ Rep.{} No.{} 12''.
      Default is ``\verb+No.{}+''.

  \item[\cmd{\BNUMS}] Numbers, as in ``Nos.{} 3--5''.
      Default is ``\verb+Nos.{}+''.

  \item[\cmd{\BEd}] Edition, as in ``2nd ed.{}''.
      Default is ``\verb+ed.{}+''.

  \item[\cmd{\BPG}] Page, default is ``\verb+p.{}+''.

  \item[\cmd{\BPGS}] Pages, default is ``\verb+pp.{}+''.

  \item[\cmd{\BTR}] The (default) technical report type name, which can be
      overridden by the \fieldname{type} field. Default is
      ``\verb+Tech.\ Rep.{}+''

  \item[\cmd{\BPhD}] The (default) PhD thesis type name, which can be
      overridden by the \fieldname{type} field. Default is
      ``\texttt{Doctoral dissertation}''.

  \item[\cmd{\BUPhD}] The (default) Unpublished PhD thesis type name,
      which can be overridden by the \fieldname{type} field. Default is
      ``\texttt{Unpublished doctoral dissertation}''.

  \item[\cmd{\BMTh}] The (default) master's thesis type name, which can be
      overridden by the \fieldname{type} field. Default is
      ``\texttt{Master's thesis}''.

  \item[\cmd{\BUMTh}] The (default) unpublished master's thesis type name,
      which can be overridden by the \fieldname{type} field. Default is
      ``\texttt{Unpublished master's thesis}''.

  \item[\cmd{\BOWP}] ``Original work published'', default is
      ``\texttt{Original work published}''.

  \item[\cmd{\BREPR}] ``Reprinted from'', default is
      ``\texttt{Reprinted from}''.

  \item[\cmd{\refname}] Name of the reference list if it's a section.
      (So it's the section name.) Default is ``\verb+References+''.

  \item[\cmd{\bibname}] Name of the reference list if it's a chapter.
      Default is ``\verb+References+''.

  \item[\cmd{\bibliographyprenote}] This command is used before
      the reference list, but after the section or chapter heading,
      and immediately after \cmd{\bibliographytypesize}.
      It is intended for an optional note just before the references
      in the reference list. It defaults to nothing, but can be changed
      to a list of commands (e.g., if some commands should be different
      in the reference list than in the text) and/or some text.

  \item[\cmd{\APACmetaprenote}] If a meta-analysis is reported, the
      works included in the meta-analysis should in the bibliography be
      preceded by an asterisk, see the description of \cmd{\APACmetastar}
      above. This should then be explained in a note before the references.
      This command contains the text of that note. Default is
      ``\texttt{References marked with an asterisk indicate studies
      included in the meta-analysis.}''.

  \item[\cmd{\authorindexname}] Name (heading) of the author index.
      Default is ``\verb+Author Index+''.
\end{description}

%%%%%%%%%%%%%%%%%%%%%%%%%%%%%%%%%%%%%%%%%%%%%%%%%%%%%%%%%%%%%%%%%%%%%%%%%%%%
\section{Language support}
\label{sec:compat-babel}
%%%%%%%%%%%%%%%%%%%%%%%%%%%%%%%%%%%%%%%%%%%%%%%%%%%%%%%%%%%%%%%%%%%%%%%%%%%%
The APA is, of course, American, and therefore the rules in the APA manual
are also based on the (U.S.) English language. Because \pkg{apacite} was
primarily designed to implement the APA rules, and because I write all
my scientific work in English, \pkg{apacite} never contained explicit
support for other languages. However, most language-specific elements
have been implemented in the form of \LaTeX{} commands, so that users
could define their own \LaTeX{} package in which these elements were
changed. The labels discussed in section~\ref{sec:labels} above are the
most important part of this.

Furthermore, when writing in a non-English language, the \pkg{babel} package
will usually be loaded. The \pkg{babel} package is an extensive package that
facilitates using \LaTeX{} for documents in languages other than English.
There have been some compatibility problems between \pkg{babel} and
\pkg{apacite}, and therefore since 2003 \pkg{apacite} contained some
explicit code to overcome the compatibility problems. I think this works
well, provided that \pkg{apacite} is loaded \emph{after} \pkg{babel}.

Thus, although \pkg{apacite} did not support non-English languages
explicitly, it did facilitate them. Supporting non-English languages is
not a main objective of \pkg{apacite}, because it is not needed for
the APA. Nevertheless, it would be a useful addition. Many journals in
non-English languages and universities in countries where other languages
are spoken base their rules on the APA manual. Therefore, it would be
efficient if, with a few adaptations, \pkg{apacite} could also be used in
these circumstances. For this reason, and based on user requests, the
current version of \pkg{apacite} contains a first attempt at explicit
language support.

The \pkg{apacite} distribution now contains a subdirectory \fname{lang},
which contains files that have names according to the construction
\opt{language}\fname{.apc}, i.e., \fname{english.apc}, \fname{dutch.apc}, etc.
These files contain the language-specific modifications of \pkg{apacite},
mainly translations of terms like ``and'', ``Ed.{}'', etc., and have been
written by \pkg{apacite} users that are (native) speakers of the languages
involved. If the \pkg{babel}, \pkg{german}, or \pkg{ngerman} package is
loaded, \pkg{apacite} is able to determine the language of the document that
is processed. Then the corresponding \fname{.apc} file, when available, is
read. In this way, language-specific elements are changed to the relevant
language. This is done fully automatically, the user does not have to do
anything explicitly. However, sometimes an \fname{.apc} file makes some
assumptions (such as \fname{greek.apc}, which assumes ISO-8859-7 encoding)
or makes some choices that are nontrivial. Therefore, look at the source
code of the \fname{.apc} file if you obtain unexpected results. These files
contain some brief documentation of the specific issues relevant for the
language at hand.

The list with available \fname{.apc} files can be found in the file
\fname{manifest.txt}. One file is defined for each ``language'', which can be
used for several ``dialects'' (in \pkg{babel} terminology). For example,
\fname{english.apc} is also used if the language is ``american''. See the
documentation of the \pkg{babel} and \pkg{(n)german} packages for a list of
dialects of the language files supplied. If an \fname{.apc} file is not
available for the language you need, you can use one of the supplied ones as a
template and write your own. If you send it to me, I can include it in a next
version of \pkg{apacite}.

Note, however, that this form of language support is still very limited.
There are many aspects that are not yet covered, like different forms of
dates (month-day vs.\ day-month). I have recently discovered the
\pkg{babelbib} package, which offers very sophisticated support of
different languages. I will study this and see whether I can use some
of its features in a future version of \pkg{apacite}.

One of the issues that is not yet settled is how to treat the months.
I could define these as commands like \cmd{\January} or according
to a construction like \verb+\bibmonth{1}+. The months are now still
hard-coded in English as macros in \fname{apacite.bst}.
This means that, when using a different language, they can (and must)
be redefined in the \fname{.bib} file, by including lines like:
\begin{verbatim}
  @string{jan = "{\APACSortNoop{01}}January"}
\end{verbatim}
etc., where you replace ``\verb+January+'' by the translation of January
in the desired language. The ``\verb+{\APACSortNoop{01}}+'' ensures that
\BibTeX{} sorts January before, say, April, when that is needed.

It is likely that you would use these month definitions a lot, in which case
an even better solution would be to write a special \fname{.bib} file, say
\opt{language}\fname{.bib}, which only contains these month redefinitions.
Then you could put this file somewhere where \BibTeX{} can find it and include
this before your (other) \fname{.bib} file(s) that contain the actual
references:
\begin{verbatim}
\bibliographystyle{apacite}
\bibliography{language,otherbibfiles}
\end{verbatim}
See \citeA[p.~159]{LaTeXbook} or \citeA[p.~403]{LaTeXcomp} for a discussion
of these issues.

%%%%%%%%%%%%%%%%%%%%%%%%%%%%%%%%%%%%%%%%%%%%%%%%%%%%%%%%%%%%%%%%%%%%%%%
\section{Compatibility}
\label{sec:compat}
%%%%%%%%%%%%%%%%%%%%%%%%%%%%%%%%%%%%%%%%%%%%%%%%%%%%%%%%%%%%%%%%%%%%%%%
Generally speaking, I would want \pkg{apacite} to be compatible with other
packages, of course. However, what does compatibility mean? It can mean
various things:
\begin{enumerate}
\item \label{compat1}%
      The same \fname{.bib} file can be used with
      different \BibTeX{} styles and \LaTeX{} citation packages;
\item \label{compat2}%
      The same citation commands as other citation packages;
\item \label{compat3}%
      Another \LaTeX{} citation package can be used with a
      \fname{.bbl} file that is generated by the \pkg{apacite}
      \BibTeX{} style;
\item \label{compat4}%
      The \pkg{apacite} \LaTeX{} citation package can be used with a
      \fname{.bbl} file that is generated by another \BibTeX{} style;
\item \label{compat5}%
      The \pkg{apacite} \LaTeX{} citation package can be loaded at the
      same time as other packages without errors or undesirable
      side-effects.
\end{enumerate}
Because of the complicated rules of the APA manual, it is not always possible
to achieve all kinds of compatibility with all other packages. Later in this
section, I will discuss compatibility issues with other packages with which
known incompatibility problems exist or have existed. A first example of this,
the \pkg{babel} package, has already been mentioned above in
section~\ref{sec:compat-babel}.

Point~\ref{compat1} is the most important. If you have to write a completely
different \fname{.bib} file for different citation styles, there does not seem
to be an advantage in using \BibTeX{}. You might as well write the reference
list explicitly in \LaTeX{} then. However, not even this point can be achieved
completely, although the vast majority of the items will be the same for
different styles. But there remain a number of choices that are style-specific
and that lead to differences in the \fname{.bib} file. Examples from the APA
manual are:
\begin{itemize}
\item If a referenced book is volume~III according to its title
      page, this should be referenced as Vol.~3 according to the
      APA manual rules, but that may not be the case with other
      styles;
\item Journal names are abbreviated by some styles, but not
      according to the APA manual rules;
\item Publisher names should be abbreviated according to the
      APA manual, e.g., ``John Wiley \& Sons'' becomes ``Wiley''.
      Other styles do not do this;
\item The issue number of a journal should only be given if
      the journal starts each issue with page~1, not if page
      numbers of different issues in the same volume are consecutive.
      Other styles may require the issue number in all cases;
\item Page ranges are given as ``341--351'' in APA style, whereas
      other styles use ``341--51'';
\item The APA rules require a capital after a colon (`\verb+:+'),
      so that subtitles start with a capital as well. Other styles
      may not do this;
\item The APA has specific rules about the address: For a U.S. city,
      the state (or territory) should be given as a 2-letter code
      from the U.S. Postal Services, and the province and country
      should be given if it is a non-U.S. city, unless it is one of
      the 17 (U.S. and non-U.S.) cities that are ``well-known for
      publishing''. Then, only the city should be mentioned. There
      are similar exceptions if the name of the state (and sometimes
      even city) is already mentioned in the university name and
      the university acts as publisher;
\item Yearbooks like \emph{Annual Review of Psychology} should be treated
      as journals according to the APA rules, whereas other styles
      treat these as books;
\item The additional fields (such as \fieldname{translator} and the
      \fieldname{original*}-fields) that are used by \pkg{apacite} but are
      not defined in other \BibTeX{} styles are of course
      \pkg{apacite}-specific.
\end{itemize}
Some of these problems can be handled relatively elegantly. For example,
the user could use a command like ``\verb+\RomanVol{3}+'' in the
\fieldname{volume} field. Then, the user could define a counter,
\verb+RomanVolcounter+, say, in \LaTeX{} and then define
\cmd{\RomanVol} (similar to \cmd{\BCnt} discussed earlier) as
\begin{verbatim}
\newcommand{\RomanVol}[1]{%
  \setcounter{RomanVolcounter}{#1}\theRomanVolcounter}
\end{verbatim}
where \cmd{\theRomanVolcounter} is defined as \verb+\arabic{RomanVolcounter}+
if \pkg{apacite} is loaded and as \verb+\Roman{RomanVolcounter}+ if another
package is loaded that desires this. The example with style-specific
\fname{.bib} files that contain the definitions of journal-strings was already
given in section~\ref{sec:bib} above. However, many users will not go through
all this trouble, and not all problems can be solved in this way. So we have
to realize that not all entries in the \fname{.bib} file will be suitable for
all citation styles. On the other hand, most citation styles and journals are
not as critical as APA journals and many styles in psychology and other social
sciences (including economics) are very similar, so a \fname{.bib} file that
is tailor-made for \pkg{apacite} is likely to be suitable enough for the
styles of most relevant alternative journals.

Point~\ref{compat2} above is currently not fulfilled. The \pkg{apacite}
citation commands are directly based on those of its immediate predecessor,
\pkg{theapa}. But the use of `\verb+<+' and `\verb+>+' for prefixes is not
used by other packages. The influential \pkg{natbib} package uses \cmd{\citep}
and \cmd{\citet} instead of \cmd{\cite} and \cmd{\citeA}, and uses many more
alternative commands, see section~\ref{sec:compat-natbib} below. The
\pkg{chicago} package uses \cmd{\citeA} instead of \cmd{\citeauthor}, many
``numerical'' citation styles only recognize \cmd{\cite}, and so forth. As
discussed below, I now consider \pkg{natbib} as the standard with which other
packages should comply. Therefore, future versions of \pkg{apacite} will
support the \pkg{natbib} citation commands, but the current version does not
do that.

I think that points~\ref{compat3} and~\ref{compat4} are not that important. It
seems to me that there is not much wrong with defining a style through a
combination of a \fname{.sty} and \fname{.bst} file, each requiring the other
to work. However, given the possibilities and philosophy of the \pkg{natbib}
package, it would be desirable that \fname{natbib.sty} can be used with
\fname{apacite.bst}. As discussed below, this works to some extent, although
it requires \fname{apacite.sty} to be loaded as well, before
\fname{natbib.sty}.

Point~\ref{compat5} is very important, especially with other packages that are
complementary to \pkg{apacite} in some sense. It would be very annoying having
to choose between using \pkg{babel} and \pkg{apacite}, or between
\pkg{hyperref} and \pkg{apacite}, etc. These packages serve totally different
purposes, cannot usefully be compared, and are more valuable when they can be
used jointly. The rest of this section addresses this issue for a number of
packages with which compatibility problems are known to exist or have existed.

%%%%%%%%%%%%%%%%%%%%%%%%%%%%%%%%%%%%%%%%%%%%%%%%%%%%%%%%%%%%%%%%%%%%%
\subsection{Compatibility of \pkg{apacite} and \pkg{natbib}}
\label{sec:compat-natbib}
%%%%%%%%%%%%%%%%%%%%%%%%%%%%%%%%%%%%%%%%%%%%%%%%%%%%%%%%%%%%%%%%%%%%%
The \pkg{natbib} package is a general purpose citation package that is
intended to work with a broad range of \BibTeX{} (and non-\BibTeX{}) styles
that generate the bibliography. The \pkg{natbib} package is quite advanced and
can be used to switch easily between completely different citation styles.
Furthermore, the apparent popularity of \pkg{natbib} has inspired writers of
packages that would otherwise be incompatible with \pkg{natbib} to write code
to resolve these incompatibilities. I will mention some of these packages
below.

I consider \pkg{natbib} as the \emph{de facto} standard with which
other packages should be compatible. Unfortunately, \pkg{apacite} is still
not entirely compatible with \pkg{natbib}. The main incompatibility is that
different citation commands are used, so that it is not possible to use the
same document with \pkg{natbib} or \pkg{apacite}.

\pkg{apacite}, building on its predecessors, uses \cmd{\cite} for
parenthetical citations and \cmd{\citeA} for in-text citations, whereas
\pkg{natbib} uses \cmd{\citep} and \cmd{\citet} for these purposes. Moreover,
text that should precede the citation is entered between \verb+<+ and \verb+>+
marks and text that should follow the citation is entered between square
brackets (\verb+[+ and \verb+]+, i.e., a standard \LaTeX{} optional argument)
in \pkg{apacite},  whereas \pkg{natbib} uses \emph{two} optional arguments
between brackets. If there is one optional argument, \pkg{natbib} interprets
this as text following the citation (just like \pkg{apacite}). If there is
text that should precede the citation, but not text that should follow the
citation, two optional arguments must be used in \pkg{natbib}, the second
being empty.

Furthermore, both packages have defined some alternative citation commands,
such as \cmd{\citeNP} (\pkg{apacite}) and \cmd{\citealp} (\pkg{natbib}), some
of which serve the same purpose and could therefore be mapped onto each other
easily, but some others have no equivalent in the other package.

A partial conversion table, listing the main \pkg{apacite} commands and their
(closest) \pkg{natbib} equivalent is given in Table~\ref{tab:APAnatConv}. The
APA requirement and \pkg{apacite} behavior of listing the full author list for
the first citation (if there are at most 5 authors) and the short author list
in subsequent citations is accomplished by loading \pkg{natbib} with the
\pkgoption{longnamesfirst} option. This is assumed in the table. An example of
better APA-like behavior of \pkg{natbib} is that, by using the
\pkgoption{sort} option, \pkg{natbib} is able to sort the references within
the same citation command, as required by the APA manual, whereas
\pkg{apacite} does not have this option yet.

%%%%%%%%%%%%%%%%%%%%%%%%%%%%%%%%%%%%%%%%%%%%%%%%%%%%%%%%%%%%%%%%%%%%%
\begin{table}[htp]
\caption{Partial conversion table of the main \pkg{apacite}
         citation commands and their (closest) \pkg{natbib} equivalents.}
\label{tab:APAnatConv}
\begin{tabular}{lcl}
\hline
\pkg{apacite}          & $\mbox{}\qquad\qquad\mbox{}$ & \pkg{natbib}       \\
\cline{1-1}\cline{3-3}
\cmd{\cite}            &                              & \cmd{\citep}       \\
\cmd{\citeA}           &                              & \cmd{\citet}       \\
\cmd{\citeNP}          &                              & \cmd{\citealp}     \\
\cmd{\citeauthor}      &                              & \cmd{\citeauthor}  \\
\\
\cmd{\fullcite}        &                              & \cmd{\citep*}      \\
\cmd{\fullciteA}       &                              & \cmd{\citet*}      \\
\cmd{\fullciteNP}      &                              & \cmd{\citealp*}    \\
\cmd{\fullciteauthor}  &                              & \cmd{\citeauthor*} \\
\\
\cmd{\shortcite}       &     & \cmdtwo{shortcites}{keys}\cmd{\citep}       \\
\cmd{\shortciteA}      &     & \cmdtwo{shortcites}{keys}\cmd{\citet}       \\
\cmd{\shortciteNP}     &     & \cmdtwo{shortcites}{keys}\cmd{\citealp}     \\
\cmd{\shortciteauthor} &     & \cmdtwo{shortcites}{keys}\cmd{\citeauthor}  \\
\\
\cmd{\citeyear}        &                              & \cmd{\citeyearpar} \\
\cmd{\citeyearNP}      &                              & \cmd{\citeyear}    \\
\cmd{\nocite}          &                              & \cmd{\nocite}      \\
\hline
\end{tabular}
\end{table}
%%%%%%%%%%%%%%%%%%%%%%%%%%%%%%%%%%%%%%%%%%%%%%%%%%%%%%%%%%%%%%%%%%%%%

Many labeling and punctuation commands are also different but similar between
\pkg{natbib} and \pkg{apacite}. I haven't studied the \pkg{natbib} commands
in detail, but will do so in the future. It is my intention to add
\pkg{natbib}-compatible commands to apacite in the future, so that the same
document can be processed either with \pkg{natbib} or with \pkg{apacite}.

With previous versions of \pkg{apacite}, there used to be some erratic
behavior (error messages and wrong formatting) when the \pkg{natbib}
\LaTeX{} package (\fname{natbib.sty}) was used with the \pkg{apacite}
\BibTeX{} style (\fname{apacite.bst} or \fname{apacitex.bst}). These
problems have now been almost entirely resolved (I believe).

When you want to use \pkg{natbib} for citation and \pkg{apacite} for the
reference list, you still need to load the \LaTeX{} package \fname{apacite.sty}
(with \cmd{\usepackage}), because of the commands that are included in the
\BibTeX{} output (as defined in section~\ref{sec:custom} above). The
\fname{apacite.sty} file must be loaded \emph{before} \fname{natbib.sty},
otherwise you will get lots of error messages. I may try to fix that for
a future version of \pkg{apacite}.

Using \pkg{natbib} for the citations has some advantages over using
\pkg{apacite} for the citations. Apart from the abovementioned sorting
of citations within a single citation command, these are mainly
\pkg{natbib}'s better compatibility with other important packages,
such as \pkg{hyperref}.

However, \pkg{natbib} does not fully comply with the APA rules.
Known incompatibilities between \pkg{natbib} and the APA rules are: (1)
Multiple adjacent citations with the same author and year are formatted
as ``Johnson, 1991a,b'' by \pkg{natbib}, whereas the APA manual requires
this to be ``Johnson, 1991a, 1991b''; (2) The APA manual requires ``and''
between authors to be ``and'' in text and ``\&'' in parenthetical citation.

I could change the \fname{apacite.bst} and \fname{apacitex.bst} \BibTeX{}
style files so that \pkg{natbib} would be ``fooled'' and the first
problem would be resolved. However, this has the drawback of imposing APA
style citations when you request \pkg{natbib} style citations, and you
may actually want the behavior mentioned above. I am only able to provide
one of these possibilities. It would be better to add an option to
\pkg{natbib} with which the desired behavior can be chosen. Thus, you
could try to put pressure on the author of \pkg{natbib} to achieve this.
Alternatively, I might attempt to provide an option to \pkg{apacite}
in the future that controls this behavior of \pkg{natbib}.

The second incompatibility between \pkg{natbib} and the APA rules mentioned
above occurs because \pkg{apacite} uses the re-definable \cmd{\BBA}
command for this usage of ``and''. With \pkg{natbib}, it is not possible
that \cmd{\BBA} ``knows'' whether it is parenthetical or in-text, so you can
choose only one of these, or you must redefine \cmd{\BBA} explicitly before
each citation.

%%%%%%%%%%%%%%%%%%%%%%%%%%%%%%%%%%%%%%%%%%%%%%%%%%%%%%%%%%%%%%%%%%%%%%%%%%%%
\subsection{Compatibility of \pkg{apacite} and \pkg{hyperref}}
\label{sec:compat-hyper}
%%%%%%%%%%%%%%%%%%%%%%%%%%%%%%%%%%%%%%%%%%%%%%%%%%%%%%%%%%%%%%%%%%%%%%%%%%%%
The \pkg{hyperref} package turns (cross-) references into hypertext links.
This can be used in conjunction with a program such as \latextohtml{} to
compose \fname{.html} files with clickable links, to internet pages or within
the same document, or it can be used to create a \fname{.pdf} document with
clickable crossreferences. Evidently, citations are also references.
Therefore, the \pkg{hyperref} package transforms these into hypertext links as
well, and needs to redefine the citation commands and reference list commands
in order to do so. These redefinitions are incompatible with redefinitions of
the citation commands by \pkg{apacite}.

Similar problems occur between \pkg{hyperref} and \pkg{natbib}, and some code
is included in both packages, which jointly resolves these incompatibilities.
In order to make \pkg{apacite} compatible with \pkg{hyperref}, similar code
should be developed and included in \pkg{apacite}.

An attempt to achieve this can be tried through the \pkgoption{hyper} option
of \pkg{apacite}. This activates some code provided by Ross Moore (thanks,
Ross) that makes \pkg{apacite} and \pkg{hyperref} work together to some
extent. However, different things go wrong depending on the order in which the
two packages are loaded.

If the \pkg{hyperref} package is loaded \emph{after} \pkg{apacite}, different
citations with the same author(s) and year do not work well: \pkg{hyperref}
suppresses the ``a'', ``b'', etc., so ``(Johnson, 1991a)'' becomes ``(Johnson,
1991)'', which is undesirable. If the \pkg{hyperref} package is loaded
\emph{before} \pkg{apacite}, this problem does not occur, but the citations in
the text do not link to the reference list anymore.

The following problems are independent of the order of loading:
(1) Citations of the form ``(Author, year1, year2)'', i.e., with multiple
    references to the same author(s) in the same citation command are not
    recognized as such and are thus formatted as ``(Author, year1; Author,
    year2)'';
(2) If the \cmd{\url} command from the \pkg{url} package is used for internet
    addresses (e.g., for retrieval information in the reference list), line
    breaks are not inserted within the address, even if it's way too long for
    the line, and ``\verb+#+'' characters in the internet address are not
    processed well, see the discussion of the \pkg{backref} package below.

Because of these known problems, the \pkgoption{hyper} option is not executed
by default (the \pkgoption{nohyper} option is), but if you find the advantages
more important than the disadvantages, you can request this option. Note that
you should load the \pkg{hyperref} package yourself, this is not done by
\pkg{apacite}. Thus, you can choose the order in which to load the packages.

%%%%%%%%%%%%%%%%%%%%%%%%%%%%%%%%%%%%%%%%%%%%%%%%%%%%%%%%%%%%%%%%%%%%%%%%%%%%
\subsection{Compatibility of \pkg{apacite} and \pkg{backref}}
\label{sec:compat-backref}
%%%%%%%%%%%%%%%%%%%%%%%%%%%%%%%%%%%%%%%%%%%%%%%%%%%%%%%%%%%%%%%%%%%%%%%%%%%%
The \pkg{backref} package adds to each entry in the bibliography a list
of pages (or sections) on which it is referenced. Hence, it serves as
an alternative to an author index, in which not the entries themselves
are backreferenced, but the authors of these entries, see
section~\ref{sec:autindex}. Note that \pkg{natbib} offers yet another
possibility, viz.\ the citations in the standard index, or a separate
citation index, see the \pkg{natbib} documentation. In my opinion,
the \pkg{backref} way of indexing the page numbers of the citations
is more logical. Moreover, it is more condensed as well.

Obviously, the \pkg{backref} package has to change the citation commands
and reference list to be able to do its backreferencing.
Because \pkg{apacite} redefines these as well, they are conflicting
to some degree. A similar compatibility problem between \pkg{backref}
and \pkg{natbib} was noticed by the author of \pkg{backref}. Therefore,
\pkg{backref} contains some code to make it compatible with \pkg{natbib}.
This code has been adapted for \pkg{apacite} and incorporated in
\pkg{apacite}. Thus, \pkg{backref} as a standalone package has been made
compatible with \pkg{apacite}. The drawback of this solution is that if
changes are made in the \pkg{backref} package, this may result in new
incompatibilities with \pkg{apacite} unless \pkg{apacite} is changed as
well. Whether, and if so, when, this will occur will be a question for
the future.

Ex.~74 of the APA manual references an internet address that contains a
``\verb+#+''. This leads to an error with \pkg{backref} and when entered
through, \LaTeX{} inserts ``\verb+##+'' in the output instead of ``\verb+#+''.
This will not occur often in practice, but unfortunately occurs in the
mentioned example, which might give the impression that it is an \pkg{apacite}
bug, but it is not an \pkg{apacite} problem or even a compatibility problem,
but a \pkg{backref} problem. I had to use some nasty code in the current
document (\fname{apacite.tex}) to have the bibliography entry of this
reference formatted correctly when \pkg{backref} is loaded. When you
encounter this problem, you can use this code.

Apart from being a standalone package, \pkg{backref} can also be
combined with \pkg{hyperref}. It is then not loaded separately,
but through the \pkgoption{backref} or \pkgoption{pagebackref}
option of the \pkg{hyperref} package. In this case, the backreferences
become hypertext links. See the discussion of the \pkg{hyperref}
package above.

%%%%%%%%%%%%%%%%%%%%%%%%%%%%%%%%%%%%%%%%%%%%%%%%%%%%%%%%%%%%%%%%%%%%%%%%%%%%
\subsection{Multiple bibliographies}
\label{sec:compat-chapterbib}
%%%%%%%%%%%%%%%%%%%%%%%%%%%%%%%%%%%%%%%%%%%%%%%%%%%%%%%%%%%%%%%%%%%%%%%%%%%%
The \pkg{chapterbib} package allows separate bibliographies for separate
chapters of a book, or, technically, separate \cmd{\include}d files. To work
properly, it needs some small changes to the citation package, which are
clearly described in the \pkg{chapterbib} documentation. These changes have
been incorporated in \pkg{apacite}, and therefore \pkg{apacite} is compatible
with \pkg{chapterbib}.

It is generally advisable not to use the \pkgoption{sectionbib} option of
\pkg{chapterbib}, because this redefines \cmd{\thebibliography}, which
interferes with the redefinition by \pkg{apacite}. You can use the
\pkgoption{sectionbib} option of \pkg{apacite} to accomplish that the
bibliography or bibliographies are sections, rather than chapters.

Moreover, \pkg{apacite} is configured such that it takes a section as default
if it is in the main matter (signified by \cmd{\mainmatter}) of the document.
In the back matter (\cmd{\backmatter}) the bibliography is put in a chapter.
This is also the case if \cmd{\chapter} is defined but \cmd{\mainmatter} not.
(Is this possible? Perhaps for reports?) Consequently, the
\pkgoption{sectionbib} option is only useful in specific circumstances. The
\pkgoption{nosectionbib} forces the bibliography to be a chapter, provided
\cmd{\chapter} is defined. (I figured a \pkgoption{chapterbib} option would be
a bit confusing, so I stuck with \pkgoption{nosectionbib}.)

The \pkg{bibunits} package is an alternative to \pkg{chapterbib}. It allows a
greater flexibility in choosing the scope of a bibliography (chapter, section,
or arbitrarily defined part of the document). I haven't tested it thoroughly,
but it seems like there are no compatibility problems between \pkg{apacite}
and \pkg{bibunits}.

%%%%%%%%%%%%%%%%%%%%%%%%%%%%%%%%%%%%%%%%%%%%%%%%%%%%%%%%%%%%%%%%%%%%%%%%%%%%
\subsection{Programs for conversion to html, rtf, etc.}
%%%%%%%%%%%%%%%%%%%%%%%%%%%%%%%%%%%%%%%%%%%%%%%%%%%%%%%%%%%%%%%%%%%%%%%%%%%%
There are several programs that convert \LaTeX{} files to other types
of files, most notably html and rtf. This may, for example, be useful if you
want to publish your document on the internet or if you are requested to send
a ``Word'' file to a journal. I have received reports that \latextohtml{}
and \LaTeXrtf{} do not handle documents that use \pkg{apacite} well.
Therefore, I have studied these programs.

\latextohtml{} is a \Perl{} program that converts a \LaTeX{} input
file into one or more \fname{.html} files, so that a document that is prepared
with \LaTeX{} can be used as a (user-friendly) internet document. For papers,
lecture notes, and transparencies of classes, I always convert the
\fname{.dvi} file output by \LaTeX{} to a \fname{.pdf} file (through the
\pkg{dvipdfm} program), which can be read by most people, e.g., using the
\pkg{Adobe Reader} program or browser plug-in. I find \fname{.pdf} files
more convenient, because I can download a whole paper or presentation
with a single click and then print it and read it when and where I desire,
whereas typical examples of conversions to \fname{.html} use many
\fname{.html} files, which can only be conveniently read from the computer
screen when and where you're connected to the internet. Nevertheless, I can
imagine that someone would like to convert a document to \fname{.html}, e.g.,
because it loads faster and reads easier from the screen, because for some
documents most people will have to use only a small part and can easier browse
or search through it, or just because you want to make a home page and do not
want to learn the html-language.

Similarly, \LaTeXrtf{} is a standalone program that converts a \LaTeX{}
document to \fname{.rtf} format, which can be read by most word processors.
This may be convenient if you (a \LaTeX{} user) want to share a document with
someone who does not work with \LaTeX{}, or as mentioned above, if you are
required to send a ``Word'' file to a journal.

Both \latextohtml{} and \LaTeXrtf{} implement \LaTeX{} macros etc.\ in
another programming language (\latextohtml{}: \Perl{}; \LaTeXrtf{}: C) and
process the \fname{.tex} files directly. This design implies that the
conversion program must know the definitions of all \LaTeX{} macros that
are used. A drawback of this design is that essentially the same code has
to be written in different programming languages (\LaTeX{} macros for the
\fname{.sty} file, \Perl{} for \latextohtml{}, and C for \LaTeXrtf{}), but
typically with some adaptations specific for the application at hand.

With \LaTeXrtf{}, the \LaTeX{} macros that it is able to process are included
in the C source code that is compiled. Consequently, no user additions are
possible in principle. However, the source code is freely available, so if you
define a \LaTeX{} macro, you could adapt \LaTeXrtf{}'s source code and
recompile \LaTeXrtf{}. It is obvious that this is very inconvenient,
especially if \LaTeX{} macros are defined differently for different styles or
if they are defined differently in different parts of the same document.

Version~1.9.15 and higher of \LaTeXrtf{} contains some support for
\pkg{apacite}, but lags a few versions of \pkg{apacite} behind and thus
doesn't work with recent versions of \pkg{apacite}. Moreover, even if this
would be fixed, e.g., because I (or you) send patches to the \LaTeXrtf{}
authors, this wouldn't do justice to the flexibility of \pkg{apacite}, i.e.,
its customization options.

In contrast with this, \latextohtml{} allows the possibility to supply
external \Perl{} scripts. If \latextohtml{} encounters a
\cmdtwo{usepackage}{package} command in the document, it attempts to read the
corresponding \opt{package}\fname{.perl} \Perl{} script. \latextohtml{} is
shipped with many such \Perl{} scripts. There is not one for \pkg{apacite},
but there is one for \pkg{natbib}, which could possibly be adapted for
\pkg{apacite}. Alternatively, you could use the \pkg{natbib} citation commands
and use the \pkg{natbib} package, which should work. This will become more
straightforward when, in a future version, \pkg{apacite} also supports the
\pkg{natbib} citation commands. Then the \fname{natbib.perl} script may
possibly be copied to \fname{apacite.perl} without having to adapt much.
However, I have never studied \Perl{}, and I do not particularly like the
idea of having to learn that language for this purpose, so it is not very
likely that in the near future \latextohtml{} will work well with
\pkg{apacite}.

Note, however, that the design with the \Perl{} scripts has the advantage
that, if you customize the usage of \pkg{apacite} by writing your own
\fname{.sty} file that renews the definitions of some macros, you can also
supply a corresponding \Perl{} script, so that your customizations are
processed.

An alternative to both \latextohtml{} and \LaTeXrtf{} is \TeXht{}. This is
a collection of programs and style files that convert \LaTeX{} to
various formats, such as html, xml, and OpenOffice format. If you have
OpenOffice, the latter can be used to convert to Word (or rtf) format.
The principle of \TeXht{} is to let \LaTeX{} do most of the formatting and
processing of commands, and do the conversion at low-level \TeX{} commands.
In this way, all newly defined commands and user-defined changes are
automatically supported. This even carries over to redefinitions of macros
in the middle of a document. However, for some specific purposes, there are
some additional things that must be arranged. Therefore, \TeXht{} contains a
large list of \fname{.4ht} files that are necessary to support various
\LaTeX{} packages. One of these is the \fname{apacite.4ht} file. The main
purpose of this file is to make citations into hyperlinks to the corresponding
entries in the reference list. Currently, this only works partially, but
the citation text itself is formatted correctly, so this is only a
relatively minor inconvenience. But, of course, it would be better if this
would work perfectly. I suspect that this issue is related to the problems
with the compatibility of \pkg{apacite} with \pkg{hyperref}, and can be
solved in the same way. I intend to study this issue in the near future and
hope to solve this for a future version (although, technically, it is a
\TeXht{} problem and not an \pkg{apacite} problem).

The choice of which converter to use will typically not (only) depend on how
good it works in conjunction with \pkg{apacite}. Furthermore, apart from the
converters mentioned above, there are several additional converters that I
haven't studied at all. I will only study them if I receive reports about
incompatibility and even then it is far from certain that I will be willing or
able to fix the problems. It would be very demanding if I would have to define
the \pkg{apacite} commands in a large number of programming languages for a
large number of different converters. Because I prefer the design of \TeXht{}
and because this converter seems to work best at the moment, I will give a
bit more attention to this one.

%%%%%%%%%%%%%%%%%%%%%%%%%%%%%%%%%%%%%%%%%%%%%%%%%%%%%%%%%%%%%%%%%%%%%%%%%%%%
\section{Generating an author index}
\label{sec:autindex}
%%%%%%%%%%%%%%%%%%%%%%%%%%%%%%%%%%%%%%%%%%%%%%%%%%%%%%%%%%%%%%%%%%%%%%%%%%%%
The \pkg{apacite} package contains an option to (almost) automatically
generate an author index. This is done by using the \fname{apacitex.bst}
\BibTeX{} style file instead of \fname{apacite.bst} and loading the
\fname{apacite.sty} \LaTeX{} package with one of the \pkgoption{index},
\pkgoption{stdindex}, \pkgoption{tocindex}, or \pkgoption{emindex} options. In
that case, the \pkg{apacite} package automatically loads the \pkg{index}
package that supports multiple indexes, so that you can have a subject index
as well as an author index. Consequently, the \pkg{index} package must be
available in a directory that is read by \LaTeX{}. If a subject index is
desired, it must be defined by the user, it is not defined by \pkg{apacite}.
See the manual of the \pkg{index} package on how to do this. A default author
index \emph{is} defined by \pkg{apacite}. The author index is then requested
by putting
\begin{verbatim}
\printindex[autx]
\end{verbatim}
at the point in the \LaTeX{} document where the index is supposed to appear.
To include the author index in the processed document, the following sequence
must be followed: First, \LaTeX{}, which puts citation entries in the
\fname{.aux} file. Second, \BibTeX{}, which generates the bibliography in the
\fname{.bbl} file. Third, \LaTeX{} (at least) twice, which puts index entries
in a \fname{.adx} file, Fourth, \MakeIndex{}, which uses the \fname{.adx} file
as input and puts the formatted index in a \fname{.and} file, which can be
processed by \LaTeX{}. So, if the main document is \fname{filename.tex}, then
\MakeIndex{} must be called as follows:
\begin{verbatim}
makeindex -o filename.and filename.adx
\end{verbatim}
although the command with which the \MakeIndex{} program must be called may on
some systems be slightly different, e.g., \fname{makeinde} or
\fname{makeindx}. Furthermore, several additional options may be chosen on the
\MakeIndex{} command line. Fifth, run \LaTeX{} again, which (finally) includes
the formatted author index in the \fname{.dvi} file.

The \pkg{apacite} package provides some options to change the appearance of
the index somewhat. With the \pkgoption{index} option, the indexing facility
is turned on, but the \cmd{theindex} environment is not explicitly (re)defined
by \pkg{apacite}. This means that this environment is defined just like in the
\pkg{index} package, unless it is overridden by the \pkgoption{tocindex} or
\pkgoption{emindex} options or redefined by the user or another package that
is loaded.

The \pkgoption{stdindex} option explicitly uses the definition of the
\cmd{theindex} environment that is defined in the \pkg{index} package
[1995/09/28 v4.1beta Improved index support (dmj)].
With this definition, the index does not appear in the table of contents. With
the \pkgoption{tocindex} option, this definition is augmented with a table of
contents entry. Finally, with the \pkgoption{emindex} option, there are some
alternative definitions. It writes a table of contents entry as well, but the
index itself is now set in two columns using the \pkg{multicol} package
instead of the \cmd{\twocolumn} command, the text of the index is set in small
type, and the page head is not put in uppercase.

When a table of contents entry is desired, another additional \LaTeX{} run is
required to obtain the table of contents with the author index included. In
exceptional cases, such as the current document, the extra line in de table of
contents may cause the text to be moved as well, so that it may be necessary
to run \MakeIndex{} a second time, followed by yet another \LaTeX{} run.

The definitions of the index discussed apply to all indexes in the document,
so to, e.g., the subject index as well as the author index. Finally, the
formatting of the index itself can be changed by writing a \fname{.ist} file
containing the preferred options. \MakeIndex{} should then be called with the
filename of this index style file after a ``\verb+-s+'' option on the command
line, so, e.g.,
\begin{verbatim}
makeindex -s mystyle.ist -o filename.and filename.adx
\end{verbatim}
The \MakeIndex{} documentation discusses the possible options that can be put
into the \fname{.ist} file.

If you process the file \fname{apacite.tex} in the way discussed above (look
at the instructions in the \fname{README} file), it becomes clear that the
author index is almost as desired, but there are a few exceptions. First, I
can imagine that you would not want corporate authors, such as ``American
Psychological Association'', in the author index, but only human authors. I
haven't (yet) figured out a way to do this (semi-) automatically, so this has
to be done manually by editing the \fname{.bbl} file, see below.

Second, serious problems occur with cases such as ``Bender, J.~{(Director)}.''
and ``Bulatao, E.~{(with Winford, C. A.)}.''. Obviously, the description
``{(Director)}.'' should not be in the index and Winford should not be listed
as part of Bulatao, but as a separate author. These problems are caused by the
``tricks'' I used to make \BibTeX{} format them correctly in the reference
list, see section~\ref{sec:xmpl} below. They can also be manually solved by
editing the \fname{.bbl} file as discussed below.

The author index does not contain entries for the authors Shocked and
Goodenough (as well as several corporate authors) as cited in text, although
it does for the reference list part, because the \fieldname{key} field was
used for the citations. If you use the \fieldname{key} (and
\fieldname{firstkey}) field and you want the authors in the author index as
well, you have to include the index command \verb+\protect\AX+ explicitly. For
example, the ``tricked'' \fieldname{key} field for Michelle Shocked is
\begin{verbatim}
  key = {{\protect\bibsong{Shocked}{1992}{Over the
         Waterfall}{track~5}}},
\end{verbatim}
and if we change this into
\begin{verbatim}
  key = {{\protect\bibsong{Shocked\protect\AX{shocked  m
         @Shocked, M.}}{1992}{Over the Waterfall}{track~5}}},
\end{verbatim}
then citations to her are included in the author index. The \cmd{\AX} command
will be discussed below.

Summarizing, the author indexing part works very well for most commonly
encountered cases. However, it does not work (entirely) correctly if special
measures need to be taken to get them formatted correctly in \BibTeX{}
(citations, reference list).

An easy solution to incorrect formatting and/or sorting in the author index is
to edit the \fname{.bbl} file manually. This should be done at a time when
\BibTeX{} will not have to be run again. Otherwise, the changes would be
overwritten by the next \BibTeX{} run. This editing of the \fname{.bbl} file
is not in the \TeX-spirit, but in my experience --- I used it for two books
with lots of references \cite{meijer1998,WaMe00}, with a previous version of
\pkg{apacite} that caused many more problematic cases --- this takes very
little time, usually only minutes (compare that with the time spent on writing
a book, or with the time that would be necessary to manually make an author
index).

The connection between the \fname{.bbl} file and the author index is through
\cmd{\AX} commands that are entered by \fname{apacitex.bst} in the
\fname{.bbl} file. These are responsible for the author indexing facilities.
By looking at the contents of the \fname{.bbl} file, some changes that lead to
correct formatting are immediately obvious. For example, the entry for Bender
is generated by the following lines:
\begin{verbatim}
\protect\AX{bender  jdirector
@Bender, J.~{\bibliteral{(Director)}}.}}{%
\end{verbatim}
It is immediately clear that this can be changed to
\begin{verbatim}
\protect\AX{bender  j
@Bender, J.}}{%
\end{verbatim}
Then, this author will be placed and formatted correctly in the author index.
Note that \pkg{apacite} uses the \opt{key}\verb+@+\opt{visual} form of
indexing. This form is used in \MakeIndex{} to distinguish between the actual
representation in the index and the place where it should occur. The \opt{key}
is used by \MakeIndex{} to sort the entry, but the entry actually appearing in
the index is \opt{visual}. In the \opt{key} part as formatted by
\pkg{apacite}, all names are in lower case letters and all accents and
punctuation are removed. Furthermore, \pkg{apacite} inserts \emph{two} spaces
between surname and initials, and to \MakeIndex{}, two spaces are different
from one space. Spaces between surnames are removed, as well as spaces between
initials.

The correct author index could also be accomplished by editing the final
\fname{.and} file, which contains the final formatted author index. However,
it is more convenient to edit the \fname{.bbl} file, which is usually
``final'' in a much earlier stage.

%%%%%%%%%%%%%%%%%%%%%%%%%%%%%%%%%%%%%%%%%%%%%%%%%%%%%%%%%%%%%%%%%%%%%%%
\section{Known problems, things to be done, etc.}
\label{sec:todo}
%%%%%%%%%%%%%%%%%%%%%%%%%%%%%%%%%%%%%%%%%%%%%%%%%%%%%%%%%%%%%%%%%%%%%%%
With any kind of software, there is usually a list with known problems (bugs)
and desirable future work (to-do), and \pkg{apacite} is no exception. These
subjects will be discussed in this section.

A class of problems was already discussed in section~\ref{sec:compat} above.
This concerns the incompatibility of the \fname{.bib} file with other citation
styles, through \pkg{apacite}-specific fields, APA-specific contents of
fields, or through special tricks. These lead to correctly formatted
references in APA style, but may cause problems with using the \fname{.bib}
file with other styles (in case APA journals reject your article~\ldots). As
discussed there, however, this problem is unavoidable and presumably not a
very big problem to most users.

An incomplete list of other known problems and things that I would
like to do with \pkg{apacite} is:
\begin{itemize}
\item Sort entries within a single citation command. The \pkg{natbib}
      package already offers this option;
\item Swap (optionally?) the order of closing quotes and various other
      punctuation marks (\verb+.,;!?+) in a citation, for articles where
      there is no named author and the title (between quotes) takes the
      role of the author;
\item If a work is authored by someone, ``with'' someone else: This
      currently must be tricked in a nonelegant way. Perhaps I can
      think of a better solution.
\item Add the \fieldname{address} field for \entryname{lecture} entries.
      In the current version, the address is part of the description of
      the meeting in the \fieldname{howpublished} field, but it makes
      sense to disentangle these;
\item Give explicit support for citation to the Bible and other
      ``classic'' works and the \emph{DSM};
\item Give explicit support for music recordings in \entryname{incollection}
      entries;
\item Study referencing legal materials (Appendix~D of the APA manual);
\item Define \BibTeX{} macros or strings for commonly encountered
      journals (at least the APA journals);
\item Support \pkg{natbib} citation commands (\cmd{\citet}, \cmd{\citep},
      etc.);
\item Improve and extend language support;
\item Make a \pkg{doc}/\pkg{docstrip} version, so that the whole suite
      consists of a \fname{.dtx} file containing all code, documentation,
      and test documents, and a \fname{.ins} file that extracts the various
      desired files from the \fname{.dtx} file.
\end{itemize}
Quite likely, other problems will come up when using the current version of
\pkg{apacite}, but at least I got it working on the APA manual examples, see
section~\ref{sec:xmpl} below. I have lots of ideas for improvements,
extensions, options, etc., but not much time to devote to it. Therefore, it is
difficult to give a good estimate of when I will release a new update.
However, in the meantime, if you have questions, remarks, suggestions, or bug
reports, you can e-mail them to me.

%%%%%%%%%%%%%%%%%%%%%%%%%%%%%%%%%%%%%%%%%%%%%%%%%%%%%%%%%%%%%%%%%%%%%%%
\section{Examples of the APA manual}
\label{sec:xmpl}
%%%%%%%%%%%%%%%%%%%%%%%%%%%%%%%%%%%%%%%%%%%%%%%%%%%%%%%%%%%%%%%%%%%%%%%
In this section, the \pkg{apacite} package is tested by citing the examples of
the APA manual (5th ed., pp.~207--281), and some additional references for
additional purposes.

The following list gives the examples from chapter~3 of the APA manual, with
section number. For these and all later examples, whenever I thought it
necessary or desirable, I added some comments. These are indicated by the \EM{}
symbol. Especially for later examples (from chap.~4), the comments frequently
pertain to the corresponding reference list entry.

\begin{itemize}
\item[3.94]  \citeA{3.94-1} compared reaction times\\
             In a recent study of reaction times \cite{3.94-1}\\
             In \citeyearNP{3.94-1} \citeauthor{3.94-1} compared
               reaction times\\
             In a recent study of reaction times, \citeA{3.94-1}
             described the method\ldots. \citeauthor{3.94-1} also found
\item[3.95]  \citeA{3.95-1} found\\
             \citeA{3.95-1} found\\
             \citeauthor{3.95-1} found\\
             \fullcite{3.95-2} and \fullcite{3.95-3}\\
             \citeA{3.95-2} and \citeA{3.95-3}\\
             \EM It seems that if ``et al.'' would refer to one additional
             author, then this author is named instead of the ``et al.''
             clause.\\
             \citeA{3.95-4} and \citeA{3.95-5}\\
             as \citeA{3.95-6} demonstrated\\
             \EM The comma after ``Nightlinger'' in the reference list
             does not look right, but is explicitly required by the APA.
             Of course, with people as authors,
             one almost always has initials (although I have seen
             an example of an econometrician who only had
             one name), so this problem is not likely to occur.
             With nicknames (``screen names'' as they are called in the
             APA manual below ex. 85) in messages to newsgroups, initials
             are typically missing, but then there is usually (always?)
             only one author, so the problem does not exist either.
             The problem can, however, come up when there are two authors,
             the first of which is a corporate author, or with artists,
             e.g., the song ``Into the Groove'' is written by
             Madonna and Stephen Bray. (But perhaps the APA would want
             Madonna to be referred to as ``Ciccone, M.'').\\
             as has been shown \cite{3.95-7}
\item[3.96]  \cite{3.96-1}\\
             \cite{3.96-1}\\
             \cite{3.96-2}\\
             \EM For this document, I defined the \cmd{\bibcorporate}
             macro, which indicates that the author is a corporate
             author. The way it is used now only has the effect that
             in the reference list, the name is treated as a whole,
             and not as a firstname-lastname combination. My idea is
             to think of a way to define this macro such that it is
             able to suppress inclusion of the corporate author in the
             author index, but I have not succeeded in this.
             Note, however, that this is not part of \pkg{apacite}, but
             part of the trick box of the user.\\
             \EM If the \fieldname{firstkey} field is different
             and the \fieldname{key} field is the same, then, analogous
             to the use of extra authors in 3.95 above, \pkg{apacite}
             uses the \fieldname{firstkey} field for all citations:\\
             \citeA{3.96-3}, \citeA{3.96-4}, \citeA{3.96-5}, \citeA{3.96-6};\\
             \citeA{3.96-3}, \citeA{3.96-4}, \citeA{3.96-5}, \citeA{3.96-6}.\\
             Although this is a constructed example, these organizations
             really exist. The ``Koninklijke Nederlandse Schaakbond''
             is the Royal Dutch Chess Association and the
             ``Koninklijke Nederlandse Schaatsbond'' is the Royal Dutch
             Skating Association. Both are abbreviated to KNSB and
             presumably, both publish an annual report each year.
\item[3.97]  on free care \cite{3.97-1}\\
             \EM The order of the
             closing quotes and the comma is different from the
             APA manual. I think this one is more logical, although the
             APA manual's is the conventional one, presumably for
             aesthetic reasons. I hope to implement this in a future
             version.\\
             the book \citeA{3.97-2}\\
             \cite{3.97-3}\\
             \EM If two articles both have no author and the title and
             the year are the same, a's and b's should be used:
             \citeA{3.97-4}, \citeA{3.97-5}, and \citeA{3.97-6}.
             The latter two entries are correctly sorted by month,
             through the way the month macros are defined in
             \fname{apacitex.bst}.
\item[3.98]  \citeA{3.98-1} and \citeA{3.98-2} also found\\
             \citeA{3.98-3} and \citeA{3.98-4} studied
\item[3.99]  Past research \cite{3.99-1,3.99-2}\\
             Past research \cite{3.99-3,3.99-4,3.99-5}\\
             Several studies
               \cite{3.99-6,3.99-7,3.99-8,3.99-9,3.99-10,3.99-11}\\
             Several studies \cite{3.99-12,3.99-13,3.99-14}\\
             \EM \pkg{apacite} does not (yet?) sort the cited
             references in the text (of course it does for the reference
             list), as required by the APA manual.\\
             (\citeNP{3.99-15}; see also \citeNP{3.99-16,3.99-17})\\
             \EM Note that I had to use \cmd{\citeNP} here.
\item[3.100] \cite{3.100-1}\\
             \EM This is not an example, but a rule in the text.
             The ``\bibnodate'' is handled by the \cmd{\bibnodate}
             macro. It is not clear whether it is really meant that the
             in-text form is \citeauthor{3.100-1}, \citeyearNP{3.100-1}
             or the more logically consistent \citeA{3.100-1}.
             The former can (currently) only be accomplished by using
             \cmd{\citeauthor} and \cmd{\citeyearNP} explicitly,
             whereas the latter is simply obtained by \cmd{\citeA}.\\
             \cite{3.100-2}\\
             \citeA{3.100-3}

             \EM For citations to the bible, I defined a \cmd{\biblecite}
             command and several obvious variations on it. For example:\\
             \cmd{\bibleciteA}, first cite:  \bibleciteA{1 Cor.\ 13:1}\\
             \cmd{\bibleciteA}, second cite: \bibleciteA{1 Cor.\ 13:1}\\
             \cmd{\biblecite}, first cite: \fullbiblecite{1 Cor.\ 13:1}\\
             \cmd{\biblecite}, second cite: \biblecite{1 Cor.\ 13:1}\\
             (\cmd{\bibleciteNP}, first cite:
              \fullbibleciteNP{1 Cor.\ 13:1} and some text surrounding it)\\
             (\cmd{\bibleciteNP}, second cite:
              \bibleciteNP{1 Cor.\ 13:1} and some text surrounding it)\\
             \EM Here, there probably should be a comma if it's the first
             cite and not a comma when it's a later cite. A difficult one.\\
             \EM Are there other ``classical works'' that should be treated
             like this? If so, which and how?
\item[3.101] \cite[p.~332]{3.101-1}\\
             \cite[chap.~3]{3.101-2}\\
             \EM If you like, you can use the \pkg{apacite}-defined
             abbreviation commands \verb+\BPG+ and \verb+\BCHAP+.\\
             \cite[\P~5]{3.101-3}\\
             \cite[Conclusion section, para.~1]{3.101-4}
\item[3.102] \EM Personal communication is not really citation and should
             be done manually:\\
             T.~K. Lutes (personal communication, April 18, 2001)\\
             (V.-G. Nguyen, personal communication, September 28, 1998)
\item[3.103] \cite<see Table~2 of>[for complete data]{3.103-1}
\end{itemize}

\noindent
Chapter~4 of the APA manual deals with the reference list
and only mentions some in-text citation issues in passing.
Therefore, the rest mainly requires a lot of moving back and forth
between the reference list and this part. I start with the
more ``general'' part, sections 4.01--4.15, presented similar to
the sections from chapter~3 as presented above.
\begin{itemize}
\item[4.04] \EM First, some examples mentioned in the text:\\
            \citeA{4.04-t1},
            \citeA{4.04-t2},
            \citeA{4.04-t3},
            \citeA{4.04-t4},
            \citeA{4.04-t5},
            \citeA{4.04-t6},
            \citeA{4.04-t7},
            \citeA{4.04-t8},
            \citeA{4.04-t9}\\
            \EM The APA manual states that prefixes must be treated
            according to the rules of the language of origin. These rules
            can differ a lot between languages (and countries) and it
            is impossible to know all these rules. The APA manual gives two
            examples, one in which ``De Vries'' is treated as the surname,
            and one in which ``Helmholtz'' is the surname and ``von'' is
            the ``von'' part of the name.

            In the former case, the author is referred to as, say, ``De Vries
            (1999)'' in text and ``De Vries, J. (1999).'', alphabetized under
            ``D'' in the reference list. In the second example, the author is
            referred to as, say, ``Helmholtz (1870)'' and listed in the
            reference list as ``Helmholtz, H. L. F. von. (1870).'',
            alphabetized under ``H''. To get this right, the author must be
            defined as
\begin{verbatim}
  author = {H. L. F. von Helmholtz},
\end{verbatim}
            or
\begin{verbatim}
  author = {von Helmholtz, H. L. F.},
\end{verbatim}
            then \BibTeX{} will assign the correct parts to the first names,
            ``von''-part, and last name. The seemingly logical
\begin{verbatim}
  author = {Helmholtz, H. L. F. von},
\end{verbatim}
            will not be formatted correctly, because ``von'' is now
            considered to be part of the first names and therefore
            abbreviated to ``v.''.

            Below, under example B-2, I will further discuss some
            issues about ``von'' parts.

            \EM The APA manual requires that numerals are alphabetized
            as if they were spelled out. This is not done by \pkg{apacite},
            so when applicable should be done by the user, e.g.\ by using
            the \cmd{\SortNoop} command as defined in
            \citeA[p.~404]{LaTeXcomp}. I give one fictitious example
            \cite{4.04-t10}.

            \citeA{4.04-1}\\
            \citeA{4.04-2}\\
            \citeA{4.04-3}\\
            \citeA{4.04-4}\\
            \citeA{4.04-5}\\
            \citeA{4.04-6}\\
            \citeA{4.04-7}\\
            \citeA{4.04-8}\\
            \citeA{4.04-9}\\
            \citeA{4.04-10}\\
            \citeA{4.04-11}\\
            \citeA{4.04-12}\\
            \citeA{4.04-13}\\
            \citeA{4.04-14}\\
            \EM I have not (yet?) studied referencing legal materials.
            In the first version of \pkg{apacite}, I defined a
            \entryname{literal} type, so that the users can
            literally format  such entries themselves. However,
            I think that if you need to refer to legal cases, you
            can find a way to use the \entryname{misc} type to get
            it right. I might study this and give examples in a
            next version of \pkg{apacite}.
\item[4.05] \citeA{4.05-1}\\
            \nocitemeta{4.05-2}\citeA{4.05-2}\\
            \EM Here I used the \cmd{\nocitemeta} command to let
            \pkg{apacite} know that a meta-analysis is reported and this
            work is included in it. As you can see, you can still cite
            the work by using the normal citation commands such as
            \cmd{\citeA}.
\item[4.08] \citeA{4.08-1}\\
            \citeA{4.08-2}\\
            \citeA{4.08-3}\\
            \EM I used the \verb+\bibliteral+ construction,
            which echoes its argument literally, to obtain the
            ``(with \ldots)'' description in the reference list. However,
            to \BibTeX{}, this is not a description, but a complicated
            accented character which is the second initial of the author.
            Consequently, this does not work properly with styles that put
            the initials \emph{before} the surnames, and care must be
            exercised with ``junior'' parts (see ex. 70). Furthermore,
            this construction must be put between a pair of
            braces in order for \BibTeX{} to treat it as an accented letter.
\item[4.09] \citeA{4.09-1}
\item[4.10] \citeA{4.10-1}\\
            \citeA{4.10-2}
\item[4.11] \citeA{4.11-1}\\
            \citeA{4.11-2}\\
            \EM Note that in the \fname{.bib} file, the ``junior'' part comes
            directly after the surname and a comma, with the initials
            after the ``junior'' part, whereas in the formatted
            reference list, the more logical order \opt{surname},
            \opt{initials}, \opt{junior}, is used.
\item[4.12] \citeA{4.12-1}
\item[4.13] \citeA{4.13-1}\\
            \EM ``For substantial reference works with a large editorial
            board, naming the lead editor followed by \texttt{et al.} is
            acceptable''. This is now implemented in \pkg{apacite} as
            follows: If there are 7 or more editors, only the first one
            is named, followed by ``et al.'' \cite{4.13-2}.\\
\item[4.15] \citeA{4.15-1}\\
            \EM I used the \cmd{\url} command from the \pkg{url}
            package to format internet addresses. This command, however,
            breaks addresses at different positions than the APA requires
            (in particular, \emph{after} a dot and \emph{before} the
            double slashes). Furthermore, I used a construction
            through a \verb+\bibnodot{.}+ expression, which swallows
            the dot, to prevent \BibTeX{} from adding a period
            after the internet address.\\
            \citeA{4.15-2}
\end{itemize}

\noindent
The rest is from section 4.16, the examples section. The section number
is omitted, but A--I are used instead for the unnumbered examples. For
the numbered examples, only the number is given.

The following enumerated list gives the example number, and
the two basic in-text citation commands, both in a full and
short form, of the examples.
\begin{enumerate}
\item[A-1] \cite{A-1}  \\ \cite{A-1} \\ \fullciteA{A-1} \\ \citeA{A-1}
\item \cite{ex1}  \\ \cite{ex1} \\ \fullciteA{ex1} \\ \citeA{ex1}
\item \cite{ex2}  \\ \cite{ex2} \\ \fullciteA{ex2} \\ \citeA{ex2}
\item \cite{ex3}  \\ \cite{ex3} \\ \fullciteA{ex3} \\ \citeA{ex3}
\item \cite{ex4}  \\ \cite{ex4} \\ \fullciteA{ex4} \\ \citeA{ex4}\\
      \cite{ex4-2} \\ \cite{ex4-2} \\ \fullciteA{ex4-2} \\ \citeA{ex4-2}\\
      \EM Note that the list of authors can be finished with
      ``\verb+and others+'' if there are more than 6 authors, or
      all authors can be given, in which case \pkg{apacite} truncates
      the list after the first 6. For compatibility with other styles,
      the latter is preferable.
\item \cite{ex5}  \\ \cite{ex5} \\ \fullciteA{ex5} \\ \citeA{ex5}
\item \cite{ex6}  \\ \cite{ex6} \\ \fullciteA{ex6} \\ \citeA{ex6}
\item \cite{ex7}  \\ \cite{ex7} \\ \fullciteA{ex7} \\ \citeA{ex7}\\
      \EM Here, I used
\begin{verbatim}
  month = {Spring},
\end{verbatim}
      which is perfectly fine to \BibTeX{}.
\item \cite{ex8}  \\ \cite{ex8} \\ \fullciteA{ex8} \\ \citeA{ex8}\\
      \EM Here, again, the order of the closing quotes and the
      following comma should be reversed according to the APA rules.
\item \cite{ex9}  \\ \cite{ex9} \\ \fullciteA{ex9} \\ \citeA{ex9}\\
      \EM Here, again, the order of the closing quotes and the
      following comma should be reversed according to the APA rules.
\item \cite{ex10} \\ \cite{ex10}\\ \fullciteA{ex10}\\ \citeA{ex10}
\item \cite{ex11} \\ \cite{ex11}\\ \fullciteA{ex11}\\ \citeA{ex11}\\
      \EM Apparently, a weekly newspaper is not a magazine and should
      therefore include ``p.'' or ``pp.'' in front of the page
      number(s), whereas a magazine article should not. I do not
      understand the distinction very well and find it not very
      useful as well, but will adhere to it. It means that the
      \entryname{newspaper} type gets the ``pp.'', whereas
      \entryname{magazine} and \entryname{article}, which are now
      identical, do not.
\item \cite{ex12} \\ \cite{ex12}\\ \fullciteA{ex12}\\ \citeA{ex12}\\
      \EM Here, \verb+\emph{DSM-IV}+ must be entered without an extra
      pair of braces in the title field, see ex.~73 for the intricacies
      of braces and \verb+\emph+ in the title field.
\item \cite{ex13} \\ \cite{ex13}\\ \fullciteA{ex13}\\ \citeA{ex13}
\item \cite{ex14} \\ \cite{ex14}\\ \fullciteA{ex14}\\ \citeA{ex14}
\item \cite{ex15} \\ \cite{ex15}\\ \fullciteA{ex15}\\ \citeA{ex15}
\item \cite{ex16} \\ \cite{ex16}\\ \fullciteA{ex16}\\ \citeA{ex16}
\item \cite{ex17} \\ \cite{ex17}\\ \fullciteA{ex17}\\ \citeA{ex17}\\
      \EM This case is the opposite of the more common situation where a
      reprint is cited and an original publication year is given. I could
      implement similar code to handle this case (and I may if I get requests
      to do this), but for the moment, I decided to trick the system: I
      defined a macro \cmd{\bibreftext} which must be called with two
      arguments. Initially, it places the second argument in the text, but
      just before the bibliography, its definition is changed so that the
      first argument is placed in the reference list. By defining
\begin{verbatim}
  year = {{\protect\bibreftext{1992}{1992/1993}}},
\end{verbatim}
      in the \fname{.bib} file, the year is ``1992'' in the reference
      list and ``1992/1993'' in the text citations. The \cmd{\protect}
      is necessary to make sure that the \cmd{\bibreftext} macro is not
      written out in the label-part of the bibliography, because then
      only ``1992'' would appear in the text reference. The additional
      pair of braces are necessary to make sure that the entry
      is alphabetized correctly. When these are omitted, the entry
      is alphabetized under the ``P'' of ``protect''. Of course, the
      latter is only important if the reference list contains
      multiple works by the same author(s), but the same phenomenon
      is encountered when the \fieldname{author} or \fieldname{editor}
      field starts with a command, or when there is no author or editor
      and the \fieldname{title} field starts with a command, as in
      the entry \citeA{4.04-t10} introduced in 4.04 above.

      Of course, you can save yourself a lot of trouble by
      obtaining the original article and citing that.
\item \cite{ex18} \\ \cite{ex18}\\ \fullciteA{ex18}\\ \citeA{ex18}
\item \cite{ex19} \\ \cite{ex19}\\ \fullciteA{ex19}\\ \citeA{ex19}
\item \cite{ex20} \\ \cite{ex20}\\ \fullciteA{ex20}\\ \citeA{ex20}
\item \cite{ex21} \\ \cite{ex21}\\ \fullciteA{ex21}\\ \citeA{ex21}\\
      \EM Apparently, a translator and original publication date
      do not have to be mentioned here.
\item Seidenberg and McClelland's study \cite<as cited in>{ex22}\\
      %\cite{ex22} \\ \cite{ex22}\\ \fullciteA{ex22}\\ \citeA{ex22}
\item[B-1] \cite{B-1}  \\ \cite{B-1} \\ \fullciteA{B-1} \\ \citeA{B-1}
\item \cite{ex23} \\ \cite{ex23}\\ \fullciteA{ex23}\\ \citeA{ex23}
\item \cite{ex24} \\ \cite{ex24}\\ \fullciteA{ex24}\\ \citeA{ex24}\\
      \EM The APA manual gives a number, which would not ordinarily be
      done with proper books and which suggests that it could also
      be treated as a report. In that case, the default ``Tech. Rep.''
      description should be left out, however. This can be accomplished
      by defining
\begin{verbatim}
  type = {\bibnotype},
\end{verbatim}
      which flags that the type must be left out, see example 42.
      However, for books, it now works fine leaving the \fieldname{type}
      field empty and giving the number in the \fieldname{number} field.
\item \cite{ex25} \\ \cite{ex25}\\ \fullciteA{ex25}\\ \citeA{ex25}\\
      \EM According to the note after ex.~25 in the APA manual,
      an edited book with ``just one'' author should list
      the editor (and translator, when available) after the
      title, just like the translator. An example of this is
      \citeA{ex25-t1}. (Should the original years of publication
      be used here?)

      However, what does ``just one author'' mean here? Does it mean that if
      there are 2 authors and 1 or more editors, the editors should
      not be mentioned, or the authors should not be mentioned?
      Presumably, it means the following:
      In the default case, an edited book consists of several chapters,
      with different chapters written by different authors. In this case,
      the reference to the whole book lists only the editor(s).
      If every chapter is written by the same author, and there are 1
      or more editors, their condition holds. I think that if all
      chapters are written by the same $n$ ($\geq 2$) authors, it should
      also be treated in this way, i.e., with the authors in the
      author position and the editor behind the title. A difficult
      case would appear if the book were a collection of works of, say,
      2 authors, with some chapters written by one and some by the
      other, and some jointly, as in \citeA{ex25-t2}. These choices have
      to be made by the user, however, \pkg{apacite} will format author,
      editor, and translator, whenever available.
\item \cite{ex26} \\ \cite{ex26}\\ \fullciteA{ex26}\\ \citeA{ex26}\\
      \EM According to section~3.97, the title should be in italics.
      In this example, in the citation, it is not given in italics,
      but this must be an error, because they do give it in italics
      on p.~219 (but give no date there).
\item \cite{ex27} \\ \cite{ex27}\\ \fullciteA{ex27}\\ \citeA{ex27}
\item \cite{ex28} \\ \cite{ex28}\\ \fullciteA{ex28}\\ \citeA{ex28}
\item \EM I defined a \verb+\DSMcite+ macro and some related
      alternatives to get the \emph{DSM} references right:\\
      \cmd{\DSMcite}, first citation: \DSMcite{ex29} \\
      \cmd{\DSMcite}, second citation: \DSMcite{ex29} \\
      \cmd{\DSMciteA}, first citation: \DSMfullciteA{ex29} \\
      \cmd{\DSMciteA}, second citation: \DSMciteA{ex29} \\
      \cmd{\DSMciteNP}, first citation: \DSMfullciteNP{ex29} \\
      \cmd{\DSMciteNP}, second citation: \DSMciteNP{ex29} \\
      \EM Actually, these macros are not related to the \emph{DSM} at all,
      except that they are defined to handle these. These macros
      act as a normal citation the first time a \emph{DSM} is
      referenced and all subsequent times as a suitably formatted
      \cmd{\citeauthor}. So the only difference with ordinary
      citations is that the year is left out the second and later
      times. By defining
\begin{verbatim}
  firstkey  = {{American Psychiatric Association}},
  key       = {{\APACcitebtitle{DSM-IV}}},
\end{verbatim}
      using the \cmd{\APACcitebtitle} macro introduced before,
      the \emph{DSM} citations are correctly formatted.
      However, the \cmd{\DSM*} macros are still very rough. Multiple
      citations are not handled well, so only one work should
      be cited with such a macro. Furthermore, the pre- and
      postfixes (through \verb+<...>+ and \verb+[...]+) are not
      available.

      \EM The following two are for testing the ordering (sorting)
      in \BibTeX{} and \MakeIndex{} \cite{ex29-2,ex29-3}.

\item \cite{ex30} \\ \cite{ex30}\\ \fullciteA{ex30}\\ \citeA{ex30}\\
      \EM Here, ``6th ed.'' comes before ``Vols.\ 1--20'', so it is
      Vols.\ 1--20 of the 6th edition. \pkg{apacite} now uses this
      order. Pre-[2003/09/05] versions placed volumes before editions,
      which is appropriate when different volumes of a series
      are not updated jointly, so that you may have the 3rd edition
      of Vol.~2, but only the 1st edition of Vol.~6. Should you
      want this, you have to trick the system by putting both
      elements in the \fieldname{edition} field:
\begin{verbatim}
  edition = {\BVOLS\ 1--20, 6th},
\end{verbatim}
      and leaving the \fieldname{volume} field empty.\\
      \EM Does the remark about large editorial boards mean that the
      APA intended to add ``et al.'' here? Anyway, here is an example
      of such a situation: \citeA{ex30-2}.
\item \cite{ex31} \\ \cite{ex31}\\ \fullciteA{ex31}\\ \citeA{ex31}
\item \cite{ex32} \\ \cite{ex32}\\ \fullciteA{ex32}\\ \citeA{ex32}
\item \cite{ex33} \\ \cite{ex33}\\ \fullciteA{ex33}\\ \citeA{ex33}
\item[B-2] \cite{B-2} \\ \cite{B-2} \\ \fullciteA{B-2} \\ \citeA{B-2} \\
      \EM This is a suitable point to discuss ``von'' parts further.
      One of the editors in the current example is called
      P. van den Broek. I am Dutch and I recognize this as a Dutch
      name, so let's assume that this person is Dutch. Let us now
      consider the situation that this person is the (only) author.
      Then according to the Dutch rules, this person would be listed in the
      reference list as ``Broek, P. van den. (1992)'' and listed under ``B'',
      but referred to in the text as ``Van den Broek (1992)'', including
      the ``von'' part and capitalizing the first letter. When an
      initial precedes the ``von'' part (e.g., when there are two
      primary authors with this surname, but with different initials),
      then this person would be referred to in the text as
      ``P. van den Broek (1992)'', including the ``von'' part, but now
      in lower case. The Dutch rules are very difficult to do right in
      \BibTeX{}/\LaTeX{} and they conflict with the rules for other
      languages, and presumably with the rules of the APA, which would
      require this person to be alphabetized under ``V'', at least.

      For the current editor position, there is no big problem with
      the APA rules. The ``von'' part can really be put in the ``von''
      part as recognized by \BibTeX{}, which I did, or it can be enclosed
      in braces, making the whole last name one part. Either will come
      out right and the editor will not be referred to in the text
      anyway. With styles that put the initials behind the surname,
      it will only work right when put in the ``von'' part. In the
      author position, however, things are different. Then this author
      could be formatted as
\begin{verbatim}
  author = {Van {\lowercase{D}}en Broek, P.},
\end{verbatim}
      where the \verb+{\lowercase{D}}+ construction is used to prevent
      \BibTeX{} from interpreting it as a ``von'' part, or
\begin{verbatim}
  author = {Van{\ }den{\ }Broek, P.},
  author = {{Van den Broek}, P.},
\end{verbatim}
      which in most situations gives the correct version according to the
      Dutch rules with the correct alphabetizing according to the APA
      rules. Only in situations when the initial would appear
      \emph{before} the surname (when there are other authors with the
      same surname or when using other styles), this would lead to
      unwanted results.

      If you really want to alphabetize according to the Dutch rules
      (which is \emph{not} APA in my interpretation), you can use the
      following trick. Define a macro \verb+\Dutchvon+ with
      two arguments which is equivalent to the \verb+\bibreftext+
      macro discussed in ex.~17 above. That is, in the beginning
      of the document, it is defined as echoing its second argument,
      and just before the reference list, it is redefined to echo
      its first argument. Then define the author as
\begin{verbatim}
  author = {van den {\protect\Dutchvon{Broek}{Van den Broek}}, P.},
\end{verbatim}
      which, combined with the \pkg{apacite} formatting implies
      that in text, ``Van den Broek'' is used, whereas the author
      is alphabetized under ``B'' in the reference list, and listed
      as ``Broek, P. van den''. Of course, we could simply have
      used \verb+\bibreftext+ again, but that was used to cover a
      completely different situation, and I can easily imagine that
      when switching to another citation style, the former application
      (i.e., 1992/1993) must be kept, but the ``von'' part is handled
      differently, so that \verb+\Dutchvon+ must be redefined.
      Based on this example, users should be able to define their
      own tricks for similar peculiarities.

      Finally, note that in Belgium, where many people
      speak Dutch as well, it is customary to alphabetize under ``V''
      anyway (``Van Damme'').
\item \cite{ex34} \\ \cite{ex34}\\ \fullciteA{ex34}\\ \citeA{ex34}
\item \cite{ex35} \\ \cite{ex35}\\ \fullciteA{ex35}\\ \citeA{ex35}
\item \cite{ex36} \\ \cite{ex36}\\ \fullciteA{ex36}\\ \citeA{ex36}\\
      \EM I had to use some tricks to get the Series and Volume Editors'
      descriptions right. The ``(Series Ed.)'' is according to
      \BibTeX{} part of the last name of the first editor,
      and I used a \verb+\bibeditortype+ macro to get the volume
      editor right. The \verb+\bibeditortype+ macro has one argument.
      The macro changes the definitions of the macros \verb+\BED+ and
      \verb+\BEDS+ to the argument and then immediately changes the
      definitions back to their old definitions. So it makes a one-time
      change.

      It seems to me that only mentioning the
      volume editor as ``editor'' would be sufficient, but the APA
      requires both. It raises the question when series editors
      should be mentioned: only if the specific volume in the
      series is itself an edited work (as with this handbook),
      or also with books that are published in a series. Many books
      are published in some series, with one or more series editors
      and I have never seen any mentioning of the series editors of
      such books, but it is not clear to me when this should be done
      then.
\item \cite{ex37} \\ \cite{ex37}\\ \fullciteA{ex37}\\ \citeA{ex37}
\item \cite{ex38} \\ \cite{ex38}\\ \fullciteA{ex38}\\ \citeA{ex38}
\item \cite{ex39} \\ \cite{ex39}\\ \fullciteA{ex39}\\ \citeA{ex39}\\
      \EM Here, it is stated that the translator's name should be put
      after the editor's name when both are different, but in example
      40, the translators are put after the title, before the editors.
      Apparently, if the book is a collection of works by one
      author(-group), edited and translated, then the translator
      should come after the editor, but if different chapters
      are translated by different translators, or perhaps not all
      are translated works, then the translator should come after
      the title. Of course, \pkg{apacite} has no way of knowing this.
      In the current implementation, for an \entryname{incollection},
      if the editor and translator are the same, they are formatted as
      in this example, whereas if they are different, they are treated
      as in example 40 below. I think this will be satisfactory in
      most cases. If you really want the translator after the editor
      when they are different, you can trick the system in a way
      similar to example 36.
\item \cite{ex40} \\ \cite{ex40}\\ \fullciteA{ex40}\\ \citeA{ex40}\\
      \EM See my comments to example 39 above.
\item[C-1] \cite{C-1}  \\ \cite{C-1} \\ \fullciteA{C-1} \\ \citeA{C-1}
\item \cite{ex41} \\ \cite{ex41}\\ \fullciteA{ex41}\\ \citeA{ex41}
\item \cite{ex42} \\ \cite{ex42}\\ \fullciteA{ex42}\\ \citeA{ex42}\\
      \EM I used the \cmd{\bibnotype} construction discussed
      in example~24 to suppress the default ``(Tech.\ Rep.{})''
      description. Actually, I don't think there is anything wrong with
      calling a report a report, be it ``technical'' by default, or by
      specifying
\begin{verbatim}
  type = {Report},
\end{verbatim}
      except that it's not done so in this specific example in the
      APA manual. But there does not seem to be a rule against it.
\item \cite{ex43} \\ \cite{ex43}\\ \fullciteA{ex43}\\ \citeA{ex43}
\item \cite{ex44} \\ \cite{ex44}\\ \fullciteA{ex44}\\ \citeA{ex44}
\item \cite{ex45} \\ \cite{ex45}\\ \fullciteA{ex45}\\ \citeA{ex45}
\item \cite{ex46} \\ \cite{ex46}\\ \fullciteA{ex46}\\ \citeA{ex46}\\
      \EM Note that in the example in the APA manual, ``Western Australia''
      \emph{is} mentioned in the publisher location, whereas their rule
      explicitly states that this should not be the case in this
      situation, and they stress that again below the example.
      Therefore, I did not mention ``Western Australia'' in the
      \fieldname{publisher} field.
\item \cite{ex47} \\ \cite{ex47}\\ \fullciteA{ex47}\\ \citeA{ex47}
\item \cite{ex48} \\ \cite{ex48}\\ \fullciteA{ex48}\\ \citeA{ex48}
\item \cite{ex49} \\ \cite{ex49}\\ \fullciteA{ex49}\\ \citeA{ex49}
\item \cite{ex50} \\ \cite{ex50}\\ \fullciteA{ex50}\\ \citeA{ex50}
\item \cite{ex51} \\ \cite{ex51}\\ \fullciteA{ex51}\\ \citeA{ex51}
\item \cite{ex52} \\ \cite{ex52}\\ \fullciteA{ex52}\\ \citeA{ex52}
\item \cite{ex53} \\ \cite{ex53}\\ \fullciteA{ex53}\\ \citeA{ex53}
\item \cite{ex54} \\ \cite{ex54}\\ \fullciteA{ex54}\\ \citeA{ex54}
\item \cite{ex55} \\ \cite{ex55}\\ \fullciteA{ex55}\\ \citeA{ex55}\\
      \EM It seems logical to suppress the original year if it's the
      same as the year of the abstract.

      \EM According to example 54 and the text below it, an ``A''
      or ``B'' should be added to the volume number of \emph{DAI}
      of this entry, presumably an ``A'', but because they did not
      give one, I left it out as well.
\item \cite{ex56} \\ \cite{ex56}\\ \fullciteA{ex56}\\ \citeA{ex56}
\item \cite{ex57} \\ \cite{ex57}\\ \fullciteA{ex57}\\ \citeA{ex57}
\item \cite{ex58} \\ \cite{ex58}\\ \fullciteA{ex58}\\ \citeA{ex58}
\item \cite{ex59} \\ \cite{ex59}\\ \fullciteA{ex59}\\ \citeA{ex59}
\item \cite{ex60} \\ \cite{ex60}\\ \fullciteA{ex60}\\ \citeA{ex60}
\item \cite{ex61} \\ \cite{ex61}\\ \fullciteA{ex61}\\ \citeA{ex61}\\
      \EM The description of the data goes into the
      \fieldname{type} field. It may be more logical to put it
      in the \fieldname{title} field, with some trickery to indicate
      that it's a description and not a title, but putting it in the
      \fieldname{type} field already ensures that it's formatted
      correctly, so I think I'm gonna be lazy on this one. The same
      applies more or less to reviews (G-1, 63, 64), although there
      it is logical to put at least ``Review'' in the \fieldname{type}
      field, so as we're already there, we might as well make it
      complete.
\item \cite{ex62} \\ \cite{ex62}\\ \fullciteA{ex62}\\ \citeA{ex62}
\item[G-1] \cite{G-1}  \\ \cite{G-1} \\ \fullciteA{G-1} \\ \citeA{G-1}\\
      \EM See my comments to example 61 above about the logic of the
      \fieldname{type} field. For a review, the \fieldname{type}
      field must contain the message that it is a review of a
      \emph{book} (or motion picture, or whatever it is a review of)
      and give the title of the reviewed work in italics, which must
      be manually formatted, e.g., by using the \cmd{\APACcitebtitle}
      macro. The author of the reviewed work should apparently not be
      mentioned.
\item \cite{ex63} \\ \cite{ex63}\\ \fullciteA{ex63}\\ \citeA{ex63}
\item \cite{ex64} \\ \cite{ex64}\\ \fullciteA{ex64}\\ \citeA{ex64}
\item \EM The rules for movies, television series, etc.\ do not appeal
      very much to me, although they are not nearly as bad as those
      for music recordings (ex.~69 and further, see below). My
      primary objection is that when referring to movies and TV series,
      the title is the most important characteristic. The director
      may be a good second in some cases, but producers and writers are
      almost always unknown to the general public and not relevant
      when referring to them. You talk about the movie \emph{Jaws}
      and not about Spielberg (1975) or whoever wrote the script.
      Recognizing the primary creative input or authorship may be
      politically correct, but not particularly informative.
      However, you may think that my objections are mainly
      driven by the difficulties of trying to format the entries
      in \BibTeX{}. Anyway, here are my attempts to implement
      the APA rules.\\
      \cite{ex65-1} \\ \cite{ex65-1}\\ \fullciteA{ex65-1}\\ \citeA{ex65-1} \\
      \cite{ex65-2} \\ \cite{ex65-2}\\ \fullciteA{ex65-2}\\ \citeA{ex65-2} \\
      \EM Note that this and the following ``Available from'' must be between
      parentheses (accomplished by putting it in the \fieldname{note} field),
      whereas ``Available from'' some web site must not be between
      parentheses, cf.\ ex.~95 (accomplished by putting it in the
      \fieldname{howpublished} field).

      \EM Here I also used the \cmd{\bibliteral} construction again, which
      was introduced in my discussion of 4.08 above.
      Furthermore, I used the \cmd{\bibskipbracenodot}
      macro, which suppresses the period after the closing
      brace after ``(Producer)''.\\
      \cite{ex65-3} \\ \cite{ex65-3}\\ \fullciteA{ex65-3}\\ \citeA{ex65-3}\\
      \EM Here, I simply let ``(Producer)'' be part of the author name,
      but because of this addendum, the entry is not sorted correctly
      in the reference list: It should be before \citeA{APAManual},
      but because of the addendum, the authors are different to \BibTeX{}
      and this one comes after \citeA{APAManual}. I don't see an easy
      solution to this (presumably extremely rare) problem. Therefore,
      in such a case, the \fname{.bbl} file should be edited manually.
\item \cite{ex66} \\ \cite{ex66}\\ \fullciteA{ex66}\\ \citeA{ex66} \\
      \EM Here, I used the same tricks as in example 65.
\item \cite{ex67} \\ \cite{ex67}\\ \fullciteA{ex67}\\ \citeA{ex67} \\
      \EM I used several tricks introduced before: the \cmd{\bibeditortype}
      (cf.\ ex.~36) construction to be able to use ``(Producer)'' in
      the editor position instead of  ``(Ed.)'', and the ones used
      in the previous examples. Because there is another Miller as well,
      the initial is given here. To make sure that \BibTeX{} thinks
      that there is only one initial, but the ``(Producer)'' is formatted
      correctly, the name is defined as follows:
\begin{verbatim}
  editor = {Miller, {\bibliteral{R\protect\bibeditortype{Producer}}}},
\end{verbatim}
      It is then formatted correctly both in the text and in the reference
      list if \cmd{\bibeditortype} is defined to swallow its argument
      in the text, and redefined prior to the bibliography. After the
      bibliography (before the index), it has to be redefined again
      to swallow its argument. However, it leads to two different entries
      in the author index, so if you're requesting an author index,
      you still have to edit the \fname{.bbl} file manually, as with
      all such types of ``authors'' or ``editors''.
\item \cite{ex68} \\ \cite{ex68}\\ \fullciteA{ex68}\\ \citeA{ex68}\\
      \EM Again, I used several tricks that were also used in the previous
      examples.
\item \EM The rules for music recordings do not make any sense, and they
      are not entirely clear as well. It would seem to me that
      most pop, rock, and other ``light'' music songs are almost always
      attributed to the performing artist. You can usually read who the
      composer and lyricist of the song are on the inlay-details
      of the CD, but if you want the reader to refer to ``My Way'',
      the reader will easier find it (in a CD store or on the
      internet, for example) under ``Frank Sinatra'' than under
      the composer's name (whoever that may be). Furthermore, the
      year of release of the CD (or other medium) seems more relevant
      in referring to the CD than the year of copyright of the
      specific song, or the recording date.

      For classical music, things are different. Then, the writer (composer)
      is usually more important than the performing artist, although
      the percentage of people who know who wrote the lyrics for
      Mozart's operas may not be high. Furthermore, it may be
      time-consuming and not very relevant for most authors and
      readers to find the year in which a certain piece was written,
      although strictly speaking, we should use the ``date of copyright'',
      which is not relevant for a lot of classical music.

      However, when referring to Beethoven's 9th symphony, say, it will
      typically not be very informative to refer to a specific recording on a
      specific CD, unless that specific recording is the subject of interest,
      but then it would seem that the ``performing artist'' (conductor or
      orchestra) should take the role of ``author''. For example, when
      comparing a performance of this symphony conducted in 1975 by Masur with
      a performance of this symphony conducted in 1990 by Bernstein (assuming
      this exists), it does not seem logical to compare ``9th Symphony''
      (Beethoven, 1823a, tracks~1--4) with ``9th Symphony'' (Beethoven, 1823b,
      tracks~1--4), but more logical to compare Masur (1975) with
      Bernstein (1990).

      Finally, it seems a bit strange that the track number should be
      mentioned in text, but not in the reference list.

      Again, you may think that my objections are mainly
      driven by the difficulties of trying to format the entries
      in \BibTeX{}. Nevertheless, here are my attempts to implement
      the APA rules.\\
      \begin{tabular}{ll}
      \verb+\citeAsong+:  & \citeAsong{ex69-1} \\
      \verb+\citesong+:   & \citesong{ex69-1}  \\
      \verb+\citesongNP+: & \citesongNP{ex69-1}\\
      \verb+\citeAsong+:  & \citeAsong{ex69-2} \\
      \verb+\citesong+:   & \citesong{ex69-2}  \\
      \verb+\citesongNP+: & \citesongNP{ex69-2}
      \end{tabular}\\
      \EM I defined and used several macros, i.e., \verb+\citesong+
      and some variations. These cite a song as required by the APA
      (except for the relative placement of closing quotes and comma).
      For that, the \fieldname{key} field (and possibly \fieldname{firstkey})
      should contain the relevant information in the form of a protected
      \verb+\bibsong+ command:
\begin{verbatim}
  key = {{\protect\bibsong{author}{year}{song title}{track info}}},
\end{verbatim}
      Furthermore, I used some tricks to get the reference list
      entry right. I used the \entryname{incollection} type as
      a basis, with the CD title in the \fieldname{booktitle} field.
      At first sight, it seems natural to define
\begin{verbatim}
  type = {CD},
\end{verbatim}
      but with an \entryname{incollection}, the \fieldname{type}
      field is put after the title, not after the book title.
      I plan to change this for music-types (CD, record, etc.),
      but this is currently not yet implemented. This means that the
      ``[CD]'' description must be put in the \fieldname{booktitle}
      field, after the CD title and with explicit formatting
      commands:
\begin{verbatim}
  booktitle = {Arkansas Traveler {\upshape[\uppercase{CD}]}},
\end{verbatim}
      On the other hand, we can now use the \fieldname{type}
      field to put the recording artists in for the second example:
\begin{verbatim}
  type = {Recorded by G. Bok, A. Mayo, \& E. Trickett},
\end{verbatim}
      We have to change the ``In'' string to ``On''
      (``In'' a book versus ``On'' a CD). This is done with a
      macro similar to the \verb+\bibeditortype+ macro discussed
      in example 36. In this case, the macro is \verb+\bibInstring+,
      which has one argument. The macro changes the definition of
      the macro \verb+\BIn+ to the argument and then immediately changes
      the definition back to the old definition. So it makes a one-time
      change. Of course, the \verb+\bibInstring+ macro only works
      if it is executed \emph{before} the \verb+\BIn+ macro. The latter
      macro comes before the \fieldname{booktitle}, so
      \verb+\bibInstring+ must be added to a field that comes before
      that. I put it in the \fieldname{title} field:
\begin{verbatim}
  title = {Over the Waterfall{\bibInstring{\BOn}}},
\end{verbatim}
      where \verb+\BOn+ is defined by default as ``\BOn''.
      Finally, the recording date of the second example (1990) must
      be put in the \fieldname{note} field.

      Note that this is all trickery within the \LaTeX{} domain
      and the \fname{.bib} file, i.e., the user-definable and
      user-customizable area.
\item \cite{ex70} \\ \cite{ex70}\\ \fullciteA{ex70}\\ \citeA{ex70}
\item \cite{ex71-1} \\ \cite{ex71-1}\\ \fullciteA{ex71-1}\\ \citeA{ex71-1} \\
      \cite{ex71-2} \\ \cite{ex71-2}\\ \fullciteA{ex71-2}\\ \citeA{ex71-2} \\
      \EM Here and in almost all remaining examples, the
      ``Retrieved \ldots'' message must be put in the
      \fieldname{howpublished} field if it refers to an internet
      address or aggregated database. See the discussion of 4.15
      above for the use of \verb+\url+ and \verb+\bibnodot{.}+.
\item \cite{ex72} \\ \cite{ex72}\\ \fullciteA{ex72}\\ \citeA{ex72}
\item \cite{ex73} \\ \cite{ex73}\\ \fullciteA{ex73}\\ \citeA{ex73} \\
      \EM Note that the APA manual breaks a line \emph{after} a
      period instead of before. I take it that it's not very
      detrimental to use slightly different line breaking through
      the \pkg{url} package. Furthermore, the APA manual uses a
      typeface in which `1' (one) and `l' (ell) are not distinguishable.
      This requires some guessing of the correct url's. I think that this
      conflicts with their own stress on accuracy w.r.t.\ url's.

      \EM Note the subtle use of braces to get \emph{Homo sapiens}
      correctly formatted: The capital in \verb+\emph{Homo}+
      is retained by \BibTeX{}, whereas the capital in
      \verb+{\emph{Sapiens}}+ is changed into a lower case letter
      by \BibTeX{}, because of the extra pair of braces.
\item \cite{ex74} \\ \cite{ex74}\\ \fullciteA{ex74}\\ \citeA{ex74}\\
      \EM As mentioned in section~\ref{sec:compat-backref}, I had
      to use some nasty code in the current document to have the
      bibliography entry of this reference formatted correctly
      when \pkg{backref} is loaded.
\item \cite{ex75} \\ \cite{ex75}\\ \fullciteA{ex75}\\ \citeA{ex75}
\item \cite{ex76} \\ \cite{ex76}\\ \fullciteA{ex76}\\ \citeA{ex76}
\item \cite{ex77} \\ \cite{ex77}\\ \fullciteA{ex77}\\ \citeA{ex77}
\item \cite{ex78} \\ \cite{ex78}\\ \fullciteA{ex78}\\ \citeA{ex78}
\item \cite{ex79} \\ \cite{ex79}\\ \fullciteA{ex79}\\ \citeA{ex79}
\item \cite{ex80} \\ \cite{ex80}\\ \fullciteA{ex80}\\ \citeA{ex80}
\item \cite{ex81} \\ \cite{ex81}\\ \fullciteA{ex81}\\ \citeA{ex81}
\item \cite{ex82} \\ \cite{ex82}\\ \fullciteA{ex82}\\ \citeA{ex82}
\item \cite{ex83} \\ \cite{ex83}\\ \fullciteA{ex83}\\ \citeA{ex83} \\
      \EM Here and in the next example, both ``Paper presented \ldots''
      and ``Retrieved \ldots'' must be put in a single
      \fieldname{howpublished} field.
\item \cite{ex84} \\ \cite{ex84}\\ \fullciteA{ex84}\\ \citeA{ex84}
\item \cite{ex85} \\ \cite{ex85}\\ \fullciteA{ex85}\\ \citeA{ex85} \\
      \EM When referring to a message to a newsgroup, internet forum, etc.,
      use the \entryname{misc} type. Then define
\begin{verbatim}
  type = {\bibmessage},
\end{verbatim}
      and use the \fieldname{number} field when relevant. Furthermore,
      you should include a description like ``Message posted to ... ''
      in the \fieldname{howpublished} field.  Then the entry is formatted
      correctly: If the message has a number, say 1, it reverts to
      ``\verb+[\bibmessage\ 1]+'', i.e., ``[Msg\ 1]'' with the default
      definition of \cmd{\bibmessage}. If there is no number, the
      type identifier is omitted (because it must be mentioned in the
      \fieldname{howpublished} field anyway). In both cases,
      however, the title is formatted as an article title, i.e., not
      italicized.
\item \cite{ex86} \\ \cite{ex86}\\ \fullciteA{ex86}\\ \citeA{ex86}
\item \cite{ex87} \\ \cite{ex87}\\ \fullciteA{ex87}\\ \citeA{ex87}
\item \cite{ex88} \\ \cite{ex88}\\ \fullciteA{ex88}\\ \citeA{ex88}
\item \cite{ex89} \\ \cite{ex89}\\ \fullciteA{ex89}\\ \citeA{ex89}
\item \cite{ex90} \\ \cite{ex90}\\ \fullciteA{ex90}\\ \citeA{ex90} \\
      \EM Here, ``de Ridder'' may well be a Dutch name. If the APA
      manual is serious about its rule to format names as they
      should according to the rule of the country of origin,
      this name should then be formatted as ``De Ridder''.
\item \cite{ex91} \\ \cite{ex91}\\ \fullciteA{ex91}\\ \citeA{ex91}
\item \cite{ex92} \\ \cite{ex92}\\ \fullciteA{ex92}\\ \citeA{ex92}
\item \cite{ex93} \\ \cite{ex93}\\ \fullciteA{ex93}\\ \citeA{ex93}\\
      \EM Strangely enough, the APA apparently does not find it necessary
      to give a retrieval date or more exact publication date for
      downloaded software, whereas it does require a retrieval date
      for downloaded papers.
\item \cite{ex94} \\ \cite{ex94}\\ \fullciteA{ex94}\\ \citeA{ex94}\\
      \EM The example in the APA manual does not list a date, not even
      a ``n.d.''. Given the logic of the APA manual, this seems an error
      to me, so I added a ``\verb+\bibnodate+'' in the \fieldname{year}
      field, which becomes ``\bibnodate'' in the output. If it would
      really be the rule to omit the year, then it is not clear when
      a year must be omitted and when not, or replaced by ``n.d.''.
      Furthermore, formatting in both the \fname{.sty} file and
      the \fname{.bst} file would have to be adapted to anticipate
      a missing year. Currently, I assume that there should always
      be something that acts like a date (a year, an ``in press'',
      or a ``n.d.'').

      \EM To get the version number correctly formatted, I added the
      information and the correct formatting to the title field:
      \verb+{\upshape(\uppercase{V}ersion~4)}+. A version is a kind
      of edition, but it is more difficult to obtain the correct
      formatting if the \fieldname{edition} field would be used,
      because ``Version'' comes before the number and ``ed.'' after.
\item \cite{ex95} \\ \cite{ex95}\\ \fullciteA{ex95}\\ \citeA{ex95}\\
      \EM Note that here, unlike ex.~44 and ex.~91,
      ``U.S.'' is omitted before ``Department of Health and Human
      Services''.
\end{enumerate}

%%%%%%%%%%%%%%%%%%%%%%%%%%%%%%%%%%%%%%%%%%%%%%%%%%%%%%%%%%%%%%%%%%%%%%%
%
% Redefine commands that should be different in the bibliography.
\renewcommand{\bibreftext}[2]{#1}%
\renewcommand{\Dutchvon}[2]{#1}%
\renewbibeditortype
\def\bibskipbracenodot{\aftergroup\swallowdot}
%
% Dirty trick: request apacitex.bst if apacite is loaded with one of the
% index options, and request apacite.bst otherwise.
\makeatletter
\if@APAC@index
  \bibliographystyle{apacitex}% with index option
\else
  \bibliographystyle{apacite}% or without
\fi
\makeatother
%
% Solve the `#' problem of backref, see the backref section above.
% This hack provided by Heiko Oberdiek.
\ifbackrefloaded
  \catcode`\#=12\relax
\fi
%
% Include the bibliography
\bibliography{apa5ex}
%
% Reset the above hack.
\ifbackrefloaded
  \catcode`\#=6\relax
\fi
%
% Again redefine a command.
\renewcommand{\bibeditortype}[1]{}%
%
% Include the author index if the index option is on.
\printindex[autx]
%%%%%%%%%%%%%%%%%%%%%%%%%%%%%%%%%%%%%%%%%%%%%%%%%%%%%%%%%%%%%%%%%%%%%%%
\end{document}
%%%%%%%%%%%%%%%%%%%%%%%%%%%%%%%%%%%%%%%%%%%%%%%%%%%%%%%%%%%%%%%%%%%%%%%
% EOF apacite.tex

