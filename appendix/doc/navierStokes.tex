\section{\texttt{navierStokes}}
The main structure for the Navier-stokes class \texttt{navierStokes}. This class contains all the functions related to computation of the Navier-stokes problem. Below is set of functions that acts as the interface to the class.

\begin{figure}[h]
\centering
\begin{tikzpicture}[%
  grow via three points={one child at (0.5,-0.7) and
  two children at (0.5,-0.7) and (0.5,-1.4)},
  edge from parent path={(\tikzparentnode.south) |- (\tikzchildnode.west)}]
  \node {\texttt{hybrid}}
    child { node {\texttt{vortexMethod}}
    	child {node {\texttt{vortexBlobs}}}
    	child {node {\texttt{panels}}}  	
    }
    child [missing] {}				
    child [missing] {}				
    child { node [selected] {\texttt{navierStokes}}};
    %child { node [selected] {tex}
    %  child { node {generic}}
    %  child { node [optional] {latex}}
    %  child { node {plain}}
    %}
    %child [missing] {}
    %child { node {texdoc}};
\end{tikzpicture}
\end{figure}


\subsection*{Class structure:}
\begin{figure}[h]
\begin{tikzpicture}[%
  grow via three points={one child at (0.5,-0.7) and
  two children at (0.5,-0.7) and (0.5,-1.4)},
  edge from parent path={(\tikzparentnode.south) |- (\tikzchildnode.west)}]
  \node [selected] {\texttt{navierStokes}}
    child { node {\texttt{1. \_\_init\_\_}}}
    child {node  {\texttt{2. getCoordinates}}}
    child { node {\texttt{3. setVelocity}}}
    child { node {\texttt{4. setPressure}}}
    child { node {\texttt{5. getBoundaryCoordinates}}}                        
    child { node {\texttt{6. evolve}}}
    child { node {\texttt{7. getVorticity}}}                        
    child { node {\texttt{8. getMeshPosition}}}                        
    child { node {\texttt{9. getProbeGridParameters}}}                        
    child { node {\texttt{10. plotVelocity}}}                        
    child { node {\texttt{11. plotPressure}}}                        
    child { node {\texttt{12. plotVorticity}}}                        
    child { node {\texttt{13. save}}}                        
    child { node {\texttt{14. saveVelocity}}}                                            
    child { node {\texttt{15. savePressure}}}                                            
    child { node {\texttt{16. saveVorticity}}}
    child { node [optional]{\texttt{17. \_moveMesh}}};
	%  child { node [selected] {tex}
    %  child { node {generic}}
    %  child { node [optional] {latex}}
    %  child { node {plain}}
    %}
    %child { node {texdoc}};
\end{tikzpicture}
\end{figure}

%\subsection*{Class structure:}
%\begin{table}[h]
%\begin{tabular}{rl}
%	\texttt{navierStokes}: &  \\ \hline
%	-- & \texttt{\_\_init\_\_} \\
%	& \\
%	-- & \texttt{initialConditions} \\ 
%	-- & \texttt{evolve} \\ 
%	-- & \texttt{boundaryCoordinates} \\ 
%	-- & \texttt{updateMesh} \\ 
%	-- & \texttt{moveMesh} \\ 
%	-- & \texttt{computeVorticity} \\ 
%	-- & \texttt{meshPosition}\\
%	-- & \texttt{probeGridParameters} \\ 	
%	-- & \texttt{plotVelocity} \\ 		
%	-- & \texttt{plotPressure} \\ 		
%	-- & \texttt{plotVorticity} \\ 			
%	-- & \texttt{saveClass} \\ 					
%	-- & \texttt{saveVelocity} \\ 		
%	-- & \texttt{savePressure} \\ 		
%	-- & \texttt{saveVorticity} \\ 	
%	-- & \texttt{saveVelocityPlot} \\ 		
%	-- & \texttt{savePressurePlot} \\ 		
%	-- & \texttt{saveVorticityPlot} \\ 	
%\end{tabular}
%%\caption{\texttt{navierStokes} class structure}
%\end{table}

\newpage

\subsection{\texttt{\_\_init\_\_}}
	\paragraph{Description:} Initialize the \texttt{navierStokes} class either using a \texttt{fileName} containing all the necessary parameter for initialization or by explicitly inputing the parameters.\\

	\begin{tabular}{l|lp{7cm}}
		\multicolumn{2}{l}{\textbf{Input Parameters}} & \\ \hline
		\textit{File Name} & \multicolumn{2}{l}{Containing all the parameters to re-initalize the class.} \\ \hline
		\multicolumn{3}{c}{-- or --} \\ \hline
		\multirow{5}{*}{\textit{Parameters}} & Mesh data &: \texttt{mesh, boundaryDomains}\\ \cline{2-3}
		& Geometry position &: \texttt{cmGlobal, thetaLocal} \\ \cline{2-3}
		& Fluid parameters &: \texttt{uMax, nu}\\ \cline{2-3}
		& Solver options &: \texttt{cfl}\\ \cline{2-3}
		& Probe grid parameters &: \texttt{x0, y0, Lx, Ly, hx, hy}\\ \cline{2-3}

	\end{tabular}\\
	\subsubsection*{Descrition of the parameters:}
	
	\begin{tabular}{lp{10cm}}
				\textit{Mesh data} & \\ \hline
				\texttt{mesh} &: the mesh data file.\\ 
				\texttt{boundaryDomains} &: the boundary mesh domain data file.\\ 			
	\end{tabular}\\ 
    \\ \\
	\begin{tabular}{lp{10cm}}
				\textit{Geometry position} & \\ \hline
				\texttt{cmGlobal} &: the $x,y$ position of the geometry in global coordinates.\\ 
				\texttt{thetaGlobal} &: the rotation angle (in $rad$) of the geometry in global coordinate system.\\ 			
	\end{tabular}\\
    \\ \\
	\begin{tabular}{lp{10cm}}
				\textit{Fluid parameters} & \\ \hline
				\texttt{uMax} &: the maximum fluid velocity $U_{max}$. Used to determine the maximum time step size $\Delta t_{max}$.\\ 
				\texttt{nu} &: the fluid kinematic viscosity $\nu$, for incompressible navier-stokes problem.\\ 			
	\end{tabular}\\	
    \\ \\
	\begin{tabular}{lp{10cm}}
				\textit{Solver options} & \\ \hline
				\texttt{cfl} &: the $CFL$ stability parameter. If explicit time marching scheme, $CFL<1$.\\ 		
	\end{tabular}\\	
    \\ \\
	\begin{tabular}{lp{10cm}}
				\textit{Probe grid parameters} & \\ \hline
				\texttt{x0,y0} &: the $x,y$ coordinate of the origin of the probe grid.\\ 
				\texttt{Lx,Ly} &: the $x,y$ size (width and height) of the probing grid.\\ 			
				\texttt{hx,hy} &: the $x,y$ spacing of the probe grid cell.\\ 							
	\end{tabular}\\	

\subsection{\texttt{getCoordinates}}
	\paragraph{Description:} Function to get all the coordinates of the velocity function spaces $\mathbf{V}$. With the returned coordinates, one could calculate the velocity field in the navier-stokes domain. \textit{Note}: The coordinates and just a list of DOF coordinate of the vector function space and is given in the same order as the data that is to be stored.\\
	
	    \begin{tabular}{lp{10cm}}
			\textit{Parameters} & \\ \hline
			\texttt{xCoordinates,yCoordinates} &: the $x,y$ coordinates of the velocity vector function space $\mathbf{V}$. \\		
		\end{tabular} \vspace{5 mm}
	\\		
	\begin{tabular}{lp{10cm}}
		\textbf{Input parameters} &: \texttt{-}\\ 
		\textbf{Assigns} &: \texttt{-}\\ 			
		\textbf{Returns} &: \texttt{xCoordinates,yCoordinates}\\ 					
	\end{tabular}

\subsection{\texttt{setVelocity}}
	\paragraph{Description:} Function to apply the current velocity field.\\
	
	    \begin{tabular}{lp{10cm}}
			\textit{Parameters} & \\ \hline
			\texttt{vFieldNew} &: the \textit{new} velocity at the navier-stokes DOF coordinates of the vector function space $\mathbf{V}$.\\		
		\end{tabular} \vspace{5 mm}
	\\		
	\begin{tabular}{lp{10cm}}
		\textbf{Input parameters} &: \texttt{vFieldNew}\\ 
		\textbf{Assigns} &: \texttt{vField}\\ 			
		\textbf{Returns} &: \texttt{-}\\ 					
	\end{tabular}
	
\subsection{\texttt{setPressure}}
	\paragraph{Description:} Function to apply the current pressure field.\\
	
	    \begin{tabular}{lp{10cm}}
			\textit{Parameters} & \\ \hline
			\texttt{pFieldNew} &: the \textit{new} pressure field at the navier-stokes DOF coordinates of the scalar function space $\mathbf{V}$.\\ 
		\end{tabular} \vspace{5 mm}
	\\		
	\begin{tabular}{lp{10cm}}
		\textbf{Input parameters} &: \texttt{pFieldNew}\\ 
		\textbf{Assigns} &: \texttt{pField}\\ 			
		\textbf{Returns} &: \texttt{-}\\ 					
	\end{tabular}

\subsection{\texttt{getBoundaryCoordinates}}
	\paragraph{Description:} Function to return the boundary DOF coordinates \texttt{xBoundary,yBoundary} of the vector function space $\mathbf{V}$.\\
	
		\begin{tabular}{lp{10cm}}
			\textit{Parameters} & \\ \hline
			\texttt{xBoundary,yBoundary} &: $x,y$ boundary coordinates of the vector function space  $\mathbf{V}$. \\ 
		\end{tabular} \vspace{5 mm}
	\\
	\begin{tabular}{lp{10cm}}
		\textbf{Input parameters} &: \texttt{-}\\ 
		\textbf{Assigns} &: \texttt{-}\\ 			
		\textbf{Returns} &: \texttt{xBoundary,yBoundary}\\ 					
	\end{tabular}	

\subsection{\texttt{evolve}}
	\paragraph{Description}: Function to evolve the Navier-stokes by one step with the $x,y$ velocity boundary condition \texttt{vxBoundary,vyBoundary} at the Navier-stokes finite element mesh boundary \texttt{xBoundary,yBoundary}. The function will calculate the new velocity and the pressure fields. The \textit{new} mesh position is used to update the mesh position, whereas the \textit{current} mesh velocity is used to calculate the modified convective term to take in account of the rigid mesh motion.\\
	
		\begin{tabular}{lp{10cm}}
			\textit{Parameters} & \\ \hline
			\texttt{vxBoundary,vyBoundary} &: $x,y$ velocity at the navier-stokes dof boundary coordinates as described by \texttt{xBoundary,yBoundary}.\\ 
			\texttt{xBoundary,yBoundary} &: $x,y$ boundary coordinates of the vector function space.\\ 
			\texttt{cmGlobalNew,thetaGlobalNew} &: the \textit{new} mesh position and the global mesh rotational angle\\
			\texttt{cmDotGlobal, thetaDotGlobal} &: the \textit{current} mesh velocities (displacement velocity and rotational velocity) in the global reference frame.\\
		\end{tabular} \vspace{5 mm}
	\\	
	\begin{tabular}{lp{10cm}}
		\textbf{Input parameters} &: \texttt{vxBoundary,vyBoundary, cmGlobalNew,thetaGlobalNew, cmDotGlobal, thetaDotGlobal}\\ 
		\textbf{Assigns} &: \texttt{vField, pField}\\ 			
		\textbf{Returns} &: \texttt{-}\\ 					
	\end{tabular}	


	

%
%\subsection{\texttt{moveMesh}}
%	\paragraph{Description:} Function to move the mesh using body displacement and rotational velocities. This function can be then used to calculate the mesh velocity of the current instant.\\
%	
%		\begin{tabular}{lp{10cm}}
%			\textit{Parameters} & \\ \hline
%			\texttt{cmDotGlobal} &: the $x,y$ global mesh coordinate displacement velocity.\\ 
%			\texttt{thetaDotGlobal} &: the polar rotational velocity of the navier-stokes domain w.r.t global coordintes.\\ 			
%			\texttt{vMesh} &: the mesh velocity w.r.t to global $x,y$-axis\\	
%		\end{tabular} \vspace{5 mm}
%	\\
%	\begin{tabular}{lp{10cm}}
%		\textbf{Input parameters} &: \texttt{cmDotGlobal,thetaDotGlobal}\\ 
%		\textbf{Assigns} &: \texttt{xBoundary,yBoundary,vMesh}\\ 			
%		\textbf{Returns} &: \texttt{-}\\ 					
%	\end{tabular}
	
	
\subsection{\texttt{getVorticity}}
	\paragraph{Description:} Function to evaluate the vorticity at probe coordinates defined by the probe mesh.\\
	
		\begin{tabular}{lp{10cm}}
			\textit{Parameters} & \\ \hline
			\texttt{vortProbeGrid} &: the vorticity $\omega$ at the probe grid coordinates \texttt{xProbeGrid, yProbeGrid}.\\ 
		\end{tabular} \vspace{5 mm}
	\\
	\begin{tabular}{lp{10cm}}
		\textbf{Input parameters} &: \texttt{-}\\ 
		\textbf{Assigns} &: \texttt{-}\\ 			
		\textbf{Returns} &: \texttt{wProbeGrid}\\ 					
	\end{tabular}


\subsection{\texttt{getMeshPosition}}

	\paragraph{Description:} Function to return the current mesh position and rotational angle.\\
	
		\begin{tabular}{lp{10cm}}
			\textit{Parameters} & \\ \hline
			\texttt{cmGlobal} &: the $x,y$ position of the mesh in global coordinates.\\ 
			\texttt{thetaGlobal} &: the rotational angle of the mesh w.r.t the global $x$ axis.\\
		\end{tabular} \vspace{5 mm}
	\\
	\begin{tabular}{lp{10cm}}
		\textbf{Input parameters} &: \texttt{-}\\ 
		\textbf{Assigns} &: \texttt{-}\\ 			
		\textbf{Returns} &: \texttt{cmGlobal,thetaGlobal}\\ 					
	\end{tabular}

\subsection{\texttt{getProbeGridParameters}}

	\paragraph{Description:} Function to return the probe grid parameters.\\
	
		\begin{tabular}{lp{10cm}}
			\textit{Parameters} & \\ \hline
			\texttt{x0,y0} &: the global $x,y$ coordinates of the probe mesh origin.\\ 
			\texttt{Lx,Ly} &: the local width and height of the probe mesh.\\
			\texttt{hx,hy} &: the probe spacing of the structure probe mesh.\\			
		\end{tabular} \vspace{5 mm}
	\\
	\begin{tabular}{lp{10cm}}
		\textbf{Input parameters} &: \texttt{-}\\ 
		\textbf{Assigns} &: \texttt{-}\\ 			
		\textbf{Returns} &: \texttt{x0,y0,Lx,Ly,hx,hy}\\ 					
	\end{tabular}

\subsection{plots \ldots}
	\paragraph{Description:} Function to plot and/or save (\textit{optional}) all the results.\\
	
		\begin{tabular}{lp{10cm}}
			\textit{Plot variables} & \\ \hline
			\texttt{plotVelocity} &: the velocity $\mathbf{V}$ of the navier-stokes domain, \texttt{u1}.\\ 
			\texttt{plotPressure} &: the pressure $\mathbf{p}$ of the navier-stokes domain, \texttt{p1}\\ 
			\texttt{plotVorticity} &: the vorticity $\mathbf{\omega}$ of the navier-stokes domain, \texttt{w1}.\\ 
		\end{tabular} \vspace{5 mm}
	\\
	\begin{tabular}{lp{10cm}}
		\textbf{Input parameters} &: \texttt{-}\\ 
		\textbf{Assigns} &: \texttt{-}\\ 			
		\textbf{Returns} &: \texttt{figureHandle} or \texttt{saveFile}\\ 					
	\end{tabular}

\subsection{save datas \ldots}
	\paragraph{Description:} Function to save the navier-stokes data as binaries.\\
	
		\begin{tabular}{lp{10cm}}
			\textit{Save variables} & \\ \hline
			\texttt{save} & : all the data of the \texttt{navierStokes} class, to be used to restart later.\\
			\texttt{saveVelocity} &: the velocity $\mathbf{V}$ of the navier-stokes domain, \texttt{u1}.\\ 
			\texttt{savePressure} &: the pressure $\mathbf{p}$ of the navier-stokes domain, \texttt{p1}\\ 
			\texttt{saveVorticity} &: the vorticity $\mathbf{\omega}$ of the navier-stokes domain, \texttt{w1}.\\ 
		\end{tabular} \vspace{5 mm}
	\\
	\begin{tabular}{lp{10cm}}
		\textbf{Input parameters} &: \texttt{-}\\ 
		\textbf{Assigns} &: \texttt{-}\\ 			
		\textbf{Returns} &: \texttt{.npz}\\ 					
	\end{tabular}

\subsection{\texttt{\_moveMesh}}
	\paragraph{Description:} \textit{Internal} function to update the mesh coordinates using the new global position and rotational angle of the body. The function will be called through \texttt{evolve}.\\
	
		\begin{tabular}{lp{10cm}}
			\textit{Parameters} & \\ \hline
			\texttt{cmGlobal} &: the $x,y$ global coordinates of the body.\\ 
			\texttt{thetaGlobal} &: the polar rotational angle of the navier-stokes domain w.r.t global $x$-coordinate axis.\\ 			
		\end{tabular} \vspace{5 mm}
	\\
	\begin{tabular}{lp{10cm}}
		\textbf{Input parameters} &: \texttt{cmGlobal,thetaGlobal}\\ 
		\textbf{Assigns} &: \texttt{xBoundary,yBoundary}\\ 			
		\textbf{Returns} &: \texttt{-}\\ 					
	\end{tabular}


%\subsection{save plots \ldots}
%	\paragraph{Description:} Function to save the navier-stokes plots are scientific visualization format \texttt{.pvd}.\\
%	
%		\begin{tabular}{lp{10cm}}
%			\textit{Save variables} & \\ \hline
%			\texttt{saveVelocityPlot} &: the velocity $\mathbf{V}$ plot of the navier-stokes domain, \texttt{u1}.\\ 
%			\texttt{savePressurePlot} &: the pressure $\mathbf{p}$ plot of the navier-stokes domain, \texttt{p1}\\ 
%			\texttt{saveVorticityPlot} &: the vorticity $\mathbf{\omega}$ plot of the navier-stokes domain, \texttt{w1}.\\ 
%		\end{tabular} \vspace{5 mm}
%	\\
%	\begin{tabular}{lp{10cm}}
%		\textbf{Input parameters} &: \texttt{-}\\ 
%		\textbf{Assigns} &: \texttt{-}\\ 			
%		\textbf{Returns} &: \texttt{.pvd}\\ 					
%	\end{tabular}