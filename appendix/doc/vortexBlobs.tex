\section{\texttt{vortexBlobs}}
The main structure of the \texttt{vortexBlobs} class. This class contains all the function related to the calculation of the vortex blobs.
\begin{figure}[h]
\centering
\begin{tikzpicture}[%
  grow via three points={one child at (0.5,-0.7) and
  two children at (0.5,-0.7) and (0.5,-1.4)},
  edge from parent path={(\tikzparentnode.south) |- (\tikzchildnode.west)}]
  \node {\texttt{hybrid}}
    child { node {\texttt{vortexMethod}}
    	child {node [selected] {\texttt{vortexBlobs}}}
    	child {node {\texttt{panels}}}  	
    }
    child [missing] {}				
    child [missing] {}				
    child { node {\texttt{navierStokes}}};
    %child { node [selected] {tex}
    %  child { node {generic}}
    %  child { node [optional] {latex}}
    %  child { node {plain}}
    %}
    %child [missing] {}
    %child { node {texdoc}};
\end{tikzpicture}
\end{figure}

\subsection*{Class structure:}
\begin{figure}[h]
\begin{tikzpicture}[%
  grow via three points={one child at (0.5,-0.7) and
  two children at (0.5,-0.7) and (0.5,-1.4)},
  edge from parent path={(\tikzparentnode.south) |- (\tikzchildnode.west)}]
  \node [selected] {\texttt{vortexBlobs}}
    child { node {\texttt{1. \_\_init\_\_}}}
    child { node {\texttt{2. setDeltaTc}}}
    child { node {\texttt{3. setVInf}}}    
    child { node {\texttt{4. setPopulationControl}}}
    child { node {\texttt{5. addBlobs}}}
    child { node {\texttt{6. removeBlobs}}}    
    child { node {\texttt{7. modifyBlobs}}}    
    child { node {\texttt{8. evolve}}}    
    child { node {\texttt{9. remesh}}}    
    child { node {\texttt{10. evaluteVelocity}}}                    
    child { node {\texttt{11. evaluteVorticity}}}                        
    child { node {\texttt{12. plotBlobs}}}                        
    child { node {\texttt{13. plotVelocity}}}                        
    child { node {\texttt{14. plotVorticity}}}                            
    child { node {\texttt{15. save}}}                        
    child { node {\texttt{16. saveBlobs}}}                                        
    child { node {\texttt{17. saveVelocity}}}                                            
    child { node {\texttt{18. saveVorticity}}};
%    child { node [optional] {19. \texttt{\_generateBlobs}}}
%    child { node [optional] {20. \texttt{\_populationControl}}};        
        %child { node [selected] {tex}
    %  child { node {generic}}
    %  child { node [optional] {latex}}
    %  child { node {plain}}
    %}
    %child { node {texdoc}};
\end{tikzpicture}
\end{figure}

%\subsection*{Structure:}
%\begin{table}[h]
%\begin{tabular}{rl}
%	\texttt{vortexBlobs}: &  \\ \hline
%	-- & \texttt{\_\_init\_\_} \\
%	& \\
%	-- & \texttt{set\_deltaTc} \\ 
%	-- & \texttt{set\_popControlParameters} \\ 
%	-- & \texttt{addBlobs} \\ 
%	-- & \texttt{removeBlobs} \\ 
%	-- & \texttt{modifyBlobs} \\ 
%	-- & \texttt{evolve} \\ 
%	-- & \texttt{remesh} \\ 
%	-- & \texttt{evaluteVelocity} \\ 
%	-- & \texttt{evaluteVorticity} \\ 
%	-- & \texttt{plotBlobs} \\	
%	-- & \texttt{plotVelocity} \\
%	-- & \texttt{plotVorticity} \\
%	-- & \texttt{saveClass} \\ 
%	-- & \texttt{saveBlob} \\ 
%	-- & \texttt{saveVelocity} \\ 	
%	-- & \texttt{saveVorticity} \\ 
%	-- & \texttt{saveBlobPlot} \\ 
%	-- & \texttt{saveVelocityPlot} \\ 	
%	-- & \texttt{saveVorticityPlot} \\ 		
%	& \\
%	-- & \texttt{\_generateParticles} \\ 
%	-- & \texttt{\_popControl} \\
%\end{tabular}
%\end{table}

\newpage



\subsection{\texttt{\_\_init\_\_}}
	\paragraph{Description:} Initialize the \texttt{vortexBlobs} class with either the given input parameters or by a reading a \texttt{file} containing all the necessary parameters.\\
	
	\begin{tabular}{l|lp{7cm}}
		\multicolumn{2}{l}{\textbf{Input Parameters}} & \\ \hline
		\textit{File Name} & \multicolumn{2}{l}{Containing all the parameters to re-initalize the class.} \\ \cline{2-3}
		\multicolumn{3}{c}{--- or ---} \\ \cline{2-3}
		\multirow{4}{*}{\textit{Parameters}} & Vorticity Field &: \{\texttt{xBlob, yBlob, gBlob}\} or \{\texttt{wFunction, xBounds, yBounds}\}\\ \cline{2-3}
		& Blob parameters &: \texttt{overlap, h} \\ \cline{2-3}
		& Time Step parameters &: \texttt{deltaTc, nu, stepRedistribution, integrationMethod, computationMethod}\\ \cline{2-3}
		& Population control parameters &: \texttt{stepPopulationControl, gThreshold}\\ \cline{2-3}
	\end{tabular}\\
	
	\subsubsection*{Descriptions of the parameters:}
	\begin{tabular}{p{3.5cm}p{9cm}p{1cm}}
				\textit{Vorticity field} & & \textit{Default}\\ \hline
				\texttt{xBlob,yBlob} &:  the $x,y$ blob coordinates. & - \\
				\texttt{gBlob} &: the circulation $\Gamma_i$ associated to each of the vortex blobs. & - \\
				& & \\
				\multicolumn{2}{c}{\textit{--- or ---}} & \\
				& & \\
				\texttt{wExactFunction} &: the function that returns the exact value of vorticity $\omega$ at any given $x,y$ coordinates. &\\
				& 	\begin{tabular}{lp{10cm}}
						\textbf{Input parameters} &: \texttt{xEval,yEval}\\ 
						\textbf{Assigns} &: \texttt{-}\\ 			
						\textbf{Returns} &: \texttt{wEval}\\ 					
					\end{tabular} & - \\
				\texttt{xBounds, yBounds} &: the $x,y$ bounds of the domain where the particles was originally distributed. & - \\		 
	\end{tabular}\\
	\\ \\
	%	\begin{tabular}{lp{10cm}}
	%		\textbf{Input parameters} &: \texttt{xEval,yEval}\\ 
	%		\textbf{Assigns} &: \texttt{-}\\ 			
	%		\textbf{Returns} &: \texttt{vortEval}\\ 					
	%	\end{tabular}\\
	\begin{tabular}{p{3.5cm}p{9cm}p{1cm}}
				\multicolumn{2}{l}{\textit{Blob parameters}} & \textit{Default} \\ \hline				
				\texttt{overlap} &: the overlap ratio $h/\sigma$. & 1.0\\
				\texttt{h} &: the size of the cell $h$ associated to the blobs. \textit{Note:} Cells are square. & -\\
	\end{tabular}\\
	\\ \\
	\begin{tabular}{p{3.5cm}p{9cm}p{1cm}}
				\multicolumn{2}{l}{\textit{Time step parameters}} & \textit{Default}\\ \hline
				\texttt{deltaTc} &:  the size of the convective time step $\Delta t_c$. & - \\
				\texttt{nu} &: the fluid kinematic viscosity $\nu$, used to calculate the diffusion coefficient $c^2$ and diffusion time step size $\Delta T_d$.& - \\
				\texttt{stepRedistribution} &: the redistribution step frequency. & 1 \\
				\texttt{integrationMethod} &: the time integration method (\texttt{FE}: Forward euler , \texttt{RK4}: $4^{th}$ order Runge-Kutta). & RK4 \\
				\texttt{computationMethod} &: the calculation method to evolve the blobs, (\texttt{Direct}: Direct Method, \texttt{FMM}: Fast-Multipole Method) using (\texttt{CPU}, \texttt{GPU}). & \{FMM, GPU\}.\\
	\end{tabular}\\ 
    \\ \\ 
	\begin{tabular}{p{3.5cm}p{9cm}p{1cm}}
				\multicolumn{2}{l}{\textit{Population control parameters}} & \textit{Default} \\ \hline
				\texttt{stepPopulationControl} &: population control step frequency & 1.\\
				\texttt{gThreshold} &: the tuple with minimum \textbf{and} maximum value of the circulation $\Gamma_{min}$. & - \\
	\end{tabular}\\ 
    \\ \\  
	\begin{tabular}{p{3.5cm}p{9cm}p{1cm}}
				\multicolumn{2}{l}{\textit{Free stream velocity}} & \textit{Default}\\ \hline
				\texttt{vInf} &: The free-stream velocity function, returning the velocity action on the vortex blobs. & -\\		
				&		\begin{tabular}{lp{10cm}}
							\textbf{Input parameters} &: \texttt{t}\\ 
							\textbf{Assigns} &: \texttt{-}\\ 			
							\textbf{Returns} &: \texttt{vx,vy}\\ 					
						\end{tabular} & - \\
				
	\end{tabular}\\


\subsection{\texttt{setDeltaTc}}
	\paragraph{Description:} Function change the convective time step size $\Delta t_c$.\\
	
	 \begin{tabular}{p{3.5cm}p{10cm}p{1cm}}
				\multicolumn{2}{l}{\textit{Parameters}} & \textit{Default} \\ \hline
		   		\texttt{deltaTc} &: the new convection time step size $\Delta t_c$. & - \\
		\end{tabular} \vspace{5 mm}\\
	\\		
	\begin{tabular}{lp{10cm}}
		\textbf{Input parameters} &: \texttt{deltaTcNew}\\
		\textbf{Assigns} &:  \texttt{deltaTc}\\
		\textbf{Returns} &: \texttt{-}\\
	\end{tabular}

\subsection{\texttt{setPopulationControl}}
	\paragraph{Description:} function to modify the population control parameters.\\
	
	 \begin{tabular}{lp{10cm}}
				\textit{Parameters} & \\ \hline
		   		\texttt{gThresholdNew} &: the minimum and maximum circulation of the blobs, $\Gamma_{min}$\\
		   		\texttt{stepPopulationControlNew} &: the step number (frequency) of the population control.\\
		\end{tabular} \vspace{5 mm}\\
	\\		
	\begin{tabular}{lp{10cm}}
		\textbf{Input parameters} &: \texttt{gThresholdNew, stepPopulationControlNew}\\
		\textbf{Assigns} &:  \texttt{gThreshold, stepPopulationControl}\\
		\textbf{Returns} &: \texttt{-}\\
	\end{tabular}

		
\subsection{\texttt{addBlobs}}
	\paragraph{Description:} adds vortex particles by appending to the current set of particles.\\
	
	 \begin{tabular}{lp{10cm}}
				\textit{Parameters} & \\ \hline
		   		\texttt{xBlobNew,yBlobNew,gBlobNew} &: the coordinates and the strength of the new set of particles.\\
		\end{tabular} \vspace{5 mm}\\
	\\		
	\begin{tabular}{lp{10cm}}
		\textbf{Input parameters} &: \texttt{xBlobNew,yBlobNew,gBlobNew}\\
		\textbf{Assigns} &: \texttt{xBlob,yBlob,gBlob}\\
		\textbf{Returns} &: \texttt{-}\\
	\end{tabular}


\subsection{\texttt{removeBlobs}}
	\paragraph{Description:} removes vortex particles from the current set of particles. Using, the particle index, the associated $x,y$ and $\Gamma_i$ will be removed.\\
	
	 \begin{tabular}{lp{10cm}}
				\textit{Parameters} & \\ \hline
		   		\texttt{iBlob} &: the list of blob indices that is to be removed.\\
		\end{tabular} \vspace{5 mm}\\
	\\		
	\begin{tabular}{lp{10cm}}
		\textbf{Input parameters} &: \texttt{iBlob}\\
		\textbf{Assigns} &: \texttt{xBlob,yBlob,gBlob}\\
		\textbf{Returns} &: \texttt{-}\\
	\end{tabular}
	
	
\subsection{\texttt{modifyBlobs}}
	\paragraph{Description:} Replace the vortex particle strengths with the new strength.\\
	
	 \begin{tabular}{lp{10cm}}
				\textit{Parameters} & \\ \hline
		   		\texttt{iBlob} &: the list of blob indices that is to be modified.\\
		   		\texttt{gBlobNew} &: the new strength of the blobs.\\		   		
		\end{tabular} \vspace{5 mm}\\
	\\		
	\begin{tabular}{lp{10cm}}
		\textbf{Input parameters} &: \texttt{iBlob,gBlobNew}\\
		\textbf{Assigns} &: \texttt{gBlob}\\
		\textbf{Returns} &: \texttt{-}\\
	\end{tabular}	

\subsection{\texttt{evolve}}
	\paragraph{Description:} Evolves the vortex blobs according to the \texttt{\_\_init\_\_} definition. The \texttt{evolve} function, knows when (has a counter) to redistribute and perform population control. Depending on the diffusion time step $\Delta t_d$, the evolve function will also perform the diffusion process (modified interpolation).\\
	
	 \begin{tabular}{lp{10cm}}
				\textit{Parameters} & \\ \hline
		   		\texttt{xBlobNew,yBlobNew,gBlobNew} &: the new set of particle after the evolution process.\\
		\end{tabular} \vspace{5 mm}\\
	\\		
	\begin{tabular}{lp{10cm}}
		\textbf{Input parameters} &: \texttt{-}\\
		\textbf{Assigns} &: \texttt{xBlob,yBlob,gBlob}\\	
		\textbf{Returns} &: \texttt{-}\\
	\end{tabular}			


\subsection{\texttt{remesh}}
	\paragraph{Description:}  Function to remesh the particles on to the remeshing grid. When $c=0$, the remeshing will be done without diffusion. If $c>0$, the modified interpolation will perform the diffusion.
	\\
	\\		
	\begin{tabular}{lp{10cm}}
		\textbf{Input parameters} &: \texttt{c}\\ 
		\textbf{Assigns} &: \texttt{xBlob,yBlob,wBlob}\\ 			
		\textbf{Returns} &: \texttt{-}\\ 					
	\end{tabular}


\subsection{\texttt{evaluateVelocity}}
	\paragraph{Description:} Function to evaluate the total induced velocity due to the blobs, and the external velocity at a given target locations.\\
	
	    \begin{tabular}{lp{10cm}}
			\textit{Parameters} & \\ \hline
			 \texttt{xTarget,yTarget} &: the $x,y$ coordinate of the target location, where the total velocity is to be evaluated.\\
			\texttt{vxTarget,vyTarget} &: the $x,y$ induced velocity at the target points in global coordinate system.\\
		\end{tabular} \vspace{5 mm}
	\\		
	\begin{tabular}{lp{10cm}}
		\textbf{Input parameters} &: \texttt{xTarget,yTarget}\\ 
		\textbf{Assigns} &: \texttt{-}\\ 			
		\textbf{Returns} &: \texttt{vxTarget,vyTarget}\\ 					
	\end{tabular}	

\subsection{\texttt{evaluateVorticity}}
	\paragraph{Description:} Function to evaluate the total induced vorticity due to the blobs, and the external velocity at a given target coordinates.\\
	
	    \begin{tabular}{lp{10cm}}
			\textit{Parameters} & \\ \hline
			 \texttt{xTarget,yTarget} &: the $x,y$ coordinate of the target location, where the total velocity is to be evaluated.\\
			\texttt{wTarget} &: the $x,y$ induced vorticity at the target points in global coordinate system.\\
		\end{tabular} \vspace{5 mm}
	\\		
	\begin{tabular}{lp{10cm}}
		\textbf{Input parameters} &: \texttt{xTarget,yTarget}\\ 
		\textbf{Assigns} &: \texttt{-}\\ 			
		\textbf{Returns} &: \texttt{wTarget}\\ 					
	\end{tabular}


		
\subsection{plots \ldots}
	\paragraph{Description:} functions to plot and/or save all the results in a given region. The data should be store for scientific visualization (paraview format)\\
	
		\begin{tabular}{lp{10cm}}
			\textit{Plot variables:} & \\ \hline
			\texttt{plotBlob} &: plot the coordinates and the circulation of the blobs.\\
			\texttt{plotVelocity} &: plot the velocity field.\\ 
			\texttt{plotVorticity} &: plot the vorticity field.\\ 
		\end{tabular} \vspace{5 mm}
		
		\begin{tabular}{lp{10cm}}
			\textit{Parameters:} & \\ \hline
			\texttt{xBounds,yBounds} &: $x,y$ bounds of the grid, where the data is to be evaluated.\\ 
			\texttt{nGrid} &: $x,y$ number of grid points.\\
		\end{tabular} \vspace{5 mm}\\
	\\
	\begin{tabular}{lp{10cm}}
		\textbf{Input parameters} &: \texttt{xBounds,yBounds,nGrid}\\
		\textbf{Assigns} &: \texttt{-}\\ 			
		\textbf{Returns} &: \texttt{figureHandle} or \texttt{.pvd}\\ 					
	\end{tabular}	

\subsection{save data \ldots}
	\paragraph{Description:} functions to save the data. The data file will be in compressed, binary format to store efficiently.\\

		\begin{tabular}{lp{10cm}}
			\textit{Save variables:} & \\ \hline
			\texttt{save} &: all the data of the \texttt{vortexBlob} class is saved. This can be used later to restart the problem, i.e the parameter to init the problem.\\
			\texttt{saveBlobs} &: the function to save the blob data at the current time instant. List of numpy array.\\ 			
			\texttt{saveVelocity} &: save the velocity field of a given region or a given set of points.\\ 
			\texttt{saveVorticity} &: save the vorticity field of the given region or the given set of points.\\ 
		\end{tabular} \vspace{5 mm}
	
		\begin{tabular}{lp{10cm}}
			\textit{Parameters:} & \\ \hline
			\texttt{xBounds,yBounds} &: $x,y$ bounds of the grid, where the data is to be evaluated and saved.\\ 
			\texttt{nGrid} &: $x,y$ number of grid points.\\
			\texttt{xEval,yEval} &: $x,y$ coordinates of the location where the data is to be evaluated and saved.\\ 
		\end{tabular} \vspace{5 mm}\\
	\\
	\begin{tabular}{lp{10cm}}
		\textbf{Input parameters} &: \texttt{xBounds,yBounds,hGrid} or \texttt{xEval,yEval}\\ 
		\textbf{Assigns} &: \texttt{-}\\ 			
		\textbf{Returns} &: \texttt{.npz, .bin or similar}\\ 					
	\end{tabular}



%\subsection{save plots \ldots}
%	\paragraph{Description:} Function to save the plots as scientific visualization format \texttt{.pvd}.\\
%	
%		\begin{tabular}{lp{10cm}}
%			\textit{Save variables:} & \\ \hline
%			\texttt{saveBlobPlot} &: save the particle position and strengths as glyphs.\\
%			\texttt{saveVelocityPlot} &: save the velocity plot of a given region.\\ 
%			\texttt{saveVorticityPlot} &: save the vorticity of a given region.\\ 
%		\end{tabular} \vspace{5 mm}
%		
%		\begin{tabular}{lp{10cm}}
%			\textit{Parameters:} & \\ \hline
%			\texttt{xBounds,yBounds} &: $x,y$ bounds of the grid, where the data is to be evaluated and saved.\\ 
%			\texttt{hGrid} &: $x,y$ spacing of the evaluation grid.\\
%		\end{tabular} \vspace{5 mm}\\
%	\\
%	\begin{tabular}{lp{10cm}}
%		\textbf{Input parameters} &: \texttt{xBounds,yBounds,hGrid}\\
%		\textbf{Assigns} &: \texttt{-}\\ 			
%		\textbf{Returns} &: \texttt{.pvd}\\ 					
%	\end{tabular}
	


%\subsection{\texttt{\_generateParticles}}
%	\paragraph{Description:} \textit{Internal} function to generate/initialize the particles.\\
%	\\
%	\\
%		\begin{tabular}{lp{10cm}}
%			\textbf{Input parameters} &: \texttt{wExactFunction, xBounds, yBounds}\\ 
%			\textbf{Assigns} &: \texttt{xBlob,yBlob,wBlob}\\ 			
%			\textbf{Returns} &: \texttt{-}\\ 					
%		\end{tabular}	
%
%
%\subsection{\texttt{\_popControl}}
%	\paragraph{Description:} \textit{Internal} function to perform population control on the current set of particles.\\
%	\\
%	\\	
%		\begin{tabular}{lp{10cm}}
%			\textbf{Input parameters} &: \texttt{-}\\ 
%			\textbf{Assigns} &: \texttt{xBlob,yBlob,wBlob}\\ 			
%			\textbf{Returns} &: \texttt{-}\\ 					
%		\end{tabular}	